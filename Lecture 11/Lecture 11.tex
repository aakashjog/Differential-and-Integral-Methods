\documentclass[fleqn]{article}
\usepackage{amsmath, amssymb, amsthm, esdiff}
\usepackage[table]{xcolor}
\usepackage{commath}
\usepackage{gensymb}
\usepackage{hyperref}
\usepackage{tikz, pgfplots}
\usetikzlibrary{calc}
\usepackage{datetime}
\usepackage{setspace}
\usepackage{ulem}

\usepackage{enumerate, enumitem}

\setcounter{secnumdepth}{4}

\newcommand\numberthis{\addtocounter{equation}{1}\tag{\theequation}}

\theoremstyle{definition}
\newtheorem{example}{Example}
\newtheorem{definition}{Definition}

\theoremstyle{theorem}
\newtheorem{theorem}{Theorem}

\theoremstyle{remark}
\newtheorem{remark}{Remark}
\newtheorem{case}{Case}

\newenvironment{solution}
{\begin{proof}[Solution]\let\qed\relax}
	{\end{proof}}

%opening
\title{Lecture 11}
\author{Aakash Jog}
\date{\formatdate{30}{11}{2014}}

\begin{document}
	
\maketitle
%\setlength{\mathindent}{0pt}

\tableofcontents

\newpage

\section{Indefinite Integrals}

\begin{definition}[Anti-derivative of a function]
	A function $F(x)$ is called an anti-derivative of $f(x)$ in the domain $D$, if
	\begin{equation*}
	F'(x) = f(x) \quad;\forall x \in D
	\end{equation*}
	If $F(x)$ is an anti-derivative of $f(x)$, then $F(x) + c$ is also an anti-derivative of $f(x)$.
\end{definition}

\begin{definition}[Indefinite integral of a function]
	Let $F(x)$ be an anti-derivative of $f(x)$. The set of all anti-derivatives of $f(x)$, i.e., $F(x) + c$ is called the indefinite integral of $f(x)$, and is denoted as 
	\begin{equation*}
		\int f(x) \dif x = F(x)
	\end{equation*}
\end{definition}

\begin{align}
	\int x^a \dif x &= 
	\begin{cases}
		\dfrac{x^{a + 1}}{a + 1} + c\quad; a \neq -1 \\
		\ln |x| + c \quad; a = -1 \\
	\end{cases}\\
	\int \cos x \dif x &= \sin x + c\\
	\int \sin x \dif x &= -\cos x + c\\
	\int \dfrac{\dif x}{1 + x^2} &= \arctan x + c\\
	\int \dfrac{f'(x)}{f(x)} \dif x &= \ln |f(x)| + c
\end{align}

\subsection{Integration by Parts}

\begin{align*}
	\int u v' \dif x &= u v - \int u' v \dif x\\
	\int u \dif v &= u v - \int v \dif u\\
	\int u v \dif x &= u \int v \dif x - \int u' \left(\int v \dif x\right) \dif x
\end{align*}

\subsection{Integration of Trigonometric Functions}

\begin{example}
	\begin{align*}
		\int \sin(3x) \cos (7x) \dif x &= \dfrac{1}{2} \int (\sin (10x) - \sin (4x)) \dif x\\
		&= \dfrac{1}{2} \left(- \dfrac{\cos (10x)}{10} + \dfrac{\cos(4x)}{4}\right) + c
	\end{align*}
\end{example}

\begin{example}
	\begin{align*}
		\int \sin^{2k+1} x \cos^m x \dif x &= \int \sin^2k x \cos^m x \sin x \dif x\\
		&= \int (1 - \cos^2 x)^k \cos^m x \sin x \dif x
		\intertext{Let $t = \cos x$. Therefore, $\dif t = -\sin x \dif x$.}
		\therefore \int (1 - \cos^2 x)^k \cos^m x \sin x \dif x &= \int (1 - t^2)^k t^m \dif t
	\end{align*}
\end{example}

\subsection{Special Trigonometric Substitutions}
\doublespacing
\rowcolors{1}{white}{white!90!black}
\begin{tabular}{|c|c|c|}
	\hline
	$\sqrt{a^2 - x^2}$ & $x = a \sin \theta \quad; -\dfrac{\pi}{2} \leq \theta \leq \dfrac{\pi}{2}$ & $1 - \sin^2 \theta = \cos^2 \theta$\\
	\hline
	$\sqrt{a^2 + x^2}$ & $x = a \tan \theta \quad; -\dfrac{\pi}{2} < \theta < \dfrac{\pi}{2}$ & $1 + \tan^2 \theta = \sec \theta$\\
	\hline
	$\sqrt{x^2 - a^2}$ & $x = \sec \theta \quad; 0 \leq \theta \leq \dfrac{\pi}{2}$ & $\sec^2 \theta - 1 = \tan^2 \theta$\\
	\hline
\end{tabular}
\singlespacing

\begin{example}
	Find
	\begin{equation*}
		\int \dfrac{\sqrt{9 - x^2}}{x^2} \dif x
	\end{equation*}
\end{example}

\begin{solution}
	\begin{align*}
	\intertext{Let $x = 3 \sin \theta$. Therefore, $\dif x = 3 \cos \theta \dif \theta$}
	\int \dfrac{\sqrt{9 - x^2}}{x^2} \dif x &= \int \dfrac{\sqrt{9 - 9 \sin^2 \theta}}{9 \sin^2 \theta} 3 \cos \theta \dif \theta\\
	&= \int \dfrac{\cos^2 \theta}{\sin^2 \theta} \dif \theta\\
	&= \left(\dfrac{1}{\sin^2 \theta} - 1\right) \dif \theta\\
	&= -\cot \theta - \theta + c\\
	&= - \dfrac{9 - x^2}{x} - \arcsin \dfrac{x}{3} + c
	\end{align*}
\end{solution}

\section{Integration of Rational Functions}

\begin{definition}[Simple rational function]
	A rational function $\dfrac{P(x)}{Q(x)}$ is called simple if the degree of the polynomial $P(x)$ is less than the degree of the polynomial $Q(x)$.
\end{definition}

\begin{remark}
	If a rational function $\dfrac{P(x)}{Q(x)}$ is not simple, then it can be written as a sum of a polynomial $M(x)$ and a simple rational function $\dfrac{R(x)}{Q(x)}$.
	\begin{equation*}
		\dfrac{P(x)}{Q(x)} = M(x) + \dfrac{R(x)}{Q(x)}
	\end{equation*}
\end{remark}

\end{document}
