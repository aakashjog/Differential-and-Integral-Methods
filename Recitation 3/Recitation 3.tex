\documentclass[fleqn]{article}
\setcounter{secnumdepth}{4}
\usepackage{amsmath, amssymb, esdiff}
\usepackage{datetime}
\usepackage{ulem}
\usepackage{enumerate}
\newcommand\numberthis{\addtocounter{equation}{1}\tag{\theequation}}


%opening
\title{Recitation 3}
\author{}
\date{\formatdate{12}{11}{2014}}

\begin{document}

\maketitle
%\setlength{\mathindent}{0pt}

\tableofcontents

\newpage
\section{Arithmetic of Limits}

\subsection{Theorem: If $\lim\limits_{x \rightarrow x_0} f(x) = a$ and $\lim\limits_{x \rightarrow x_0} g(x) = b$ , then, $\lim\limits_{x \rightarrow x_0} (f(x) \pm g(x)) = a \pm b$ , $\lim\limits_{x \rightarrow x_0} (f(x) \cdot g(x)) = a \cdot b$ , $\lim\limits_{x \rightarrow x_0} \dfrac{f(x)}{g(x)} = \dfrac{a}{b}$}

\subsection{Examples}

\subsubsection{Example 1}

\begin{align*}
	\lim\limits_{x \rightarrow \infty} \dfrac{x^2 - 4x + 2}{x^2 + 4} &= \lim\limits_{x \rightarrow \infty} \dfrac{\dfrac{x^2 - 4x +2}{x^2}}{\dfrac{x^2 + 4}{x^2}} \\
	&= \lim\limits_{x \rightarrow \infty} \dfrac{1 - \dfrac{4}{x} + \dfrac{2}{x^2}}{1 + \dfrac{4}{x^2}} \\
	&= \dfrac{1}{1} \\
	&= 1
\end{align*}

\subsubsection{Example 2}

\begin{align*}
	\lim\limits_{x \rightarrow 1} \dfrac{\sqrt[3]{x} - 1}{\sqrt[4]{x} - 1}
\end{align*}
Let $x = t^12$
\begin{align*}
	\therefore \lim\limits_{x \rightarrow 1} \dfrac{\sqrt[3]{x} - 1}{\sqrt[4]{x} - 1} &= \lim\limits_{t \rightarrow 1} \dfrac{t^4 - 1}{t^3 - 1} \\
	&= \lim\limits_{t \rightarrow 1} \dfrac{(t-1)(t^3 + t^2 + t + 1)}{(t-1)(t^2 + t + 1)} \\
	&= \lim\limits_{t \rightarrow 1} \dfrac{t^3 + t^2 + t + 1}{t^2 + t + 1} \\
	&= \dfrac{4}{3}
\end{align*}

\subsection{Infinite Arithmetic}

\begin{align*}
	\dfrac{``\infty"}{``a"} &= 
	\begin{cases}
		+ \infty ; a > 0 \\
		- \infty ; a < 0 \\
	\end{cases} \\
	\dfrac{``a"}{``\infty"} &= 0
\end{align*}

\section{Useful Limits}

If $\lim\limits_{x \rightarrow x_0} g(x) = 0$, 
\begin{equation*}
	\lim\limits_{x \rightarrow x_0} (1 + g(x))^{\frac{1}{g(x)}} = e
\end{equation*}

\begin{align*}	
	\lim\limits_{x \rightarrow \infty} \left(1 + \dfrac{1}{x}\right)^x &= e \\
	\lim\limits_{x \rightarrow 0} \left(1 + x\right)^{\frac{1}{x}} &= e \\
	\lim\limits_{x \rightarrow \infty} \left(1 + \dfrac{a}{x}\right)^x &= e^a \\
	\lim\limits_{x \rightarrow 0} \dfrac{\sin x}{x} &= 1
\end{align*}

\section{Sandwich Theorem}

If, in a punctured neighbourhood of $x_0$ we have 
\begin{equation*}
	g(x) \leq h(x) \leq f(x)
\end{equation*} and \begin{equation*}
\lim\limits_{x \rightarrow x_0} f(x) = \lim\limits_{x \rightarrow x_0} g(x) = l
\end{equation*}
then
\begin{equation*}
\lim\limits_{x \rightarrow x_0} h(x) = l
\end{equation*} 

\section{Continuity}

A function $f$ is continuous at $x_0$ if $\lim\limits_{x \rightarrow x_0} f(x) = f(x_0)$.

\subsection{Types of Discontinuity}

Let $f(x)$ be defined in a neighbourhood of $x_0$. 

\subsubsection{Removable Discontinuity}

\begin{equation*}
	\lim\limits_{x \rightarrow x_0} f(x) \neq f(x_0)
\end{equation*}

\subsubsection{Discontinuity of First Kind}

\begin{equation*}
	\lim\limits_{x \rightarrow x_{0^+}} f(x) \neq \lim\limits_{x \rightarrow x_{0^-}} f(x)
\end{equation*}

\subsubsection{Discontinuity of Second Kind}

Atleast one of the one-sided limits of $f(x)$ at $x_0$ does not exist. Note that the limits are defined as finite numbers only. 

\end{document}

