\documentclass[]{article}
\usepackage{amsmath, amssymb}
\usepackage{datetime}
\usepackage{ulem} % normally replaces italics with underlining in text emphasized by \emph

%opening
\title{Lecture 1}
\author{}
\date{\formatdate{28}{10}{2014}}

\begin{document}

\maketitle

\tableofcontents

\newpage
\section{General Information}

\textbf{Dr. Yakov Yakubov}

\subsection{Office Hours}

Monday\\
16:15 - 17:15\\
Room 233, MathBuilding\\
Tel: 03-6405357\\

\newpage
\section{Functions}

\subsection{Notation}

\begin{eqnarray}
\mathbb{N} &=& \text{Set of all natural numbers}\\
\mathbb{Z} &=& \text{Set of all integers}\\
\mathbb{N} &=& \{\dfrac{m}{n} : m \in \mathbb{Z}, n \in \mathbb{N}\}
\end{eqnarray}

\subsection{Definitions}

\subsubsection{Domain, Range and Variables}

Let $D$ and $E$ be two sets of real numbers. A function $f$ from $D$ into $E$ is a well defined law which, to each $x \in D$ corresponds to a unique number $y \in E$. The set $D$ is called the \emph{domain} of $f$ and the set $E$ is called the \emph{range} of $f$. \\
Denote $f : D \rightarrow E$ or $y = f(x)$. \\
The variable $x$ is called \emph{independent variable} and the variable $y$ is called \emph{dependent variable}. \\
The variable $x$ is also called the \emph{origin} of $y$ and $y$ is also called the \emph{image} of $x$. \\

\subsubsection{Image of a function}

Given $f : D \rightarrow E$. Then the image of $f$ is a set of all $y \in E$ s.t. $\exists x \in D, y = f(x) : I(f) = \{ y \in E : \exists x \in D, y = f(x)\}$ \\

\subsubsection{Existence domain}

The biggest possible domain of a function $f$ is called the \emph{existence domain} of $f$. \\

\subsubsection{Graph}

A set of point $\{(x, f(x)) : x \in D\}$ in the plane $\mathbb{R}^2$ is called a \emph{graph} of a function $y = f(x)$. \\

\subsubsection{Even function}

If $f(-x) = f(x) ; (x, -x \in D)$ then, $f$ is called an \emph{even function}. \\
Each even function is symmeteric about the $y$-axis. \\

\subsubsection{Odd function}

If $f(-x) = -f(x) ; (x, -x \in D)$ then, $f$ is called an \emph{odd function}. \\
Each odd function is symmeteric about the origin. \\

\subsubsection{Periodical function}

A function $y = f(x)$ which is defined on $D$ is called \emph{periodical} if $\exists T \neq 0$ which is called a \emph{period} of $f$ s.t. $\forall x \in D \Rightarrow x + T \in D$ and $f(x+T) = f(x)$. \\
The smallest such $T > 0$ (if it exists) is called the \emph{minimal period}. \\

\subsubsection{Shifting with respect to $y$-axis}
$f(x+a)$ is the graph of $f(x)$, shifted by $a$, in the direction of the $x$-axis, opposite to the sign of $a$. \\

\subsubsection{Shifting with respect to $x$-axis}
$f(x) + a$ is the graph of $f(x)$, shifted by $a$, in the direction of the $y$-axis, according to the sign of $a$. \\

\subsubsection{Monotonic function}

A function $y = f(x)$ is called \emph{monotonic increasing} (\emph{strongly increasing}) in $D$ if $\forall x_1, x_2 \in D, x_1 < x_2 \Rightarrow f(x_1) \leq f(x_2) (f(x_1) < f(x_2))$. \\
A function $y = f(x)$ is called \emph{monotonic decreasing} (\emph{strongly increasing}) in $D$ if $\forall x_1, x_2 \in D, x_1 > x_2 \Rightarrow f(x_1) \geq f(x_2) (f(x_1) > f(x_2))$. \\

\subsubsection{One-to-one function}

A function $f : D(f) \rightarrow E$ is called \emph{one-to-one} if $\forall y \in I(f) \Rightarrow \exists ! x \in D(f)$ s.t. $y = f(x)$. \\
Equivalently, $\forall x_1, x_2 \in D(f)$, if $f(x_1) = f(x_2) \Rightarrow x_1 = x_2$.\\

\end{document}
