\documentclass[fleqn]{article}
\usepackage{amsmath, amssymb, esdiff}
\usepackage{gensymb}
\usepackage{tikz, pgfplots}
\usepackage{datetime}
\usepackage{ulem}
\usepackage{enumerate}
\setcounter{secnumdepth}{4}
\newcommand\numberthis{\addtocounter{equation}{1}\tag{\theequation}}


%opening
\title{Lecture 2}
\author{}
\date{\formatdate{30}{10}{2014}}

\begin{document}
	
\maketitle
\setlength{\mathindent}{0pt}

\tableofcontents

\newpage
\section{Functions}

\subsection{Continuity}

If $f : D(f) \rightarrow I(f)$ is one-to-one and onto. Then, we can define $g : I(f) \rightarrow D(f)$, which is one-to-one and onto, by $g(y) = x$, where $y = f(x)$. Therefore, $g(f(x)) = x$. $g$ is called the \emph{inverse function} of $f$. \\
The inverse function is denoted as $g = f^{-1}$(Note: $f^{-1} \neq \dfrac{1}{f}$)
\begin{align}
	D(f) = I(f^{-1}) \label{Image of inverse}\\
	I(f) = D(f^{-1}) \label{Domain of inverse}
\end{align}
The graphs of a $f$ and $f^{-1}$ are symmeteric about $y=x$.

\subsection{Elementary Operations between Functions}

\subsubsection{$h = f \pm g$}

\begin{align}
	h(x) &= f(x) \pm g(x) \label{sum of functions}\\
	D(h) &= D(f) \cap D(g) \label{domain of sum of functions}
\end{align}

\subsubsection{$h = kf$}

\begin{align}
	h(x) &= k f(x) \label{product of constant and function}\\
	D(h) &= D(f)  \label{domain of product of constant and function}
\end{align}

\subsubsection{$h = f g$}

\begin{align}
	h(x) &= f(x) g(x) \label{product of functions}\\
	D(h) &= D(f) \cap D(g) \label{domain of product of functions}
\end{align}

\subsubsection{$h = \dfrac{f}{g}$}

\begin{align}
h(x) &= \dfrac{f(x)}{g(x)} \label{division of functions}\\
D(h) &= \{x \in D(f) \cap D(g) : g(x) \neq 0\} \label{domain of division of functions}
\end{align}

\subsection{Composite Functions}

Let $f : D(f) \rightarrow E$ and $g : D(g) \rightarrow F$ be two functions. A \emph{composition} of $f$ with $g$ is a function $h : D(h) \rightarrow F$ where $h(x) = g(f(x))$. It is denoted as $g \circ f$\\
\begin{equation}
	D(h) = \{x \in D(f) : f(x) \in D(g)\} \label{domain of composite functions} 
\end{equation}

\subsection{Elementary Functions}

\subsubsection{Polynomial}

\begin{align}
	y = f(x) &= a_0 + a_1 x + \dots + a_n x^n ; a_0, \dots, a_n \in \mathbb{R} \label{polynomial} \\
	D(f) &= \mathbb{R} \label{domain of polynomial}
\end{align}

\begin{enumerate}
	\item	If $n = 0, y = f(x) = a_0$ represents a constant function.
	\item	If $n = 1, y = f(x) = a_0 + a_1 x$ represents a straight line in the $X-Y$ plane.
	\item	If $n = 2, y = f(x) = a_0 + a_1 x + a_2 x^2$ represents a parabola in the $X-Y$ plane.
\end{enumerate}

\subsubsection{Power Function}

\begin{align}
	y = f(x) = x^a ; a\in R \label{power function} \\
	D(f) \text{ depends on } a \label{domain of power function}
\end{align}

\subsubsection{Exponential Function}

\begin{align}
	y = f(x) &= a^x ; a > 0, a \neq 1 \label{exponential function} \\
	D(f) &= \mathbb{R} \label{domain of exponential function} \\
	I(f) &= (0, \infty) \label{image of exponential function}
\end{align}

\subsubsection{Logarithmic Function}

\begin{align}
	y = f(x) &= \log_{a} {x} \label{logarithmic function} \\
	D(f) &= (0,\infty) \label{domain of logarithmic function} \\
	I(f) &= \mathbb{R} \label{image of logarithmic function}
\end{align}

\subsubsection{Trigonometeric Functions}

\begin{align}
	y = f(x) &= \sin x \\
	y = f(x) &= \cos x \\
	y = f(x) = \tan x &= \dfrac{\sin x}{\cos x} \\
	y = f(x)  = \cot x &= \dfrac{1}{\tan x} = \dfrac{\cos x}{\sin x} \\
	y = f(x) = \csc x &= \dfrac{1}{\sin x} \\
	y = f(x) = \sec x &= \dfrac{1}{\cos x} 
\end{align}

\subsubsection{Inverse Trigonometeric Functions}

\begin{align}
	y = f^{-1}(x) &= \sin^{-1} x = \arcsin x \\
	D(\arcsin x) &= [-1,1] \\
	I(\arcsin x) &= [-\dfrac{\pi}{2}, \dfrac{\pi}{2}]
\end{align}

\begin{align}
	y = f^{-1}(x) &= \cos^{-1} x = \arccos x \\
	D(\arccos x) &= [-1,1] \\
	I(\arccos x) &= [0,\pi]
\end{align}

\begin{align}
y = f^{-1}(x) &= \tan^{-1} x = \arctan x \\
D(\arctan x) &= \mathbb{R} \\
I(\arctan x) &= [-\dfrac{\pi}{2}, \dfrac{\pi}{2}]
\end{align}

\subsubsection{Hyperbolic Functions}

\begin{align}
	\sinh x &\dot{=} \dfrac{e^x - e^{-x}}{2} \\
	D(\sinh x) &= \mathbb{R} \\
	I(\sinh x) &= \mathbb{R}
\end{align}

\begin{align}
	\cosh x &\dot{=} \dfrac{e^x + e^{-x}}{2} \\
	D(\cosh x) &= \mathbb{R} \\
	I(\cosh x) &= [1, \infty)
\end{align}

\begin{align}
	\tanh x &\dot{=} \dfrac{\sinh x}{\cosh x} = \dfrac{e^x - e^{-x}}{e^x + e^{-x}} \\
	D(\tanh x) &= \mathbb{R} \\
	I(\tanh x) &= (-1, 1)
\end{align}

\paragraph{Identities of Hyperbolic Functions \\}

\begin{align}
	\sinh (2x) &= 2 \sinh x \cosh x \\
	\cosh ^2 x + \sinh ^2 x &= \cosh (2x) \\
	\cosh ^2 x - \sinh ^2 x &= 1 \\
	\dfrac{\cosh (2x) - 1}{2} &= \sinh ^2 x \\
	\dfrac{\cosh (2x) + 1}{2} &= \cosh ^2 x 
\end{align}

\subsubsection{Absolute Value}

\begin{align}
	y = f(x) = 
	\begin{cases}
		x ; x > 0 \\
		0 ; x = 0 \\
		-x ; x < 0 \\
	\end{cases}
\end{align}

\subsubsection{Floor Function}

\begin{align}
	y = f(x) = \lfloor x \rfloor = \text{the largest integer less than or equal to }x
\end{align}

\end{document}
