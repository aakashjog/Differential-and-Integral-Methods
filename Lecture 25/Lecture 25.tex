\documentclass[fleqn, a4paper, 12pt]{article}
\usepackage{amsmath, amssymb, amsthm, esdiff}
\usepackage[table]{xcolor}
\usepackage{todonotes, marginnote}
\usepackage{commath}
\usepackage{gensymb}
\usepackage{hyperref}
\usepackage{tikz, pgfplots}
\usetikzlibrary{calc}
\usepackage{datetime}
\usepackage{setspace}
\usepackage{ulem}
\usepackage{xfrac}
\usepackage{siunitx}

\usepackage{enumerate, enumitem}

\setcounter{secnumdepth}{4}

\newcommand\numberthis{\addtocounter{equation}{1}\tag{\theequation}}

\newcommand{\curl}{\mathrm{curl\,}}

\newcommand{\divergence}{\mathrm{div\,}}

\theoremstyle{definition}
\newtheorem{example}{Example}
\newtheorem{definition}{Definition}

\theoremstyle{theorem}
\newtheorem{theorem}{Theorem}
\newtheorem{corollary}{Corollary}

\theoremstyle{remark}
\newtheorem{remark}{Remark}
\newtheorem{case}{Case}

\newenvironment{solution}
{\begin{proof}[Solution]\let\qed\relax}
	{\end{proof}}

\makeatletter
\@addtoreset{corollary}{theorem} %resets corollary numbers after a theorem
\makeatother

%opening
\title{Lecture 25}
\author{Aakash Jog}
\date{\formatdate{22}{1}{2015}}

\begin{document}
	
\maketitle
%\setlength{\mathindent}{0pt}

\tableofcontents

\newpage

\section{Vector Form of Green's Theorem}

\begin{theorem}[Green's Theorem]
	\begin{equation*}
		\int\limits_{C} \overline{F} \cdot \hat{n} \dif s = \iint\limits_{D} \divergence \overline{F} \dif A
	\end{equation*}
	\label{Green's Theorem}
\end{theorem}

\section{Gauss Theorem}

\begin{theorem}[Gauss Theorem]
	Let $E$ be a solid bounded by a surface $S$ with positive orientation. Let $\overline{F}(x,y,z) = \left( P(x,y,z), Q(x,y,z), R(x,y,z) \right)$ be a vector field in $\mathbb{R}^3$, s.t. there exist continuous first order partial derivatives of $P$, $Q$, $R$ in some open domain which contain $E$. Then
	\begin{equation*}
		\iint\limits_{S} \overline{F} \cdot \hat{n} \dif S = \iiint\limits_{E} \divergence \overline{F} \dif V
	\end{equation*}
	\label{Gauss Theorem}
\end{theorem}

\begin{remark}
	\nameref{Gauss Theorem} is an analogy of \nameref{Green's Theorem}
\end{remark}

\begin{example}
	Calculate the flux of $\overline{F} = (x y, x e^z, 2 + z)$ through the surface $S$ which is a boundary of a solid $E$ bounded by two paraboloids
	\begin{align*}
		z &= 12 - 2x^2 - 2y^2\\
		z &= x^2 + y^2
	\end{align*}
\end{example}

\begin{solution}
	Let
	\begin{align*}
		g_1(x,y) &= 12 - 2x^2 - 2y^2\\
		g_2(x,y) &= x^2 + y^2
	\end{align*}
	The intersection of $g_1(x,y)$ and $g_2(x,y)$ is $x^2 + y^2 = 4$.
	\begin{align*}
		\iint\limits_{S} \overline{F} \cdot \hat{n} \dif S &= \iiint\limits_{E} \divergence \overline{F} \dif V\\
		&= \iiint\limits_{E} (y + 0 + 1) \dif V\\
		&= \int\limits_{-2}^{2} \int\limits_{-\sqrt{4 - x^2}}^{\sqrt{4 - x^2}} \int\limits_{x^2 + y^2}^{12 - 2x^2 - 2y^2} (y + 1) \dif z \dif y \dif x\\
		&= \int\limits_{-2}^{2} \int\limits_{-\sqrt{4 - x^2}}^{\sqrt{4 - x^2}} (y + 1) (12 - 3x^2 - 3y^2) \dif y \dif x\\
		&= \int\limits_{-2}^{2} \int\limits_{-\sqrt{4 - x^2}}^{\sqrt{4 - x^2}} \left( y(12 - 3x^2 - 3y^2) + (12 - 3x^2 - 3y^2) \right) \dif y \dif x\\
		\intertext{$y(12 - 3x^2 - 3y^2)$ is odd and $(12 - 3x^2 - 3y^2)$ is even.}
		\therefore \iint\limits_{S} \overline{F} \cdot \hat{n} \dif S &= \int\limits_{-2}^{2} 2 \int\limits_{0}^{\sqrt{4 - x^2}} (12 - 3x^2 - 3y^2) \dif y \dif x\\
		&= 6 \int\limits_{-2}^{2} \int\limits_{0}^{\sqrt{4 - x^2}} (4 - x^2 - y^2) \dif y \dif x\\
		&= 6 \int\limits_{-2}^{2} \left. \left( (4 - x^2)y - \dfrac{y^3}{3} \right) \right|_{y = 0}^{y = \sqrt{4 - x^2}} \dif x\\
		&= 6 \int\limits_{-2}^{2} (4 - x^2) \left( \sqrt{4 - x^2} - \dfrac{1}{3} (4 - x^2) \sqrt{4 - x^2} \right) \dif x\\
	\end{align*}
	Let
	\begin{align*}
		x &= 2 \sin \theta\\
		\therefore \dif x &= 2 \cos \theta \dif \theta
	\end{align*}
	Therefore, solving,
	\begin{align*}
		\iint\limits_{S} \overline{F} \cdot \hat{n} \dif S &= 24 \pi
	\end{align*}
\end{solution}

\end{document}
