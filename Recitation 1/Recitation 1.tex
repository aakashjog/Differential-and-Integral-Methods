\documentclass[fleqn]{article}
\setcounter{secnumdepth}{4}
\usepackage{amsmath, amssymb, esdiff}
\usepackage{datetime}
\usepackage{ulem}
\usepackage{enumerate}
\newcommand\numberthis{\addtocounter{equation}{1}\tag{\theequation}}


\setlength{\mathindent}{0pt}

%opening
\title{Recitation 1}
\author{}
\date{\formatdate{29}{10}{2014}}

\begin{document}

\maketitle
\setlength{\mathindent}{0pt}

\tableofcontents

\newpage
\section{General Information}

\textbf{Michael Bromberg}\\
mic1@post.tau.ca.il\\

\newpage
\section{Domain of Definition of a Function}

\subsection{Examples}

\subsubsection{Find the domain of definition of the following functions.}

\paragraph{$\ln (1 - \lvert x \rvert) + \dfrac{1}{\sin x}$\\}

For $\ln x$ to be defined, it is necessary that $1 - \lvert x \rvert > 0$ \\
\begin{equation*}
	\therefore \lvert x \rvert < 1 \Rightarrow -1 < x < 1 
\end{equation*}
For $\dfrac{1}{\sin x}$ to be defined, $\sin x \neq 0$\\
\begin{equation*}
	\therefore x \neq k\pi, k \in \mathbb{Z}
\end{equation*}
Therefore, the domain of definition is $(-1,1) - \{0\}$

\paragraph{$\sqrt{\dfrac{x+5}{\lvert	x^4 - 16	\rvert}}$\\}

For the square root to be defined, 
\begin{equation*}
	\dfrac{x+5}{\lvert	x^4 - 16	\rvert} > 0
\end{equation*}

For the ratio to be defined, 
\begin{equation*}
	x^4 - 16 /\neq 0
\end{equation*}

Therefore, the domain is $[-5, \infty) - \{-2, 2\}$

\newpage
\section{Odd and Even Functions}

\subsection{Even function}

If $f(-x) = f(x) ; (x, -x \in D)$ then, $f$ is called an \emph{even function}. \\
Each even function is symmeteric about the $y$-axis. \\

\subsection{Odd function}

If $f(-x) = -f(x) ; (x, -x \in D)$ then, $f$ is called an \emph{odd function}. \\
Each odd function is symmeteric about the origin. \\

\subsection{Examples}

\subsubsection{Prove that if $f$ is even and $g$ is odd, then $f \cdot g$ is odd}

Let $x \in D(f) \bigcap D(g)$. Then, $-x \in D(f) \bigcap D(g)$, because $f$ and $g$ are even and odd respectively.\\
\begin{align*}
	f(-x) g(-x) = f(x) (-g(x)) = -f(x) g(x)\\
	\Rightarrow f \cdot g \text{ is odd.}
\end{align*}

\subsubsection{Check if $f(x) = x^5 + x^3 - x$ is odd or even}

\begin{align*}
	f(-x) &=& (-x)^5 + (-x)^3 - (-x)\\
	&=& -x^5 - x^3 +x\\
	&=& -f(x)
\end{align*}

Therefore, $f$ is odd.\\

\subsubsection{Check if $f(x) = 2^{x^2 +x}$ is odd, even or neither}

\begin{align*}
	f(1) = 2^{1^2 + 1} = 4\\
	f(-1) = 2^{(-1)^2 +1} = 2^0 = 1\\
\end{align*}

Therefore, $f$ is neither odd nor even.\\

\newpage
\section{Image and Range of a Function}

\subsection{Examples}

\subsubsection{What is the image of}

\paragraph{$f(x) = \ln x$ in the domain $(0, 4]$\\}

As the function is monotonic, it is easy to observe that the image is $(-\infty, \ln 4]$

\paragraph{$f(x) = \cos x$ in the domain $(0, 4]$\\}

From the graph of the function, it is evident that the image is $[-1,1)$

\newpage
\section{Graphs}

\subsection{Shifting with respect to the axes}
$f(x+a)$ is the graph of $f(x)$, shifted by $a$, in the direction of the $x$-axis, opposite to the sign of $a$. \\
$f(x) + a$ is the graph of $f(x)$, shifted by $a$, in the direction of the $y$-axis, according to the sign of $a$. \\

\subsection{Mirror Images with respect to the axes}

$f(-x)$ is the mirror image of $f(x)$ w.r.t. the $y$-axis.\\
$-f(x)$ is the mirror image of $f(x)$ w.r.t. the $x$-axis.\\

\newpage
\section{Monotonic Functions}

\subsection{Examples}

\subsubsection{Are the following functions monotonic?}

\paragraph{$e^{e^x}$\\}

$e^{e^x} = e^{(e^x)}$ is a composition of two monotonically increasing functions. Therefore, it is monotonically increasing.

\paragraph{$x^2 - 1$\\}

The function is not monotonic over $\mathbb{R}$.

\newpage
\section{Inequalities}

\subsection{Examples}

\subsubsection{Solve the following}

\paragraph{$\lvert x + 6 \rvert < \lvert x - 2 \rvert$\\}
\subparagraph{Solving by dividing the domain into regions\\}
The regions are
\begin{align*}
	x \leq -6\\
	-6 < x \leq 2\\
	2 < x
\end{align*}

\subparagraph{Solving by squaring both sides}

\begin{align*}
	\lvert x + 6 \rvert < \lvert x -2 \rvert \\
	\Leftrightarrow (x+6)^2 < (x-2)^2 \\
	\Leftrightarrow 12x + 36 < -4x +4 \\
	\Leftrightarrow 16x < -32 \\
	\Leftrightarrow x < -2 \\
\end{align*}

\paragraph{$\dfrac{x-1}{x-3} > \dfrac{x+3}{x+1}$\\}

We multiply both sides by $(x-3)^2$ and $(x+1)^2$, rather than $(x-3)$ and $(x+1)$, to avoid dealing with flipping of the direction of the inequality.\\

\begin{align*}
	\therefore (x-1)(x-3)(x+1)^2 > (x+3)(x+1)(x-3)^2 \\
	\Leftrightarrow (x-3)(x+1)\left((x-1)(x+1 - (x+3)(x-3)\right) > 0 \\
	\Leftrightarrow 8(x-3)(x+1) > 0
\end{align*}

Therefore the inequality holds iff $x > 3$ or $x < -1$

\section{Periodic Functions}

$f$ is periodic iff $\exists \, T>0$ s.t. $f(x+T) = f(x), \forall x \in D(f)$\\
The smallest $T$, if it exists, for which the above equality holds true, is called the period of $f$.\\
Note that if $f$ has period $T$, and $g$ has a period which is a rational multiple of $T$, say $\dfrac{m}{n} T$, then $f$ and $g$ have a mutual period $mT$.

\end{document}
