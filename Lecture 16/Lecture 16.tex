\documentclass[fleqn, a4paper, 12pt]{article}
\usepackage{amsmath, amssymb, amsthm, esdiff}
\usepackage[table]{xcolor}
\usepackage{commath}
\usepackage{gensymb}
\usepackage{hyperref}
\usepackage{tikz, pgfplots}
\usetikzlibrary{calc}
\usepackage{datetime}
\usepackage{setspace}
\usepackage{ulem}
\usepackage{xfrac}

\usepackage{enumerate, enumitem}

\setcounter{secnumdepth}{4}

\newcommand\numberthis{\addtocounter{equation}{1}\tag{\theequation}}

\theoremstyle{definition}
\newtheorem{example}{Example}
\newtheorem{definition}{Definition}

\theoremstyle{theorem}
\newtheorem{theorem}{Theorem}
\newtheorem{corollary}{Corollary}

\theoremstyle{remark}
\newtheorem{remark}{Remark}
\newtheorem{case}{Case}

\newenvironment{solution}
{\begin{proof}[Solution]\let\qed\relax}
	{\end{proof}}

\makeatletter
\@addtoreset{corollary}{theorem} %resets corollary numbers after a theorem
\makeatother

%opening
\title{Lecture 16}
\author{Aakash Jog}
\date{\formatdate{23}{12}{2014}}

\begin{document}
	
\maketitle
%\setlength{\mathindent}{0pt}

\tableofcontents

\newpage

\section{Functions of Multiple Variables}

\begin{definition}[Function of multiple variables]
	Let $D$ be a set of points from $\mathbb{R}^2$. A function $f$ of two variables is a law which corresponds to each pair $(x, y) \in D$ a unique real number $z$. It is denoted as $f(x, y) = z$. The set $D$ is called the domain of definition of $f(x, y)$ and a set of all values of $f(x, y)$ is called the image of $f$.
\end{definition}

\begin{definition}[Graph]
	If $f$ is a function of two variables with the domain of definition $D$, then the graph of $f$ is the set 
	\begin{equation*}
		S = \{(x, y, z) \in \mathbb{R}^3 : z = f(x, y), (x, y) \in D\}
	\end{equation*}
\end{definition}

\begin{definition}[Limit]
	Assume that $f(x, y)$ is defined on some open neighbourhood of $(a, b)$, except perhaps at $(a, b)$ itself. A number $L$ is called the limit of $f(x, y)$ at $(a, b)$ if, as $(x, y) \to (a, b)$, $f(x, y) \to L$ over any curve which ends at $(a, b)$.
	\begin{equation*}
		\lim\limits_{(x,y) \to (a,b)} f(x,y) = L
	\end{equation*}
\end{definition}

\begin{definition}[Continuous function]
	A function $f(x, y)$ is defined in an open neighbourhood of $(a,b)$ is caed continuous at $(a,b)$ if
	\begin{equation*}
		\lim\limits_{(x,y) \to (a,b)} f(x,y) = f(a,b)
	\end{equation*}
\end{definition}

\begin{remark}
	If there exist two curves $C_1$ and $C_1$, s.t. 
	\begin{align*}
	\lim\limits_{(x,y) \stackrel{C_1}{\to} (0,0)} \dfrac{x^2 - y^2}{x^2 + y^2} &= L_1
	\lim\limits_{(x,y) \stackrel{C_2}{\to} (0,0)} \dfrac{x^2 - y^2}{x^2 + y^2} &= L_2
	\end{align*}
	and $L_1 \neq L_2$, then, $\nexists \lim\limits_{(x,y) \to (0,0)} \dfrac{x^2 - y^2}{x^2 + y^2}$.
\end{remark}

\begin{example}
	Does $\lim\limits_{(x,y) \to (0,0)} \dfrac{x^2 - y^2}{x^2 + y^2}$ exist?
\end{example}

\begin{solution}
	If $y = 0$, $x \to 0^+$,
	\begin{align*}
		\lim\limits_{(x,y) \to (0,0)} \dfrac{x^2 - y^2}{x^2 + y^2} &= \dfrac{x^2}{x^2}\\
		&= 1
	\end{align*}
	If $x = 0$, $y \to 0^+$,
	\begin{align*}
	\lim\limits_{(x,y) \to (0,0)} \dfrac{x^2 - y^2}{x^2 + y^2} &= \dfrac{-y^2}{y^2}\\
	&= -1
	\end{align*}
	Hence, the limit does not exist.
\end{solution}

\section{Partial Derivatives of Functions of Multiple Variables}

\begin{definition}[Partial derivative of $f(x,y)$]
	The limit
	\begin{equation*}
		f_x(a,b) = \lim\limits_{\Delta x \to 0} \dfrac{f(a + \Delta x, b) - f(a, b)}{\Delta x}
	\end{equation*}
	if it exists, is called the partial derivative of $f(x, y)$ with respect to $x$ at $(a, b)$. It can be denoted as $f_x$, $f'_x$, $\dpd{f}{x}$ or $D_x f$. Similarly for the partial derivative of $f(x, y)$ with respect to $y$.
\end{definition}

\subsection{Geometrical Interpretation}

Let $C_1$ be the intersection line of $z = f(x,y)$ and the plane $y = b$. Let $T_1$ be the tangent line to $C_1$, in the plane $y = b$ at $P$. Then, $f_x(a,b)$ is the slope of $T_1$.

\begin{definition}
	If $\exists f_x(x,y)$ and $\exists f_y(x,y)$ in some open neighbourhood of $(a,b)$ and $f_x(x,y)$ and $f_y(x,y)$ are continuous at $(a,b)$, then the tangent plane to $z = f(x,y)$ at the point $P(a, b, f(a,b))$ is a plane which passes through $P$ and contains the straight lines $T_1$ and $T_2$. It is given by
	\begin{equation*}
		z - f(a,b) = f_x(a,b) (x - a) + f_y(a,b) (y - b)
	\end{equation*}
\end{definition}

\end{document}