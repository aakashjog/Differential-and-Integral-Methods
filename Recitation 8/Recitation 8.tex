\documentclass[fleqn, 12pt]{article}
\setcounter{secnumdepth}{4}
\usepackage{amsmath, amssymb, amsthm}
\usepackage{commath, esdiff}
\usepackage{datetime}
\usepackage{graphicx, epstopdf}
\usepackage{ulem}
\usepackage{xfrac}
\usepackage{enumerate}
\usepackage{tikz}

\newcommand\numberthis{\addtocounter{equation}{1}\tag{\theequation}}

\theoremstyle{definition}
\newtheorem{example}{Example}
\newtheorem{definition}{Definition}

\theoremstyle{theorem}
\newtheorem{theorem}{Theorem}

\newenvironment{solution}
{\begin{proof}[Solution]\let\qed\relax}
	{\end{proof}}

%opening
\title{Recitation 8}
\author{}
\date{\formatdate{17}{12}{2014}}

\begin{document}

\maketitle
%\setlength{\mathindent}{0pt}

\tableofcontents

\newpage
\section{Indefinite Integrals}

\begin{example}
	\begin{equation*}
		\int \dfrac{x + 2}{x (x + 1)^2} \dif x
	\end{equation*}
\end{example}

\begin{solution}
	Let
	\begin{align*}
		\dfrac{x + 2}{x (x + 1)^2} &= \dfrac{A}{x} + \dfrac{B}{x + 1} + \dfrac{C}{(x + 1)^2}\\
		&= \dfrac{A(x+1)^2 + B(x)(x+1) + C(x)}{x(x+1)^2}\\
		\therefore x + 2 &= A(x+1)^2 + B(x)(x+1) + C(x)\\
		\therefore x + 2 &= (A + B) x^2 + (2A + B + C) x + A
	\end{align*}
	Therefore,
	\begin{align*}
		A &= 2\\
		B &= -2\\
		C &= -1
	\end{align*}
	Therefore,
	\begin{align*}
		\int \dfrac{x + 2}{x (x + 1)^2} \dif x &= \int \left( \dfrac{2}{x} - \dfrac{2}{x + 1} - \dfrac{1}{(x + 1)^2} \right) \dif x\\
		&= 2 \ln |x| - 2 \ln |x + 1| + \dfrac{1}{x + 1} + d
	\end{align*}
\end{solution}

\begin{example}
	\begin{equation*}
		\int \dfrac{\cos x}{10 + \sin x} \dif x
	\end{equation*}
\end{example}

\begin{solution}
	Let
	\begin{align*}
		y &= \sin x\\
		\therefore \dif y &= \cos x \dif x
	\end{align*}
	Therefore,
	\begin{align*}
		\int \dfrac{\cos x}{10 + \sin x} \dif x &= \int \dfrac{\dif y}{10 + y}\\
		&= \ln |10 + y| + c\\
		&= \ln |10 + \sin x| + c
	\end{align*}
\end{solution}

\begin{example}
	\begin{equation*}
		\int \dfrac{\dif x}{\sqrt{x} + \sqrt[3]{x}} \dif x
	\end{equation*}
\end{example}

\begin{solution}
	Let
	\begin{align*}
		x &= t^6\\
		\therefore \dif x &= 6 t^5 \dif t
	\end{align*}
	Therefore,
	\begin{align*}
		\int \dfrac{\dif x}{\sqrt{x} + \sqrt[3]{x}} \dif x &= \int \dfrac{6 t^5}{t^3 + t^2} \dif t\\
		&= \int \dfrac{(t^3 + t^2)(6t^2 - 6t + 6) - 6t^2}{t^3 + t^2} \dif t\\
		&= \int 6t^2 - 6t + 6 - \dfrac{6}{t + 1} \dif t\\
		&= \dfrac{6t^3}{3} - \dfrac{6t^2}{2} + 6t - 6 \ln |t + 1| + c\\
		&= \dfrac{6\sqrt{x}}{3} - \dfrac{6\sqrt[3]{x}}{2} + 6\sqrt[6]{x} - 6 \ln |\sqrt[6]{x} + 1| + c
	\end{align*}
\end{solution}

\begin{example}
	\begin{equation*}
		\int \sin^3 x \dif x
	\end{equation*}
\end{example}

\begin{solution}
	\begin{align*}
		\int \sin^3 x \dif x &= \int \sin^2 x \sin x \dif x\\
		&= \int (1 - \cos^2 x) \sin x \dif x
	\end{align*}
	Let
	\begin{align*}
		y &= \cos x\\
		\therefore dif y &= - \sin x \dif x
	\end{align*}
	Therefore,
	\begin{align*}
		\int (1 - \cos^2 x) \sin x \dif x &= -\int (1 - y^2) \dif y\\
		&= -y + \dfrac{y^3}{3} + c\\
		&= - \cos x + \dfrac{\cos^3 x}{3} + c
	\end{align*}
\end{solution}

\begin{example}
	\begin{equation*}
		\int \dfrac{\dif x}{\sqrt{28 - 12x - x^2}}
	\end{equation*}
\end{example}

\begin{solution}
	\begin{align*}
		\int \dfrac{\dif x}{\sqrt{28 - 12x - x^2}} &= \int \dfrac{\dif x}{\sqrt{-(x + 6)^2 + 36 + 28}}
	\end{align*}
	Let 
	\begin{align*}
		t &= x + 6\\
		\therefore \dif t &= \dif x
	\end{align*}
	Therefore,
	\begin{align*}
		\int \dfrac{\dif x}{\sqrt{-(x + 6)^2 + 36 + 28}} &= \int \dfrac{\dif t}{\sqrt{-t^2 + 64}}\\
		&= \int \dfrac{\dif t}{8\sqrt{-\left( \dfrac{t}{8} \right)^2 + 1}}
	\end{align*}
		Let 
		\begin{align*}
		y &= \dfrac{t}{8}\\
		\therefore \dif y &= \dfrac{1}{8} \dif t
		\end{align*}
		Therefore,
		\begin{align*}
			\int \dfrac{\dif t}{8\sqrt{-\left( \dfrac{t}{8} \right)^2 + 1}} &= \int \dfrac{\dif y}{\sqrt{1 - y^2}}\\
			&= \arcsin y + c\\
			&= \arcsin \dfrac{x + 6}{8} + c
		\end{align*}
\end{solution}

\end{document}

