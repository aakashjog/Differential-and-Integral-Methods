\documentclass[fleqn, a4paper, 12pt]{article}
\usepackage{amsmath, amssymb, amsthm, esdiff}
\usepackage[table]{xcolor}
\usepackage{commath}
\usepackage{gensymb}
\usepackage{hyperref}
\usepackage{tikz, pgfplots}
\usetikzlibrary{calc}
\usepackage{datetime}
\usepackage{setspace}
\usepackage{ulem}
\usepackage{xfrac}
\usepackage{siunitx}

\usepackage{enumerate, enumitem}

\setcounter{secnumdepth}{4}

\newcommand\numberthis{\addtocounter{equation}{1}\tag{\theequation}}

\theoremstyle{definition}
\newtheorem{example}{Example}
\newtheorem{definition}{Definition}

\theoremstyle{theorem}
\newtheorem{theorem}{Theorem}
\newtheorem{corollary}{Corollary}

\theoremstyle{remark}
\newtheorem{remark}{Remark}
\newtheorem{case}{Case}

\newenvironment{solution}
{\begin{proof}[Solution]\let\qed\relax}
	{\end{proof}}

\makeatletter
\@addtoreset{corollary}{theorem} %resets corollary numbers after a theorem
\makeatother

%opening
\title{Lecture 21}
\author{Aakash Jog}
\date{\formatdate{8}{1}{2015}}

\begin{document}
	
\maketitle
%\setlength{\mathindent}{0pt}

\tableofcontents

\newpage

\section{Triple Integrals}

\begin{definition}[Solid of the second kind]
	A solid $E$ in $\mathbb{R}^3$ is called a solid of the second kind if there exist continuous functions $\varphi_1(x,z)$ and $\varphi_2(x,z)$, s.t.
	\begin{equation*}
		E_{\textnormal{II}} = \{(x,y,z) | (x,z) \in D, \varphi_1(x,z) \leq y \leq \varphi_2(x,z)\}
	\end{equation*}
\end{definition}

\begin{theorem}
	If $f(x,y,z)$ is continuous on $E_{\textnormal{II}}$,
	\begin{equation*}
		\iiint\limits_{E_{\textnormal{II}}} f(x,y,z) \dif V = \iint\limits_{D} \left( \int\limits_{\varphi_1(x,z)}^{\varphi_2(x,z)} f(x,y,z) \dif y \right) \dif A
	\end{equation*}
\end{theorem}

\begin{definition}[Solid of the third kind]
	A solid $E$ in $\mathbb{R}^3$ is called a solid of the third kind if there exist continuous functions $\varphi_1(y,z)$ and $\varphi_2(y,z)$, s.t.
	\begin{equation*}
	E_{\textnormal{III}} = \{(x,y,z) | (y,z) \in D, \varphi_1(y,z) \leq y \leq \varphi_2(y,z)\}
	\end{equation*}
\end{definition}

\begin{theorem}
	If $f(x,y,z)$ is continuous on $E_{\textnormal{III}}$,
	\begin{equation*}
	\iiint\limits_{E_{\textnormal{III}}} f(x,y,z) \dif V = \iint\limits_{D} \left( \int\limits_{\varphi_1(y,z)}^{\varphi_2(y,z)} f(x,y,z) \dif z \right) \dif A
	\end{equation*}
\end{theorem}

\begin{example}
	Calculate $\displaystyle \iiint\limits_{E} x z \dif V$ when $E$ is a solid bounded by the planes $x = 0$, $y = 0$, $z = 0$ and $x + y + z = 1$.
\end{example}

\begin{solution}
	\begin{align*}
		\iiint\limits_{E_{\textnormal{I}}} x z \dif V &= \iint\limits_{D} \left( \int\limits_{\varphi_1(x,y)}^{\varphi_2(x,y)} x z \dif z \right) \dif A\\
		&= \iint\limits_{D} x \left. \dfrac{z^2}{2} \right\rvert_{z = 0}^{z = 1 - x - y} \dif A\\
		&= \dfrac{1}{2} \iint\limits_{D} x (1 - x - y)^2 \dif A\\
		&= \dfrac{1}{2} \int\limits_{0}^{1} \int\limits_{0}^{1 - x} x (1 - x - y)^2 \dif y \dif x\\
		&= \dfrac{1}{2} \left. \int\limits_{0}^{1} -x \dfrac{(1 - x - y)^3}{3} \right\rvert_{y = 0}^{y = 1 - x} \dif x \\
		&= \dfrac{1}{6} \int\limits_{0}^{1} x (1 - x)^3 \dif x\\
		&= \dfrac{1}{120}
	\end{align*}
\end{solution}

\section{Line Integrals of Scalar Functions}

\begin{definition}[Line integral of scalar function]
	Let a curve from $A$ to $B$ be divided into $n$ parts, s.t. $A = P_0$, $B = P_n$.\\
	Let $\Delta s_i$ be the length of the curve $P_{i - 1} P_i$.\\
	Let 
	\begin{equation*}
		\Delta T = \max \{\Delta s_i\}
	\end{equation*}
	Then, 
	\begin{equation*}
		\int\limits_{C} f(x,y) \dif s = \lim\limits_{\Delta T \to 0} \sum_{i = 1}^{n} f({x_i}^*, {y_i}^*) \Delta s_i
	\end{equation*}
	is called a line integral of the first kind or line integral of scalar function.\\
	This integral does not depend on the direction of the curve.\\
	Geometrically, the line integral is the area under the curve $z = f(x,y)$ above the line $C$.
\end{definition}	

\begin{definition}[Smooth curve]
	Let $C$ be given parametrically as
	\begin{equation*}
		\overline{r} (t) = \left( x(t), y(t) \right) \quad t : a \to b
	\end{equation*}
	The curve is said to be smooth if 
	\begin{equation*}
		\overline{r}'(t) = \left( x'(t), y'(t) \right)
	\end{equation*}
	is a continuous function on $[a,b]$, $\overline{r}'(t) \neq \overline{0}$ on $(a,b)$ and $\overline{r}'(t)$ is also continuous on $(a,b)$.
\end{definition}

\begin{theorem}
	If $f(x,y)$ is continuous and $C$ is smooth, then
	\begin{equation*}
		\int\limits_{C} f(x,y) \dif s = \int\limits_{a}^{b} f(x(t), y(t)) \sqrt{\left( x'(t) \right)^2 + \left( y'(t) \right)^2} \dif t
	\end{equation*}
\end{theorem}

\begin{example}
	Find $\displaystyle \int\limits_{C} (2 + x^2 y) \dif s$ when $C : y = \sqrt{1 - x^2} , -1 \leq x \leq 1$.
\end{example}

\begin{solution}
	\begin{align*}
		\int\limits_{C} (2 + x^2 y) \dif s &= \int\limits_{0}^{1} (2 + \cos^2 t \sin t) \sqrt{-(\sin t)^2 + (\cos t)^2} \dif t\\
		&= 2 \pi + \dfrac{2}{3}
	\end{align*}
\end{solution}

\begin{theorem}
	If the curve $C$ is a piecewise smooth curve, i.e. $C = C_1 \cup \dots \cup \dots C_n$, where each $C_i$ is smooth, and the length of all $C_i \cap C_j$ is zero, then
	\begin{equation*}
		\int\limits_{C} f(x,y) \dif s = \int\limits_{C_1} f(x,y) \dif s + \dots + \int\limits_{C_n} f(x,y) \dif s
	\end{equation*}
\end{theorem}

\begin{definition}[Line integral with respect to $x$ and $y$]
	A line integral of $f(x,y)$ over $C$ with respect to $x$ is
	\begin{equation*}
		\int\limits_{C} f(x,y) \dif x = \lim\limits_{\Delta T \to 0} \sum_{i = 1}^{n} f({x_i}^*, {y_i}^*) \Delta x_i
	\end{equation*}
	and with respect to $y$ is
	\begin{equation*}
		\int\limits_{C} f(x,y) \dif y = \lim\limits_{\Delta T \to 0} \sum_{i = 1}^{n} f({x_i}^*, {y_i}^*) \Delta y_i
	\end{equation*}
	These integrals depend on the direction of $C$.
\end{definition}

\begin{theorem}
	If $f(x,y)$ is continuous on $C$ and $C$ is smooth, then
	\begin{align*}
		\int\limits_{C} f(x,y) \dif x &= \int\limits_{a}^{b} f(x(t), y(t)) x'(t) \dif t\\
		\int\limits_{C} f(x,y) \dif y &= \int\limits_{a}^{b} f(x(t), y(t)) y'(t) \dif t
	\end{align*}
\end{theorem}

\begin{example}
	Calculate by definition $\displaystyle \int\limits_{C} \dif s$, $\displaystyle \int\limits_{C} \dif x$, $\displaystyle \int\limits_{C} \dif y$ for the curve $C$ given by the union of the line segments from $(0,0,0)$ to $(1,1,0)$ and $(1,1,0)$ to $(2,0,0)$.
\end{example}

\begin{solution}
	\begin{align*}
		\int\limits_{C} \dif s &= \lim\limits_{\Delta T \to 0} \sum_{i = 1}^{n} \Delta s_i\\
		&= \text{ length of line segment }\\
		&= 2 \sqrt{2}
	\end{align*}
	\begin{align*}
		\int\limits_{C} \dif x &= \lim\limits_{\Delta T \to 0} \sum_{i = 1}^{n} \Delta x_i\\
		&= \Delta x\\
		&= 2
	\end{align*}
	\begin{align*}
		\int\limits_{C} \dif y &= \lim\limits_{\Delta T \to 0} \sum_{i = 1}^{n} \Delta x_i\\
		&= \Delta y\\
		&= 0
	\end{align*}
\end{solution}

\section{Line Integrals of Vector Functions}

\begin{definition}[Line integral of vector function]
	\begin{align*}
		W &= \int\limits_{C} \overline{F} \cdot \hat{T} \dif s\\
		&= \int\limits_{C} \overline{F} \cdot \dif \overline{z}\\
		&= \int\limits_{C} P \dif x + Q \dif y + R \dif z
	\end{align*}
\end{definition}	

\begin{example}
	Find the work $W$ done by the force $\overline{F}(x,y) = (x, xy)$ over the curve $C : \overline{r}(t) = (2 \cos t, 2 \sin t) , t : \pi \to 2 \pi$.
\end{example}

\begin{solution}
	\begin{align*}
		W &= \int\limits_{C} \overline{F} \cdot \hat{T} \dif s\\
		&= \int\limits_{\pi}^{2 \pi} \left( 2 \cos t (-2 \sin t) + 2 \cos t \cdot 2 \sin t \cdot 2 \cos t \right) \dif t\\
		&= \int\limits_{\pi}^{2 \pi} \left( - 2 \sin (2t) + 8 \cos^2 t \sin t \right) \dif t\\
		&= \left. \cos (2t) - \dfrac{8}{3} \cos^3 t \right\rvert_{\pi}^{2 \pi}\\
		&= \left( 1 - \dfrac{8}{3} \right) - \left( 1 + \dfrac{8}{3} \right)\\
		&= -\dfrac{16}{3}
	\end{align*}
\end{solution}

\end{document}