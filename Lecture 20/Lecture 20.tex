\documentclass[fleqn, a4paper, 12pt]{article}
\usepackage{amsmath, amssymb, amsthm, esdiff}
\usepackage[table]{xcolor}
\usepackage{commath}
\usepackage{gensymb}
\usepackage{hyperref}
\usepackage{tikz, pgfplots}
\usetikzlibrary{calc}
\usepackage{datetime}
\usepackage{setspace}
\usepackage{ulem}
\usepackage{xfrac}
\usepackage{siunitx}

\usepackage{enumerate, enumitem}

\setcounter{secnumdepth}{4}

\newcommand\numberthis{\addtocounter{equation}{1}\tag{\theequation}}

\theoremstyle{definition}
\newtheorem{example}{Example}
\newtheorem{definition}{Definition}

\theoremstyle{theorem}
\newtheorem{theorem}{Theorem}
\newtheorem{corollary}{Corollary}

\theoremstyle{remark}
\newtheorem{remark}{Remark}
\newtheorem{case}{Case}

\newenvironment{solution}
{\begin{proof}[Solution]\let\qed\relax}
	{\end{proof}}

\makeatletter
\@addtoreset{corollary}{theorem} %resets corollary numbers after a theorem
\makeatother

%opening
\title{Lecture 20}
\author{Aakash Jog}
\date{\formatdate{6}{1}{2015}}

\begin{document}
	
\maketitle
%\setlength{\mathindent}{0pt}

\tableofcontents

\newpage

\section{Double Integrals}

\begin{example}
	Calculate $\displaystyle \iint\limits_{D} \sin (y^2) \dif A$ for $D$ given by the triangle enclosed by $(0,0)$, $(0,1)$, $(1,1)$.
\end{example}

\begin{solution}
	The domain $D$ is of the first and second kind.\\
	\begin{align*}
		D_{I} &: 0 \leq x \leq 1 \quad\&\quad x \leq y \leq 1\\
		D_{II} &: 0 \leq y \leq 1 \quad\&\quad 0 \leq x \leq y
	\end{align*}
	Therefore,
	\begin{align*}
		\iint\limits_{D_{I}} \sin (y^2) \dif A &= \int\limits_{0}^{1} \int\limits_{x}^{1} \sin (y^2) \dif y \dif x
	\end{align*}
	This form is unsolvable.
	\begin{align*}
		\iint\limits_{D_{II}} \sin (y^2) \dif A &= \int\limits_{0}^{1} \int\limits_{0}^{y} \sin (y^2) \dif x \dif y\\
		&= \int\limits_{0}^{1} \left. \sin (y^2) x \right\rvert_{x = 0}^{x = y} \dif y\\
		&= \int\limits_{0}^{1} y \sin (y^2) \dif y\\
		&= \left. -\dfrac{1}{2} \cos (y^2) \right\rvert_{0}^{1}\\
		&= -\dfrac{1}{2} \cos 1 + \dfrac{1}{2}
	\end{align*}
\end{solution}

\begin{theorem}
	If $D$ is not a domain of both kinds, but $D = D_{I} \cup D_{II}$, s.t. the area of $D_{I} \cap D_{II}$ is zero, then
	\begin{equation*}
		\iint\limits_{D} f(x,y) \dif A = \iint\limits_{D_{I}} f(x,y) \dif A + \iint\limits_{D_{II}} f(x,y) \dif A
	\end{equation*}
\end{theorem}

\subsection{Applications}

\begin{enumerate}
	\item If $f_x(x,y)$ and $f_y(x,y)$ are continuous in $D$, then the area of the surface $\sigma : z = f(x,y)$ above $D$ is equal to 
		\begin{equation*}
			S(\Sigma) = \iint\limits_{D} \sqrt{1 + \left( f_x(x,y) \right)^2 + \left( f_y(x,y) \right)^2} \dif A
		\end{equation*}
	\item If $\rho(x,y)$ is the density function of a thin body,
		\begin{align*}
			m &= \iint\limits_{D} \rho(x,y) \dif A
		\end{align*}
		\begin{align*}
			(x_{\text{COM}}, y_{\text{COM}}) &= \left( \dfrac{M_y}{m}, \dfrac{M_x}{m} \right)\\
			\intertext{where}
			M_x &= \iint\limits_{D} y \rho(x,y) \dif A\\
			M_y &= \iint\limits_{D} x \rho(x,y) \dif A\\
		\end{align*}
\end{enumerate}

\section{Triple Integrals}

\subsection{Triple Integrals on Parallelepiped Domains}

\begin{definition}[Triple integral on parallelepiped domain]
	Consider a parallelepiped
	\begin{align*}
		B &= \{(x,y,z) | a \leq x \leq b, c \leq y \leq d, e \leq z \leq g\}
	\end{align*}
	Consider a partition $T$
	\begin{gather*}
		a = x_0 < x_1 < \dots < x_{n - 1} < x_n = b\\
		c = y_0 < y_1 < \dots < y_{n - 1} < y_n = d\\
		e = z_0 < z_1 < \dots < z_{n - 1} < z_n = g
	\end{gather*}
	Consider a parallelepiped
	\begin{equation*}
		B_{ijk} = [x_{i-1}, x_i] \times [y_{j-1}, y-j] \times [z_{k-1}, z_k]
	\end{equation*}
	Let 
	\begin{equation*}
		{P_{ijk}}^* = \left( {x_{ijk}}^*, {y_{ijk}}^*, {z_{ijk}}^* \right)
	\end{equation*}
	be a point in $B_{ijk}$.
	Let 
	\begin{align*}
		\Delta x_i &= x_i - x_{i-1}\\
		\Delta y_j &= y_j - y_{j-1}\\
		\Delta z_k &= z_k - z_{k-1}
	\end{align*}
	Therefore,
	\begin{align*}
		\Delta V_{ijk} &= \Delta x_i \cdot \Delta y_j \cdot \Delta z_k
	\end{align*}
	Let
	\begin{equation*}
		\Delta T = \max\{\Delta x_i, \Delta y_j, \Delta z_k\}
	\end{equation*}
	$\displaystyle \sum_{i = 1}^{n} \sum_{j = 1}^{m} \sum_{k = 1}^{l} f({x_{ijk}}^*, {y_{ijk}}^*, {z_{ijk}}^*) \Delta V_{ijk}$ is called a Riemann triple integral sum.\\
	The double integral of $f(x,y,z)$ over the domain of definition $B$ is the limit, if it exists, of the Riemann double integral sum, as $\Delta T \to 0$.
	\begin{equation*}
	\iiint\limits_{B} f(x,y,z) \dif V = \lim\limits_{\Delta T \to 0} \sum_{i = 1}^{n} \sum_{j = 1}^{m} \sum_{k = 1}^{l} f({x_{ijk}}^*, {y_{ijk}}^*, {z_{ijk}}^*) \Delta V_{ijk}
	\end{equation*}
\end{definition}

\begin{theorem}[Fubini Theorem]
	If $f(x,y,z)$ is continuous on $B = [a,b] \times [c,d] \times [e,g]$, then the triple integral and six iterated integrals exist, and they are equal.
	\begin{equation*}
		\iiint\limits_{B} f(x,y,z) \dif V = \int\limits_{a}^{b} \int\limits_{c}^{d} \int\limits_{e}^{g} f(x,y,z) \dif z \dif y \dif x = \dots
	\end{equation*}
\end{theorem}

\begin{example}
	Calculate \[\iiint\limits_{[0,1] \times [-1,2] \times [1,3]} z^2 \sin (x + y) \dif V\]
\end{example}

\begin{solution}
	\begin{align*}
		\iiint\limits_{[0,1] \times [-1,2] \times [1,3]} z^2 \sin (x + y) \dif V &= \int\limits_{0}^{1} \int\limits_{-1}^{2} \int\limits_{1}^{3} z^2 \sin (x + y) \dif z \dif y \dif x\\
		&= \int\limits_{0}^{1} \int\limits_{-1}^{2} \left. \dfrac{z^3}{3} \sin (x + y) \right\rvert_{z = 0}^{z = 3} \dif y \dif x\\
		&= \dfrac{26}{3} \int\limits_{0}^{1} \int\limits_{-1}^{2} \sin (x + y) \dif y \dif x\\
		&= \dfrac{26}{3} \int\limits_{0}^{1} \left. -\cos (x + y) \right\rvert_{y = -1}^{y = 2} \dif x\\
		&= \dfrac{26}{3} \int\limits_{0}^{1} \left( -\cos (x + 2) + \cos (x - 1) \right) \dif x\\
		&= \dfrac{26}{3} \left. \left( -\sin (x + 2) + \sin (x - 1) \right) \right\rvert_{0}^{1}\\
		&= \dfrac{26}{3} \left( -\sin 3 + 0 - \left( -\sin 2 + \sin (-1) \right) \right)\\
		&= \dfrac{26}{3} \left( -\sin 3 + \sin 2 + \sin 1 \right)
	\end{align*}
\end{solution}

\subsection{Triple Integrals on Arbitrary Domains}

\begin{definition}[Triple integral on arbitrary domain]
	Let $E$ be an arbitrary solid which is bounded and closed in $\mathbb{R}^3$ and $f(x,y,z)$ be defined on $E$. Consider a parallelepiped $B$, s.t. $E \subset B$ and construct the function $F(x,y,z)$, s.t.
	\begin{equation*}
		F(x,y,x) = 
			\begin{cases}
				f(x,y,z) &;\quad (x,y,z) \in E\\
				f(x,y,z) &;\quad 0 \notin E\\
			\end{cases}
	\end{equation*}
	If $\displaystyle \exists \iiint\limits_{B} F(x,y,z) \dif V$, then $\displaystyle \exists \iiint\limits_{E} f(x,y,z) \dif V$, and
	\begin{equation*}
	\iiint\limits_{E} f(x,y,z) \dif V = \iiint\limits_{B} F(x,y,z) \dif V
	\end{equation*}
\end{definition}

\begin{definition}[Solid of the first kind]
	A solid $E$ in $\mathbb{R}^3$ is called a solid of the first kind if there exist continuous functions $\varphi_1(x,y)$ and $\varphi_2(x,y)$, s.t.
	\begin{equation*}
		E_{I} = \{(x,y,z) | (x,y) \in D, \varphi_1(x,y) \leq z \leq \varphi_2(x,y)\}
	\end{equation*}
\end{definition}

\begin{theorem}
	If $f(x,y,z)$ is continuous on $E$,
	\begin{equation*}
		\iiint\limits_{E_{I}} f(x,y,z) \dif V = \iint\limits_{D} \left( \int\limits_{\varphi_1(x,y)}^{\varphi_2(x,y)} f(x,y,z) \dif x \right) \dif A
	\end{equation*}
\end{theorem}

\end{document}