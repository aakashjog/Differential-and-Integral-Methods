\documentclass[fleqn, a4paper, 12pt]{article}
\usepackage{amsmath, amssymb, amsthm, esdiff}
\usepackage[table]{xcolor}
\usepackage{todonotes, marginnote}
\usepackage{commath}
\usepackage{gensymb}
\usepackage{hyperref}
\usepackage{tikz, pgfplots}
\usetikzlibrary{calc}
\usepackage{datetime}
\usepackage{setspace}
\usepackage{ulem}
\usepackage{xfrac}
\usepackage{siunitx}

\usepackage{enumerate, enumitem}

\setcounter{secnumdepth}{4}

\newcommand\numberthis{\addtocounter{equation}{1}\tag{\theequation}}

\newcommand{\curl}{\mathrm{curl\,}}

\newcommand{\divergence}{\mathrm{div\,}}

\theoremstyle{definition}
\newtheorem{example}{Example}
\newtheorem{definition}{Definition}

\theoremstyle{theorem}
\newtheorem{theorem}{Theorem}
\newtheorem{corollary}{Corollary}

\theoremstyle{remark}
\newtheorem{remark}{Remark}
\newtheorem{case}{Case}

\newenvironment{solution}
{\begin{proof}[Solution]\let\qed\relax}
	{\end{proof}}

\makeatletter
\@addtoreset{corollary}{theorem} %resets corollary numbers after a theorem
\makeatother

%opening
\title{Review Session 1}
\author{Aakash Jog}
\date{\formatdate{22}{1}{2015}}

\begin{document}
	
\maketitle
%\setlength{\mathindent}{0pt}

\begin{example}
	Is the function 
	\begin{equation*}
		f(x) = 
			\begin{cases}
				\dfrac{\sin x}{x} &;\quad x \neq 0\\
				1 &;\quad x = 0
			\end{cases}
	\end{equation*}
	continuous at $x = 0$? Is it differentiable at $x = 0$? If yes, calculate $f'(0)$.
\end{example}

\begin{solution}
	\begin{align*}
		\lim\limits_{x \to 0} \dfrac{\sin x}{x} &= 1\\
		&= f(0)
	\end{align*}
	Therefore, $f(x)$ is continuous at $x = 0$.
	\begin{align*}
		f'(0) &= \lim\limits_{x \to 0} \dfrac{f(x) - f(0)}{x - 0}\\
		&= \lim\limits_{x \to 0} \dfrac{\dfrac{\sin x - x}{x}}{x}\\
		&= \lim\limits_{x \to 0} \dfrac{\sin x - x}{x^2}\\
		&= \lim\limits_{x \to 0} \dfrac{\cos x - 1}{2x}\\
		&= \lim\limits_{x \to 0} \dfrac{-\sin x}{2}\\
		&= 0
	\end{align*}
\end{solution}

\begin{example}
	Calculate $\displaystyle \int\limits_{1}^{\sqrt{3}} \dfrac{\arctan{x}}{x^2} \dif x$.
\end{example}

\begin{solution}
	\begin{align*}
		\int\limits_{1}^{\sqrt{3}} \dfrac{\arctan x}{x^2} \dif x &= \left. -\dfrac{1}{x} \arctan x \right|_{1}^{\sqrt{3}} + \int\limits_{1}^{3} \dfrac{1}{x} \cdot \dfrac{1}{x^2 + 1} \dif x\\
		&= -\dfrac{1}{\sqrt{3}} \arctan \sqrt{3} + \arctan 1 + \int\limits_{1}^{\sqrt{3}} \left( \dfrac{1}{x} - \dfrac{x}{x^2 + 1} \right) \dif x\\
		&= -\dfrac{\pi}{3 \sqrt{3}} + \dfrac{\pi}{4} + \left. \left( \ln x - \dfrac{1}{2} \cdot \ln(x^2 + 1) \right) \right|_{1}^{\sqrt{3}}\\
		&= -\dfrac{\pi}{3\sqrt{3}} + \dfrac{\pi}{4} + \ln \dfrac{\sqrt{6}}{2} 
	\end{align*}
\end{solution}

\begin{example}
	Write the Taylor's formula for the function $f(x) = \tan (x)$ at $a = 0$ for $n = 1$ and calculate approximately $\tan 0.1$ using the formula and give the estimation of the accuracy of the accuracy.
\end{example}

\begin{solution}
	\begin{align*}
		\tan x &= \tan(0) + (\sec^2 0)(x) + {2 \sec c \tan c}{2}(x^2)\\
		&= 0 + x + \sec c \tan c x^2
	\end{align*}
	Therefore,
	\begin{align*}
		\tan (0.1) &= 0 + (0.1) + \sec c \tan c (0.1)^2\\
		R_{\textnormal{max}} &= \sec (0.1) \tan (0.1) (0.01)
	\end{align*}
\end{solution}

\begin{example}
	Find the minimum and maximum of
	\begin{equation*}
		f(x,y) = \cos (2x) + \cos (2y)
	\end{equation*}
	under the constraint $x - y = \dfrac{\pi}{4}$. Find the points of minimum and maximum.
\end{example}

\begin{solution}
	Let
	\begin{align*}
		g(x,y) &= x - y
	\end{align*}
	Therefore,
	\begin{align*}
		\nabla f &= \lambda \nabla g\\
		g &= \dfrac{\pi}{4}
	\end{align*}
	\begin{align*}
		f_x &= \lambda g_x\\
		f_y &= \lambda g_y\\
		g &= \dfrac{\pi}{4}
	\end{align*}
	Therefore
	\begin{align*}
		\sin (x + y) = 0\\
		\therefore x + y &= \pi k\\
		x - y &= \dfrac{\pi}{4}
	\end{align*}
	Therefore,
	\begin{align*}
		x &= \dfrac{\pi}{8} + \dfrac{\pi k}{2}\\
		y &= -\dfrac{\pi}{8} + \dfrac{\pi k}{2}
	\end{align*}
	For all even $k$, the points are points of maxima. For all odd $k$, the points are points of minima. The minimum value of the function is $-\sqrt{2}$ and the maximum value of the function is $\sqrt{2}$.
\end{solution}

\begin{example}
	Calculate the line integral \[ \int\limits_{C} \left( 1 - \dfrac{y^2}{x^2} \cos \left( \dfrac{y}{x} \right) \right) \dif x + \left( \sin \left( \dfrac{y}{x} \right) + \dfrac{y}{x} \cos \left( \dfrac{y}{x} \right) \right) \dif y \] where $C$ is any curve which starts at $(1, \pi)$ and ends at $(2,\pi)$ and does not intersect the $y$-axis.
\end{example}

\begin{solution}
	Let
	\begin{align*}
		P(x,y) = 1 - \dfrac{y^2}{x^2} \cos \left( \dfrac{y}{x} \right)\\
		Q(x,y) = \sin \left( \dfrac{y}{x} \right) + \dfrac{y}{x} \cos \left( \dfrac{y}{x} \right)
	\end{align*}
	Therefore,
	\begin{align*}
		Q_x &= \cos \left( \dfrac{y}{x} \right) \left( -\dfrac{y}{x^2} \right) + \left( - \dfrac{y}{x^2} \right) \cos \left( \dfrac{y}{x} \right) + \dfrac{y}{x} \left( -\sin \left( \dfrac{y}{x} \right) \right) \left( -\dfrac{y}{x^2} \right)\\
		P_y &= -\dfrac{2y}{x^2} \cos \left( \dfrac{y}{x} \right) - \dfrac{y^2}{x^2} \left( -\sin \left( \dfrac{y}{x} \right) \right) \dfrac{1}{x}\\
		\therefore P_y &= Q_x
	\end{align*}
	Therefore, the function is a conservative vector field. Hence, the line integral is independent of the path.\\
	Therefore,
	\begin{align*}
		\int\limits_{C} P \dif x + Q \dif y &= \int\limits_{1}^{2} \left( P x' + Q y' \right) \dif t\\
		&= \int\limits_{1}^{2} P \dif t\\
		&= \int\limits_{1}^{2} \left( 1 - \dfrac{\pi^2}{t^2} \cos \left( \dfrac{\pi}{t} \right) \right) \dif t\\
		&= \left. \left( t + \pi \sin \left( \dfrac{\pi}{t} \right) \right) \right|_{1}^{2}\\
		&= 1 + \pi
	\end{align*}
\end{solution}

\end{document}
