\documentclass[fleqn]{article}
\usepackage{amsmath, amssymb, amsthm, esdiff}
\usepackage{commath}
\usepackage{gensymb}
\usepackage{hyperref}
\usepackage{tikz, pgfplots}
\usepackage{datetime}
\usepackage{ulem}
\usepackage{enumerate, enumitem}

\setcounter{secnumdepth}{4}

\newcommand\numberthis{\addtocounter{equation}{1}\tag{\theequation}}

\theoremstyle{definition}
\newtheorem{example}{Example}
\newtheorem{definition}{Definition}

\theoremstyle{theorem}
\newtheorem{theorem}{Theorem}

\theoremstyle{remark}
\newtheorem{remark}{Remark}

\newenvironment{solution}
{\begin{proof}[Solution]\let\qed\relax}
	{\end{proof}}

%opening
\title{Lecture 8}
\author{Aakash Jog}
\date{\formatdate{20}{11}{2014}}

\begin{document}
	
\maketitle
%\setlength{\mathindent}{0pt}

\tableofcontents

\newpage

\section{Taylor's Formula}

\begin{theorem}[Taylor's Formula] \label{Taylor's Formula}
	Let $f(x)$ be differentiable $(n+1)$ times, where $n \in \mathbb{N} \cup \{0\}$ on an open interval about $a$, and $x$ be an arbitrary point in this interval. Then, there exists a point $c$, which depends on $x$, between $a$ and $x$, s.t.
	\begin{align*}
		f(x) &= f(a) + \dfrac{f'(a)}{1!} (x-a) + \dfrac{f''(a)}{2!} (x-a)^2 + \dots + \dfrac{f^{(n)}(a)}{n!} (x-a)^n + R_n (x)
		\intertext{where}
		R_n (x) &= \dfrac{f^{(n+1)}(c)}{(n+1)!} (x-a)^{n+1}
		\intertext{is called the \emph{Lagrange remainder}}
	\end{align*}
\end{theorem}

\begin{proof}
	Let 
	\begin{align*}
		T_n(x) &= f(a) + \dfrac{f'(a)}{1!} (x-a) + \dots + \dfrac{f^{(n)}(a)}{n!} (x-a)^n \\
		R_n^0 (x) &= f(x) - T_n(x)
	\end{align*}	
	Therefore, it is ETPT that 
	\begin{equation*}
		R_n^0 (x) = R_n (x)
	\end{equation*}
	Let $x \neq a$ be a fixed point. \\
	Let
	\begin{equation*}
		g(t) = f(x) - f(t) - \dfrac{f'(t)}{1!} (x-t) + \dots + \dfrac{f^{(n)}(t)}{n!} (x-t)^n - R_n^0 (x) \dfrac{(x-t)^{n+1}}{(x-a)^{n+1}}
	\end{equation*}
	\begin{align*}
		g(x) &= 0 \\
		g(a) &= f(x) - T_n(x) - R_n^0 (x) = 0 \\
		\therefore g(x) &= g(a) 
	\end{align*}
	Also, $g(x)$ is continuous and differentiable on the interval. Hence we can apply Rolle's Theorem.\\
	Therefore, $\exists c$ between $a$ and $x$, s.t. $f'(c) = 0$.
	\begin{align*}
		g'(t) &= -f'(t) - \dfrac{f''(t)}{1!} (x-t) + \dfrac{f'(t)}{1!} - \dfrac{f'''(t)}{2!} (x-t)^2 + \dfrac{f''(t)}{2!} 2(x-t) - \dots \\
		& - \dfrac{f^{(n+1)}(t)}{n!} (x-t)^n + \dfrac{f^{(n)}(t)}{n!} (x-t)^{n-1} + R_n^0 (x) \dfrac{(n+1)(x-t)^n}{(x-a)^{n+1}} \\
		\therefore g'(t) &= -\dfrac{f^{(n+1)}(t)}{n!} (x-t)^n + R_n^0 (x) \dfrac{(n+1)(x-t)^n}{(x-a)^{n+1}} \\
		\intertext{$f'(c) = 0$}
		\therefore R_n^0 (x) &= \dfrac{f^{(n+1)}(c) (x-a)^{n+1}}{(n+1)!} \\
		\therefore R_n^0 (x) &= R_n (x)
	\end{align*}
\end{proof}

\begin{remark}
	The Lagrange Theorem is a particular case of Taylor's Formula, with $n = 0, x = b$.
\end{remark}

\begin{remark}
	If $f(x)$ is infinitely differentiable on an open interval about $a$, and $\lim\limits_{n \to \infty} R_n(x) = 0$, for any $x$ in the interval, then
	\begin{equation*}
		f(x) = f(a) + \dfrac{f'(a)}{1!} (x-a) + \dots + \dfrac{f^{(n)}(a)}{n!} (x-a)^n + \dots
	\end{equation*}
	This infinite sum is called the Taylor's series.\\
	If $a = 0$, 
	\begin{equation*}
	f(x) = f(0) + \dfrac{f'(0)}{1!} (x) + \dots + \dfrac{f^{(n)}(0)}{n!} (x)^n + \dots
	\end{equation*}
	This infinite sum is called the Maclaurin series.
\end{remark}

\begin{example}
	Calculate $e^{-0.05}$ with accuracy 0.0001.
\end{example}

\begin{solution}
	Substitute $x = -0.05$ in 
	\begin{equation*}
		e^x = 1 + \dfrac{x}{1!} + \dfrac{x^n}{n!} + \dfrac{e^c x^{n+1}}{(n+1)!}
	\end{equation*}
	where $c$ is between 0 and $x$.
	Then, $\exists c \in (-0.05, 0)$, and 
	\begin{align*}
		|R_n (-0.05)| &= \left|\dfrac{e^c(-0.05)^{n+1}}{(n+1)!}\right| \\
		&= \dfrac{e^c(0.05)^{n+1}}{(n+1)!} \\
		&< \dfrac{(0.05)^{n+1}}{(n+1)!} \\
		\intertext{If $n = 2$}
		\dfrac{(0.05)^{n+1}}{(n+1)!} &\leq 0.0001 \\
		\therefore e^{-0.05} &\approx  1+ \dfrac{(-0.05)}{1!} + \dfrac{(-0.05)^2}{2!} = 0.95125
	\end{align*}
\end{solution}

\section{L'Hospital's Rule}

\begin{theorem}[L'Hospital's Rule] \label{L'Hospital's Rule}
	Let $f(x)$ and $g(x)$ be differentiable on an open interval about $a$, except possibly at $a$ itself. Assume
	\begin{enumerate}[label = \roman*.]
		\item $\lim\limits_{x \to a} f(x) = \lim\limits_{x \to a} g(x) = 0$ or $\lim\limits_{x \to a} f(x) = \pm \infty , \lim\limits_{x \to a} g(x) = \pm \infty$
		\item $g(x) \neq 0, \forall x \neq a$ from the interval
		\item $\exists \lim\limits_{x \to a} \dfrac{f'(x)}{g'(x)}$
	\end{enumerate}
	Then, 
	\begin{equation*}
	\lim\limits_{x \to a} \dfrac{f(x)}{g(x)} = \lim\limits_{x \to a} \dfrac{f'(x)}{g'(x)}
	\end{equation*}
\end{theorem}

\begin{proof}
	For a particular case where $f$ and $g$ are differentiable at $a$, and the derivatives are continuous at $a$, $g'(a) \neq 0$, consider 
	\begin{equation*}
		\lim\limits_{x \to a} f(x) = \lim\limits_{x \to a} g(x) = 0
	\end{equation*}
	Using \nameref{Taylor's Formula} with $n = 0$,
	\begin{align*}
		\lim\limits_{x \to a} \dfrac{f(x)}{g(x)} &= \lim\limits_{x \to a} \dfrac{f(a) + \dfrac{f'(c_1)}{1!} (x-a)}{g(a) + \dfrac{g'(c_2)}{1!} (x-a)} \\
		&= \lim\limits_{x \to a} \dfrac{f'(c_1)}{g'(c_2)} \\
		\intertext{$c_1$ and $c_2$ are constrained to lie between $x$ and $a$. Therefore, as $x \to a$, $c_1 \to a$ and $c_2 \to a$.}
		&= \dfrac{f'(a)}{g'(a)} \\
		&= \lim\limits_{x \to a}\dfrac{f'(x)}{g'(x)}
	\end{align*}
\end{proof}

\begin{remark}
	\nameref{L'Hospital's Rule} can be applied repeatedly to $\lim\limits_{x \to a} \dfrac{f(x)}{g(x)}$.
\end{remark}

\begin{remark}
	\nameref{L'Hospital's Rule} is also true for one sided limits.
\end{remark}

\begin{example}
	Find
	\begin{equation*}
	\lim\limits_{x \to 0^+} x^2 \ln x
	\end{equation*}
\end{example}

\begin{solution}
	\begin{align*}
		\lim\limits_{x \to 0^+} x^2 \ln x &= \lim\limits_{x \to 0^+} \dfrac{\ln x}{\dfrac{1}{x^2}} \\
		&= \dfrac{\dfrac{1}{x}}{-2 x^{-3}} \\
		&= \lim\limits_{x \to 0^+} \dfrac{x^2}{-2} \\
		&= 0
	\end{align*}
\end{solution}

\begin{example}
	Find
	\begin{equation*}
		\lim\limits_{x \to +\infty} x^{\dfrac{1}{x}}
	\end{equation*}
\end{example}

\begin{solution}
	\begin{align*}
	\lim\limits_{x \to +\infty} x^{\dfrac{1}{x}} &= \lim\limits_{x \to +\infty} e^{\ln x^{\dfrac{1}{x}}} \\
	&= e^ {\lim\limits_{x \to +\infty} \dfrac{\ln x}{x}} \\
	&= e^ {\lim\limits_{x \to +\infty} \dfrac{\dfrac{1}{x}}{1}} \\
	&= e^0 \\
	&= 1
	\end{align*}
\end{solution}

\section{Useful Properties of Derivatives}

\begin{enumerate}
	\item $f'(x) = 0$ on $(a, b)$ iff $f(x)$ is constant on $(a, b)$.
	\item $f'(x) = g'(x)$ on $(a, b)$ iff $f(x) = g(x) + \text{constant}$ on $(a, b)$.
	\item $\exists f'(x)$ and $f(x)$ is monotonically increasing on $(a, b)$ iff $f'(x) \geq 0$.
	\item $\exists f'(x)$ and $f(x)$ is monotonically decreasing on $(a, b)$ iff $f'(x) \leq 0$.
	\item If $f'(x) > 0$ on $(a, b)$, then $f(x)$ is monotonically strongly increasing on $(a, b)$.
	\item If $f'(x) < 0$ on $(a, b)$, then $f(x)$ is monotonically strongly decreasing on $(a, b)$.
\end{enumerate}

\end{document}
