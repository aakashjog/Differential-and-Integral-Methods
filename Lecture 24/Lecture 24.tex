\documentclass[fleqn, a4paper, 12pt]{article}
\usepackage{amsmath, amssymb, amsthm, esdiff}
\usepackage[table]{xcolor}
\usepackage{todonotes, marginnote}
\usepackage{commath}
\usepackage{gensymb}
\usepackage{hyperref}
\usepackage{tikz, pgfplots}
\usetikzlibrary{calc}
\usepackage{datetime}
\usepackage{setspace}
\usepackage{ulem}
\usepackage{xfrac}
\usepackage{siunitx}

\usepackage{enumerate, enumitem}

\setcounter{secnumdepth}{4}

\newcommand\numberthis{\addtocounter{equation}{1}\tag{\theequation}}

\newcommand{\curl}{\mathrm{curl\,}}

\newcommand{\divergence}{\mathrm{div\,}}

\theoremstyle{definition}
\newtheorem{example}{Example}
\newtheorem{definition}{Definition}

\theoremstyle{theorem}
\newtheorem{theorem}{Theorem}
\newtheorem{corollary}{Corollary}

\theoremstyle{remark}
\newtheorem{remark}{Remark}
\newtheorem{case}{Case}

\newenvironment{solution}
{\begin{proof}[Solution]\let\qed\relax}
	{\end{proof}}

\makeatletter
\@addtoreset{corollary}{theorem} %resets corollary numbers after a theorem
\makeatother

%opening
\title{Lecture 24}
\author{Aakash Jog}
\date{\formatdate{20}{1}{2015}}

\begin{document}
	
\maketitle
%\setlength{\mathindent}{0pt}

\tableofcontents

\newpage

\section{Vector Fields}

\subsection{Curl}

\begin{definition}
	Let
	\begin{equation*}
		\overline{F}(x,y,z) = \left( P(x,y,z), Q(x,y,z), R(x,y,z) \right)
	\end{equation*}
	be a vector field in $\mathbb{R}^3$ and let there exist first order partial derivatives of $P$, $Q$, $R$. Then $\curl \overline{F}$ is a vector field in $\mathbb{R}^3$ which is defined as
	\begin{equation*}
		\curl \overline{F} = (R_y - Q_z, P_z - R_x, Q_x - P_y) = \nabla \times \overline{F}
	\end{equation*}
\end{definition}

\begin{theorem}
	If $\overline{F} = (P, Q, R)$ is a vector field defined of $\mathbb{R}$ with continuous first order partial derivatives of $P$, $Q$, $R$ and $\curl \overline{F} = \overline{0}$, then $\overline{F}$ is a conservative vector field. 
\end{theorem}

\subsection{Divergence}

\begin{definition}
	Let $\overline{F}(x,y,z) = \left( P(x,y,z), Q(x,y,z), R(x,y,z) \right)$ be a vector field in $\mathbb{R}^3$ and let there exist $P_x$, $Q_y$, $R_z$. Then the divergence of $\overline{F}$ is
	\begin{equation*}
		\divergence \overline{F} = P_x + Q_y + R_z = \nabla \cdot \overline{F}
	\end{equation*}
\end{definition}

\section{Stoke's and Gauss' Theorem}

\begin{definition}[Curve with positive orientation]
	Let $S$ be a surface with normal $\hat{n}$ and let $C$ be a curve which bounds $S$. Then $C$ is a curve with positive orientation with respect to $S$ if, as we walk on $C$ in this direction and with our head in the direction of $\hat{n}$, the surface $S$ is always on our left.
\end{definition}

\begin{theorem}[Stoke's Theorem]
	Let $S$ be a piecewise smooth surface with normal $\hat{n}$ and let $S$ be bounded by a curve $C$ which is piecewise smooth, simple, closed and with positive orientation with respect to $S$. Let $\overline{F}(x,y,z) = \left( P(x,y,z), Q(x,y,z), R(x,y,z) \right)$ be a vector field such that there exist continuous first order partial derivatives of $P$, $Q$, $R$ in an open domain of $\mathbb{R}^3$ which contains $S$. Then 
	\begin{equation*}
		\int\limits_{C} \overline{F} \cdot \hat{T} \dif s = \iint\limits_{S} \curl \overline{F} \cdot \hat{n} \dif S
	\end{equation*}
	\label{Stoke's Theorem}
\end{theorem}

\begin{remark}
	\nameref{Stoke's Theorem} is a generalization of Green's Theorem.
\end{remark}

\end{document}
