\documentclass[fleqn]{article}
\usepackage{amsmath, amssymb, amsthm, esdiff}
\usepackage[table]{xcolor}
\usepackage{commath}
\usepackage{gensymb}
\usepackage{hyperref}
\usepackage{tikz, pgfplots}
\usetikzlibrary{calc}
\usepackage{datetime}
\usepackage{setspace}
\usepackage{ulem}

\usepackage{enumerate, enumitem}

\setcounter{secnumdepth}{4}

\newcommand\numberthis{\addtocounter{equation}{1}\tag{\theequation}}

\theoremstyle{definition}
\newtheorem{example}{Example}
\newtheorem{definition}{Definition}

\theoremstyle{theorem}
\newtheorem{theorem}{Theorem}

\theoremstyle{remark}
\newtheorem{remark}{Remark}
\newtheorem{case}{Case}

\newenvironment{solution}
{\begin{proof}[Solution]\let\qed\relax}
	{\end{proof}}

%opening
\title{Lecture 12}
\author{Aakash Jog}
\date{\formatdate{2}{12}{2014}}

\begin{document}
	
\maketitle
%\setlength{\mathindent}{0pt}

\tableofcontents

\newpage

\section{Integration of Rational Functions}

\begin{equation*}
	\dfrac{P(x)}{Q(x)} = M(x) + \dfrac{R(x)}{Q(x)}
\end{equation*}

\begin{definition}
	A simple rational function of one of the following forms is called a basic rational function.
	\begin{align*}
		\dfrac{A}{x - \alpha} &\quad; A, \alpha \in \mathbb{R}\\
		\dfrac{A}{(x - \alpha)^n} &\quad; A, \alpha \in \mathbb{R}, n \in \mathbb{N} - \{1\}\\
		\dfrac{Ax +  B}{x^2 + px + q} &\quad; A, B, p, q \in \mathbb{R}, p^2 - 4q < 0\\
		\dfrac{Ax + B}{(x^2 + px + q)^n} &\quad; A, B, p, q \in \mathbb{R}, p^2 - 4q < 0, n \in \mathbb{N} - \{1\}
	\end{align*}
\end{definition}

\begin{theorem}
	Any simple rational function $\dfrac{R(x)}{Q(x)}$ can be represented as a sum of basic rational functions.
\end{theorem}

\begin{remark}
	\begin{align*}
		\int \dfrac{P(x)}{Q(x)} \dif x &= \int M(x) \dif x + \int \dfrac{R(x)}{Q(x)} \dif x\\
		&= \int M(x) \dif x + \int \sum (\text{basic rational function}) \dif x\\
		&= \int M(x) \dif x + \sum \int (\text{basic rational function}) \dif x
	\end{align*}
\end{remark}

\subsection{Integrals of Basic Rational Functions}

\begin{align*}
	\int \dfrac{A}{x - \alpha} \dif x &= A \ln |x - \alpha| + c
\end{align*}

\begin{align*}
	\int \dfrac{A}{(x - \alpha)^n} \dif x &= A \dfrac{(x - \alpha)^{-n+1}}{-n + 1} + c
\end{align*}

\begin{align*}
	\int \dfrac{Ax +  B}{x^2 + px + q} \dif x &= \int \dfrac{Ax + B}{\left(x + \dfrac{p}{2}\right)^2 + q - \dfrac{p^2}{4}}
	\intertext{Let $a = q - \dfrac{p^2}{4}$. Let $t = x + \dfrac{p}{2}$. Therefore, $\dif t = \dif x$.}
	\therefore \int \dfrac{Ax + B}{\left(x + \dfrac{p}{2}\right)^2 + q - \dfrac{p^2}{4}} &= \dfrac{A\left(t - \dfrac{p}{2}\right) + B}{t^2 + a^2} \dif t\\
	&= \dfrac{A}{2} \int \dfrac{2t}{t^2 + a^2} \dif t + \left(B - \dfrac{A p}{2}\right) \int \dfrac{1}{t^2 + a^2} \dif t\\
	&= \dfrac{A}{2} \ln (t^2 + a^2) + \dfrac{B - \dfrac{A P}{2}}{a} \arctan \left(\dfrac{t}{a}\right) + c\\
	&= \dfrac{A}{2} \ln (x^2 + p x + q) + \dfrac{B - \dfrac{A p}{2}}{a} \arctan \left(\dfrac{x + \dfrac{p}{2}}{a}\right) + c
\end{align*}

\begin{align*}
	\int \dfrac{Ax +  B}{(x^2 + px + q)^n} \dif x &= \int \dfrac{Ax + B}{\left(\left(x + \dfrac{p}{2}\right)^2 + q - \dfrac{p^2}{4}\right)^n}
	\intertext{Let $a = q - \dfrac{p^2}{4}$. Let $t = x + \dfrac{p}{2}$. Therefore, $\dif t = \dif x$.}
	\therefore \int \dfrac{Ax + B}{\left(\left(x + \dfrac{p}{2}\right)^2 + q - \dfrac{p^2}{4}\right)^n} &= \dfrac{A\left(t - \dfrac{p}{2}\right) + B}{(t^2 + a^2)^n} \dif t\\
	&= \dfrac{A}{2} \int \dfrac{2t}{\left(t^2 + a^2\right)^n} \dif t + \left(B - \dfrac{A p}{2}\right) \int \dfrac{1}{(t^2 + a^2)^n} \dif t\\
	&= \dfrac{A}{2} \dfrac{(t^2 + a^2)^{-n+1}}{-n + 1} + \left(B - \dfrac{A p}{2}\right) \int \dfrac{1}{(t^2 + a^2)^n} \dif t + c
\end{align*}

\subsection{Finding Basic Rational Functions}

\subsubsection{Type 1}

If 
\begin{equation*}
	Q(x) = (a_1 x + b_1) \dots (a_k x + b_k)
\end{equation*}
and the multipliers are different from each other, then,
\begin{equation*}
	\dfrac{R(x)}{Q(x)} = \dfrac{A_1}{a_1 x + b_1} + \dots + \dfrac{A_2}{a_k x + b_k}
\end{equation*}

\begin{example}
	\begin{align*}
		\int \dfrac{x^2 + 2x - 1}{2x^3 + 3x^2 - 2x} \dif x &= \int \dfrac{x^2 + 2x - 1}{x(2x^2 + 3x - 2)} \dif x\\
		&= \int \dfrac{x^2 + 2x - 1}{(x)(2x - 1)(x + 2)}\\
		\dfrac{x^2 + 2x - 1}{(x)(2x - 1)(x + 2)}&= \dfrac{A}{x} + \dfrac{B}{2x - 1} + \dfrac{C}{x + 2}\\
		&= \dfrac{A(2x - 1)(x + 2) + B(x)(x + 2) + C(x)(2x - 1)}{(x)(2x - 1)(x + 2)}\\
		&= \dfrac{x^2(2A + B + C) + x(3A + 2B - C) - 2A}{(x)(2x - 1)(x + 2)}
	\end{align*}
	Therefore,
	\begin{align*}
		2A + B + C &= 1\\
		3A + 2B - C &= 2\\
		-2A &= -1
	\end{align*}
	Therefore, 
	\begin{align*}
		A &= \dfrac{1}{2}\\
		B &= \dfrac{1}{5}\\
		C &= -\dfrac{1}{10}
	\end{align*}
	Therefore, 
	\begin{align*}
		\int \dfrac{x^2 + 2x - 1}{(x)(2x - 1)(x + 2)} &= \int \dfrac{\dfrac{1}{2}}{x} + \dfrac{\dfrac{1}{5}}{2x - 1} + \dfrac{-\dfrac{1}{10}}{x + 2}\\
		&= \dfrac{1}{2} \ln |x| + \dfrac{1}{5} \dfrac{\ln |2x-1|}{2} - \dfrac{1}{10} \ln |x + 2| + d\\
		&= \dfrac{1}{2} \ln |x| + \dfrac{1}{10} \ln \left|\dfrac{2x - 1}{x + 2}\right| + d
	\end{align*}
\end{example}

\subsubsection{Type 2}

If 
\begin{equation*}
Q(x) = (a_1 x + b_1)^m (a_2 x + b_2) \dots (a_k x + b_k)
\end{equation*}
and the multipliers are different from each other, $m \in \mathbb{N} - \{1\}$, then,
\begin{equation*}
\dfrac{R(x)}{Q(x)} = \dfrac{A_1}{a_1 x + b_1} + \dots + \dfrac{A_m}{(a_1 x + b_1)^m} + \dfrac{B_2}{a_2 x + b_2} + \dots + \dfrac{B_k}{a_k x + b_k}
\end{equation*}

\begin{example}
	\begin{align*}
		\int \dfrac{-x + 2}{x(x - 1)^2} \dif x &= \int \left(\dfrac{A_1}{x} + \dfrac{B_1}{x - 1} + \dfrac{B_2}{(x - 1)^2}\right) \dif x\\
		\dfrac{-x + 2}{x(x - 1)^2} &= \dfrac{A_1 (x - 1)^2 + B_1 (x)(x - 1) + B_2 x}{x(x - 1)^2}\\
		&= \dfrac{x^2(A_1 + B_1) + x(-2A_1 - B_1 + B_2) + A_1}{x(x - 1)^2}
	\end{align*}
	Therefore,
	\begin{align*}
		A_1 + B_1 &= 0\\
		-2A_1 - B_1 + B_2 &= -1\\
		A_1 &= 2
	\end{align*}
	Therefore, 
	\begin{align*}
		A_1 &= 2\\
		B_1 &= -2\\
		B_2 &= 1
	\end{align*}
	Therefore,
	\begin{align*}
		\int \dfrac{-x + 2}{x(x - 1)^2} \dif x &= \int \left(\dfrac{2}{x} + \dfrac{-2}{x - 1} + \dfrac{1}{(x - 1)^2}\right) \dif x\\
		&= 2 \ln |x| - 2 \ln |x - 1| - \dfrac{1}{x - 1} + c\\
		&= 2 \ln \left|\dfrac{x}{x - 1}\right| - \dfrac{1}{x - 1} + x
	\end{align*}
\end{example}

\subsubsection{Type 3}

If 
\begin{equation*}
Q(x) = (ax^2 + bx + c) (a_2 x + b_2) \dots (a_k x + b_k)
\end{equation*}
and the multipliers are different from each other, $b^2 - 4ac < 0$, then,
\begin{equation*}
\dfrac{R(x)}{Q(x)} = \dfrac{Ax + B}{ax^2 + bx + c} + \dfrac{A_2}{a_2 x + b_2} + \dots + \dfrac{A_k}{a_k x + b_k}
\end{equation*}

\begin{example}
	\begin{align*}
		\int \dfrac{2x^2 - x + 4}{x^3 + 4x} \dif x &= \int \dfrac{2x^2 - x + 4}{x(x^2 + 4)} \dif x\\
		&= \int \left(\dfrac{A}{x} + \dfrac{Bx + C}{x^2 + 4}\right) \dif x\\
		\dfrac{2x^2 - x + 4}{x^3 + 4x} &= \dfrac{A(x^2 + 4) + (Bx + c)x}{x(x^2 + 4)}\\
		&= \dfrac{x^2(A + B) + x(C) + 4A}{x(x^2 + 4)}
	\end{align*}
	Therefore,
	\begin{align*}
		A + B &= 2\\
		C &= -1\\
		4A &= 4
	\end{align*}
	Therefore, 
	\begin{align*}
		A &= 1\\
		B &= 1\\
		C &= -1
	\end{align*}
	Therefore,
	\begin{align*}
		\int \dfrac{2x^2 - x + 4}{x(x^2 + 4)} \dif x &= \int \left(\dfrac{1}{x} + \dfrac{x - 1}{x^2 + 4}\right) \dif x\\
		&= \ln |x| + \int \dfrac{x}{x^2 + 4} \dif x - \int \dfrac{1}{x^2 + 4} \dif x + d\\
		&= \ln |x| + \dfrac{1}{2} \ln (x^2 + 4) - \dfrac{1}{2}\arctan\left(\dfrac{x}{2}\right) + d
	\end{align*}
\end{example}

\subsubsection{Type 4}

If 
\begin{equation*}
Q(x) = (ax^2 + bx + c)^m (a_2 x + b_2) \dots (a_k x + b_k)
\end{equation*}
and the multipliers are different from each other, $m \in \mathbb{N} - \{1\}$, $b^2 - 4ac < 0$, then,
\begin{equation*}
\dfrac{R(x)}{Q(x)} = \dfrac{A_1 x + B_1}{ax^2 + bx + c} + \dots + \dfrac{A_m x + B_m}{(ax^2 + bx + c)^m} + \dfrac{C_2}{a_2 x + b_2} + \dots + \dfrac{C_k}{a_k x + b_k}
\end{equation*}

\end{document}
