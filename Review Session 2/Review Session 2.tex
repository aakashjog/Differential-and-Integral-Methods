\documentclass[fleqn, a4paper, 12pt]{article}
\usepackage{amsmath, amssymb, amsthm, esdiff}
\usepackage[table]{xcolor}
\usepackage{todonotes, marginnote}
\usepackage{commath}
\usepackage{gensymb}
\usepackage{hyperref}
\usepackage{tikz, pgfplots}
\usetikzlibrary{calc}
\usepackage{datetime}
\usepackage{setspace}
\usepackage{ulem}
\usepackage{xfrac}
\usepackage{siunitx}

\usepackage{enumerate, enumitem}

\setcounter{secnumdepth}{4}

\newcommand\numberthis{\addtocounter{equation}{1}\tag{\theequation}}

\newcommand{\curl}{\mathrm{curl\,}}

\newcommand{\divergence}{\mathrm{div\,}}

\theoremstyle{definition}
\newtheorem{example}{Example}
\newtheorem{definition}{Definition}

\theoremstyle{theorem}
\newtheorem{theorem}{Theorem}
\newtheorem{corollary}{Corollary}

\theoremstyle{remark}
\newtheorem{remark}{Remark}
\newtheorem{case}{Case}

\newenvironment{solution}
{\begin{proof}[Solution]\let\qed\relax}
	{\end{proof}}

\makeatletter
\@addtoreset{corollary}{theorem} %resets corollary numbers after a theorem
\makeatother

%opening
\title{Review Session 2}
\author{Aakash Jog}
\date{\formatdate{4}{2}{2015}}

\begin{document}
	
\maketitle
%\setlength{\mathindent}{0pt}

\begin{example}
	\begin{equation*}
		F(x,y) = \left( - \dfrac{y}{x^2 + y^2} + 4y + 3 , \dfrac{x}{x^2 + y^2} + 4x + 4y - 2 \right)
	\end{equation*}
	Calculate $\displaystyle \int\limits_{C} F \dif r$ when
	\begin{enumerate}
		\item $c = (x - 10)^2 + (y - 7)^2 = 1$, in negative direction
		\item $c = x^2 + y^2 = 4$, in positive direction
	\end{enumerate}
\end{example}

\begin{solution}
	\begin{align*}
		P &= -\dfrac{y}{x^2 + y^2} + 4y + 3\\
		Q &= \dfrac{x}{x^2 + y^2} + 4x + 4y - 2
	\end{align*}
	Therefore,
	\begin{align*}
		Q_x &= \dfrac{x^2 + y^2 - 2x^2}{(x^2 + y^2)^2} + 4\\
		P_y &= -\dfrac{x^2 + y^2 - 2y^2}{(x^2 + y^2)^2} + 4
	\end{align*}
	Therefore, as $P_y = Q_x$, the field is conservative in $c = (x - 10)^2 + (y - 7)^2 = 1$. Therefore, the integral is over a closed curve is $0$.
	\begin{equation*}
		\therefore \int\limits_{C} F \dif r = 0
	\end{equation*}
	In the second case, the function is not defined at $(0,0)$. Therefore, the above method cannot be used.\\
	Therefore, by parametrization,
	\begin{align*}
		c(t) &= (2 \cos t, 2 \sin t)\\
		\therefore c'(t) &= (-2 \sin t, 2 \cos t) \dif t
	\end{align*}
	where $t : 0 \to 2 \pi$.\\
	Therefore,
	\begin{align*}
		\int\limits_{C} F \dif r &= \quad \int\limits_{0}^{2 \pi} \left( -\dfrac{2 \sin t}{4} + 4 \cdot 2 \sin t + 3 \right) \cdot (-2 \sin t \dif t) \\
		&\quad+ \int\limits_{0}^{2 \pi} \left( \dfrac{2 \cos t}{4} + 4 \cdot 2 \cos t + 4 \cdot 2 \sin t - 2 \right) \cdot (2 \cos t \dif t)
	\end{align*}
\end{solution}

\begin{example}
	Find the minimum $n$, such that $e = \sum_{k = 1}^{n} \dfrac{1}{k!}$ with error $< 0.001$
\end{example}

\begin{solution}
	\begin{align*}
		e^x &= \sum_{k = 1}^{n} \dfrac{x^k}{k!} + \dfrac{e^c x^{n+1}}{(n+1)!}\\
		\therefore e &= \sum_{x = 1}^{n} \dfrac{1}{k!} + \dfrac{e^c}{(n + 1)!}
	\end{align*}
	To find the minimum $n$, we need to check both upper and lower conditions for the error term. Therefore,
	\begin{equation*}
		\dfrac{1}{(n+1)!} < \dfrac{e^c}{(n+1)!} < \dfrac{e}{(n+1)!}
	\end{equation*}
	Therefore, solving,
	\begin{align*}
		n = 6
	\end{align*}
\end{solution}

\begin{example}
	Find the volume of the body bounded by $x = 0$, $x + y = 8$, $z = \dfrac{3}{4} y$, $z = \dfrac{3}{2} \sqrt{y}$.
\end{example}

\begin{solution}
	$z = \dfrac{3}{2} \sqrt{y}$ is above $z = \dfrac{3}{4} y$ for $y \in (0, 4)$. $z = \dfrac{3}{4} y$ is above $z = \dfrac{3}{2} \sqrt{y}$  for $y \in (4, 8)$.\\
	Therefore,
	\begin{align*}
		\iiint\limits_{E} \dif V &= \int\limits_{0}^{4} \int\limits_{0}^{8 - y} \int\limits_{\frac{3}{4} y}^{\frac{3}{2} \sqrt{y}} \dif z \dif x \dif y
	\end{align*}
\end{solution}

\begin{example}
	What is larger, $e^{\pi}$ or $\pi^e$?
\end{example}

\begin{solution}
	\begin{align*}
		\pi^e &\lessgtr e^\pi\\
		\iff e \ln \pi &\lessgtr \pi
	\end{align*}
	Let
	\begin{align*}
		f(x) &= e \ln x - x\\
		\therefore f'(x) &= \dfrac{e}{x} - 1
	\end{align*}
	Therefore,
	\begin{align*}
		f'(x) &> 0 &\iff&& x &< e\\
		f'(x) &< 0 &\iff&& x &> e\\
	\end{align*}
	Therefore, as $\pi > e$,
	\begin{align*}
		f(\pi) &< f(e) < 0\\
		\therefore e \ln \pi - \pi  &< 0\\
		\therefore e \ln \pi &< \pi\\
		\therefore \pi^e &< e^\pi
	\end{align*}
\end{solution}

\begin{example}
	Find
	\begin{align*}
		\int\limits_{C} -\dfrac{y}{x^2} \sin \dfrac{y}{x} \dif x + \dfrac{1}{x} \sin \dfrac{y}{x} \dif y
	\end{align*}
	where $C$ is the union of $y = (x - 1)^2 + 1$ in $[1,2]$ and $y = (x - 3)^2 + 1$ in $[2,3]$, both directed clockwise.
\end{example}

\begin{solution}
	\begin{align*}
		Q_x &= -\dfrac{1}{x^2} \sin \dfrac{y}{x} + \dfrac{1}{x} \cos \dfrac{y}{x} \cdot \left( -\dfrac{y}{x^2} \right)\\
		P_y &= -\dfrac{1}{x^2} \sin \dfrac{y}{x} + \dfrac{1}{x} \cos \dfrac{y}{x} \cdot \left( -\dfrac{y}{x^2} \right)
	\end{align*}
	Therefore, the field is conservative for $x \neq 0$.\\
	Therefore, the integral is path independent. Therefore, parametrizing and solving,
	\begin{align*}
		\int\limits_{C} -\dfrac{y}{x^2} \sin \dfrac{y}{x} \dif x + \dfrac{1}{x} \sin \dfrac{y}{x} \dif y &= - \cos \dfrac{1}{3} + \cos 1
	\end{align*}
\end{solution}

\end{document}