\documentclass[fleqn]{article}
\usepackage{amsmath, amssymb, amsthm, esdiff}
\usepackage{commath}
\usepackage{gensymb}
\usepackage{hyperref}
\usepackage{tikz, pgfplots}
\usetikzlibrary{calc}
\usepackage{datetime}
\usepackage{ulem}
\usepackage{enumerate, enumitem}

\setcounter{secnumdepth}{4}

\newcommand\numberthis{\addtocounter{equation}{1}\tag{\theequation}}

\theoremstyle{definition}
\newtheorem{example}{Example}
\newtheorem{definition}{Definition}

\theoremstyle{theorem}
\newtheorem{theorem}{Theorem}

\theoremstyle{remark}
\newtheorem{remark}{Remark}
\newtheorem{case}{Case}

\newenvironment{solution}
{\begin{proof}[Solution]\let\qed\relax}
	{\end{proof}}

%opening
\title{Lecture 10}
\author{Aakash Jog}
\date{\formatdate{25}{11}{2014}}

\begin{document}
	
\maketitle
%\setlength{\mathindent}{0pt}

\tableofcontents

\newpage

\section{Full Investigation of Functions}

\begin{enumerate}
	\item Domain of definition of $f$
	\item Points of intersection of $y = f(x)$ with $x$-axis and $y$-axis
	\item Symmetry and periodicity
	\item Extrema points
	\item Monotonicity
	\item Convexity
	\item Inflection points
	\item Asymptotes (vertical and oblique)
	\item Graph
\end{enumerate}

\begin{example}
	Investigate
	\begin{equation*}
		y = f(x) =\dfrac{(x-1)^3}{(x+1)^2}
	\end{equation*}
\end{example}

\begin{solution}
	\begin{align*}
		D(f) &= \mathbb{R} - \{-1\}
	\end{align*}
	\begin{align*}
		y &= 0 &\implies x &= 1\\
		x &= 0 &\implies y &= -1
	\end{align*}
	The function is not periodic.
	\begin{align*}
		f(-x) &\neq f(x) \\
				&\neq -f(x) 
	\end{align*}
	Therefore, the function is not symmetric.
	\begin{align*}
		f'(x) &= \dfrac{(x-1)^2 (x+5)}{(x+1)^3}
	\end{align*}
	Therefore, $x = -5$ is a local maximum point.\\
	The function is monotonically increasing in $(-\infty, -5) \cup (-1, +\infty)$ and is monotonically decreasing in $(-5, -1)$.
	\begin{align*}
		f''(x) &= \dfrac{24(x-1)}{(x+1)^4}
	\end{align*}
	Therefore, the function is convex upwards in $(-\infty, -1) \cup (-1, 1)$ and convex downwards in $(1, \infty)$.
	\begin{align*}
		\lim\limits_{x \to -1^-} \dfrac{(x-1)^3}{(x+1)^2} &= \dfrac{-8}{+0}\\
		&= -\infty\\
		\lim\limits_{x \to -1^+} \dfrac{(x-1)^3}{(x+1)^2} &= \dfrac{-8}{+0}\\
		&= -\infty
	\end{align*}
	Therefore, $x = -1$ is a vertical asymptote of $f(x)$.
	\begin{align*}
		a_1 &= \lim\limits_{x \to +\infty} \dfrac{f(x)}{x} &= 1\\
		b_1 &= \lim\limits_{x \to +\infty} \left(f(x) - a_1 x\right) &= -5\\
		a_2 &= \lim\limits_{x \to -\infty} \dfrac{f(x)}{x} &= 1\\
		b_2 &= \lim\limits_{x \to -\infty} \left(f(x) - a_1 x\right) &= -5
	\end{align*}
	Therefore, $y = x - 5$ is an oblique asymptote of the function, at $+\infty$ and $-\infty$.
	
%	\begin{tikzpicture}
%		\draw plot ;
%	\end{tikzpicture}
\end{solution}

\begin{example}
	Investigate
	\begin{equation*}
		f(x) = 
		\begin{cases}
			x^2 \sin x &\quad; x \neq 0\\
			0 &\quad; x = 0\\
		\end{cases}
	\end{equation*}
\end{example}

\begin{solution}
	\begin{align*}
		f'(0) &= \lim\limits_{\Delta x \to 0} \dfrac{f(0 + \Delta x) - f(0)}{\Delta x}\\
		&= \lim\limits_{\Delta x \to 0} \dfrac{(\Delta x)^2 \sin \dfrac{1}{\Delta x}}{\Delta x}\\
		&= \lim\limits_{\Delta x \to 0} \Delta x \sin \dfrac{1}{\Delta x}\\
		&= 0
	\end{align*}
	Therefore $x = 0$ is a critical point of $f(x)$, but it is not a local extremum point.
\end{solution}

\end{document}
