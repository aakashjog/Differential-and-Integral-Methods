\documentclass[fleqn, a4paper, 12pt]{article}
\usepackage{amsmath, amssymb, amsthm, esdiff}
\usepackage[table]{xcolor}
\usepackage{commath}
\usepackage{gensymb}
\usepackage{hyperref}
\usepackage{tikz, pgfplots}
\usetikzlibrary{calc}
\usepackage{datetime}
\usepackage{setspace}
\usepackage{ulem}
\usepackage{xfrac}

\usepackage{enumerate, enumitem}

\setcounter{secnumdepth}{4}

\newcommand\numberthis{\addtocounter{equation}{1}\tag{\theequation}}

\theoremstyle{definition}
\newtheorem{example}{Example}
\newtheorem{definition}{Definition}

\theoremstyle{theorem}
\newtheorem{theorem}{Theorem}
\newtheorem{corollary}{Corollary}

\theoremstyle{remark}
\newtheorem{remark}{Remark}
\newtheorem{case}{Case}

\newenvironment{solution}
{\begin{proof}[Solution]\let\qed\relax}
	{\end{proof}}

\makeatletter
\@addtoreset{corollary}{theorem} %resets corollary numbers after a theorem
\makeatother

%opening
\title{Lecture 18}
\author{Aakash Jog}
\date{\formatdate{30}{12}{2014}}

\begin{document}
	
\maketitle
%\setlength{\mathindent}{0pt}

\tableofcontents

\newpage

\section{Local Extrema}

\begin{theorem}[A necessary condition for local extrema existence]
	If the function $z = f(x, y)$ has a local extrema at the point $(a, b)$ and $\exists f_x (a, b)$ and $\exists f_y (a, b)$ then $f_x (a, b) = f_y (a, b) = 0$
\end{theorem}

\begin{example}
	\begin{equation*}
		z = x^2 + y^2
	\end{equation*}
\end{example}

\begin{solution}
	\begin{align*}
		f(x, y) \geq f(0, 0)
	\end{align*}
	Therefore, $(0,0)$ is a point of local minimum.
	\begin{align*}
		f_x &= 2x\\
		f_y &= 2y
	\end{align*}
	Therefore,
	\begin{equation*}
		f_x(0,0) = f_y(0,0) = 0
	\end{equation*}
\end{solution}

\begin{example}
	\begin{equation*}
		z = \sqrt{x^2 + y^2}
	\end{equation*}
\end{example}

\begin{solution}
	\begin{align*}
		\lim\limits_{\Delta x \to 0} \dfrac{f(0 + \Delta x, 0) - f(0, 0)}{\Delta x} &= \lim\limits_{\Delta x \to 0} \dfrac{\sqrt{(\Delta x)^2}}{\Delta x}\\
		&= \pm 1
	\end{align*}
	Therefore, the limit does not exist.
\end{solution}

\begin{definition}[Critical point]
	Let the function $z = f(x, y)$ be defined on some open neighbourhood of $(a, b)$. The point $(a, b)$ is called a critical point of $z = f(x, y)$ if $f_x(a, b) = f_y(a, b) = 0$ or at least one of the partial derivative $f_x(a, b)$ and $f_y(a, b)$ does not exist.
\end{definition}

\begin{remark}
	Every extremum point is a critical point but the converse is not true.
\end{remark}

\begin{example}
	Is $(0,0)$ an local extremum point of
	\begin{equation*}
		z = f(x, y) = y^2 - z^2
	\end{equation*}
	?
\end{example}
\begin{solution}
	\begin{align*}
		f_x(0,0) &= 0\\
		f_y(0,0) &= 0
	\end{align*}
	Therefore, $(0,0)$ is a critical point.\\
	If possible let $(0,0)$ be a local minimum point.\\
	Then, $f(x,y) \geq f(0,0)$ in some neighbourhood of $(0,0)$.\\
	Therefore,
	\begin{align*}
		y^2 - x^2 &\geq 0
	\end{align*}
	For any point of the form $(x, 0)$, this is a contradiction.\\
	Therefore $(0,0)$ is not a local minimum point.\\
	Similarly, $(0,0)$ is not a local maximum point.
\end{solution}

\begin{theorem}[A sufficient condition for local extrema point]
	Assume that there exist second order partial derivates of $z = f(x,y)$, they are continuous on some open neighbourhood of $(a,b)$ and $f_x(a,b) =f_y(a,b) = 0$. Denote 
	\begin{equation*}
		D(a, b) = f_{xx}(a,b) f_{yy}(a,b) - \left( f_{xy}(a,b) \right)^2
	\end{equation*}
	\begin{enumerate}
		\item If $D(a,b) > 0$ and $f_{xx} < 0$ then $(a,b)$ is a local maximum point.
		\item If $D(a,b) > 0$ and $f_{xx} > 0$ then $(a,b)$ is a local minimum point.
		\item If $D(a,b) < 0$ then $(a,b)$ is called a saddle point.
	\end{enumerate}
\end{theorem}

\begin{example}
	Find all critical points of 
	\begin{equation*}
		z = f(x,y) = x^4 + y^4 - 4xy + 1
	\end{equation*}
	and classify them.
\end{example}

\begin{solution}
	\begin{align*}
		f_x(x,y) &= 4x^3 - 4y\\
		f_y(x,y) &= 4y^3 - 4x
	\end{align*}
	For critical points,
	\begin{align*}
		f_x(x,y) &= 0\\
		f_y(x,y) &= 0
	\end{align*}
	Solving, $(0,0)$, $(1,1)$, $(-1,-1)$ are critical points.
	\begin{align*}
		f_{xx}(x,y) &= 12x^2\\
		f_{xy}(x,y) &= -4\\
		f_{yy}(x,y) &= 12y^2\\
		\therefore D(x,y) &= 144 x^2 y^2 - 16
	\end{align*}
	For $(0,0)$,
	\begin{align*}
		D &= -16
	\end{align*}
	Therefore, $(0,0)$ is a saddle point.\\
	For $(1,1)$,
	\begin{align*}
		D &= 144-16
	\end{align*}
	Therefore, $(1,1)$ is a local minimum point.\\
	For $(-1,-1)$,
	\begin{align*}
		D &= 144-16
	\end{align*}
	Therefore, $(-1,-1)$ is a local minimum point.\\
\end{solution}

\section{Global Extrema}

\subsection{Algorithm for Finding Maxima and Minima of a Function}

\begin{enumerate}[label = Step \arabic*]
	\item Find all critical points of $f(x,y)$ on the domain, excluding the end points. \label{Step 1}
	\item Calculate the values of $f(x,y)$ at the critical points. \label{Step 2}
	\item Calculate the values of $f(x,y)$ at the end points of the domain. \label{Step 3}
	\item Select the maximum and minimum values from \ref{Step 2} and \ref{Step 3}
\end{enumerate}

\begin{example}
	Find the global maxima and minima of
	\begin{equation*}
		z = x^2 - 2xy + 2y
	\end{equation*}
	in the domain
	\begin{equation*}
		D = 
		\left\lbrace
			(x,y) \left| 0 \leq x \leq 3, 0 \leq y \leq -\dfrac{2}{3} x + 2 \right.
		\right\rbrace
	\end{equation*}
\end{example}

\begin{solution}
	\begin{align*}
		f_x(x,y) &= 0\\
		\therefore 2x - 2y &= 0\\
		f_y(x,y) &= 0\\
		\therefore -2x + 2 &= 0
	\end{align*}
	Therefore, $(1,1)$ is a critical point in $D$.\\
	The boundary of $D$ is $L_1 \cup L_2 \cup L_3$, where
	\begin{align*}
		L_1 &: y = 0, 0 \leq x \leq 3\\
		L_2 &: x = 0, 0 \leq y \leq 2\\
		L_3 &:
	\end{align*}
	Therefore,\\
	over $L_1$,
	\begin{align*}
		f(x,y) &= x^2\\
		\therefore \min_{L_1} f &= f(0,0) = 0\\
		\therefore \max_{L_1} f &= f(3,0) = 9
	\end{align*}
		over $L_2$,
	\begin{align*}
		f(x,y) &= 2y\\
		\therefore \min_{L_2} f &= f(0,0) = 0\\
		\therefore \max_{L_2} f &= f(0,2) = 4
	\end{align*}
		over $L_3$,
	\begin{align*}
		f(x,y) &= x^2 - 2x \left(-\dfrac{2}{3} x + 2\right) + 2 \left(-\dfrac{2}{3} x + 2\right)\\
		&= \dfrac{7}{3} x^2 - \dfrac{16}{3} x + 4\\
		\therefore f' &= \dfrac{14}{3} x - \dfrac{16}{3}\\
		\therefore f'\left(\dfrac{8}{7}\right) &= 0\\
		\therefore f\left(\dfrac{8}{7}, \dfrac{26}{21}\right) &= 0.952\\
		\therefore \min_{L_3} f &= f\left(\dfrac{8}{7}, \dfrac{26}{21}\right) = 0.952\\
		\therefore \max_{L_3} f &= f(3,0) = 9
	\end{align*}
	Therefore,
	\begin{align*}
		\therefore \min_{D} f &= f(0,0) = 0\\
		\therefore \max_{D} f &= f(3,0) = 9
	\end{align*}
\end{solution}

\end{document}