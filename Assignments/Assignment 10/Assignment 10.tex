\date{}
\documentclass[fleqn, a4paper, draft]{amsart}
\usepackage[top=1in, left=1in, right=1in, bottom=1in]{geometry}
\usepackage{amsmath, amssymb, commath}
\usepackage{gensymb}
\usepackage{tikz, pgfplots}
\usepackage{datetime}
\usepackage{ulem}
\usepackage{xfrac}
\usepackage{multicol}
\usepackage{enumerate}
%\usepackage{background}
\setcounter{secnumdepth}{4}
\newcommand\numberthis{\addtocounter{equation}{1}\tag{\theequation}}
\renewcommand{\thesubsection}{(\alph{subsection})}
\renewcommand{\thesection}{(\arabic{section})}

%section headings on left
\makeatletter
\def\specialsection{\@startsection{section}{1}%
	\z@{\linespacing\@plus\linespacing}{.5\linespacing}%
	%  {\normalfont\centering}}% DELETED
	{\normalfont}}% NEW
\def\section{\@startsection{section}{1}%
	\z@{.7\linespacing\@plus\linespacing}{.5\linespacing}%
	%  {\normalfont\scshape\centering}}% DELETED
	{\normalfont\scshape}}% NEW
\makeatother

%opening
\title{Differential and Integral Methods - Exercise 10}
\author{Aakash Jog\\
	ID: 989323563}
%\date{\formatdate{24}{12}{2014}}

\begin{document}
	
\maketitle
\setlength{\mathindent}{0pt}

%\tableofcontents

%\begin{multicols}{2}
	
\section{Differentiate the following functions using the fundamental theorem of calculus:}

\subsection{$\displaystyle \int\limits_{0}^{x} \sin (t^2) \dif t$}

\begin{align*}
	\dod{}{x} \left( \int\limits_{0}^{x} \sin (t^2) \dif t \right) &= \sin (x^2)
\end{align*}

\subsection{$\displaystyle \int\limits_{x^2}^{x^3} \sqrt{1 + t^2} \dif t$}

\begin{align*}
	\int\limits_{x^2}^{x^3} \sqrt{1 + t^2} \dif t &= \int\limits_{x^2}^{a} \sqrt{1 + t^2} \dif t + \int\limits_{a}^{x^3} \sqrt{1 + t^2} \dif t\\
	&= -\int\limits_{a}^{x^2} \sqrt{1 + t^2} \dif t + \int\limits_{x^2}^{x^3} \sqrt{1 + t^2} \dif t\\
	\dod{}{x} \left( \int\limits_{x^2}^{x^3} \sqrt{1 + t^2} \dif t \right) &= - \sqrt{1 + {x^2}^2} \cdot 2 x + \sqrt{1 + {x^2}^2} \cdot 3 x^2\\
	&= \sqrt{1 + x^4}(3 x^2 - 2 x)
\end{align*}

\section{Calculate the volume of the body obtained by rotating the upper half of the circle $y = \sqrt{r^2 - x^2}$ around the $x$-axis.}

\begin{align*}
	V &= \pi \int\limits_{-r}^{r} (f(x))^2 \dif x\\
	&= \pi \int\limits_{-r}^{r} (r^2 - x^2) \dif x\\
	&= \left. \pi r^2 x - \pi \dfrac{x^3}{3} \right\rvert_{-r}^{r}\\
	&= \pi r^3 - (-\pi r^3) - \left( \pi \dfrac{r^3}{3} - \pi \dfrac{-r^3}{3} \right)\\
	&= 2 \pi r^3 - \pi \dfrac{2 r^3}{3}\\
	&= \dfrac{4}{3} \pi r^3
\end{align*}

\section{Calculate the volume of the body obtained by rotating the upper half of the circle $y = \sqrt{r^2 - x^2}$ around the $y$-axis.}

\begin{align*}
	y &= \sqrt{r^2 - x^2}\\
	\therefore x &= \sqrt{r^2 - y^2}\\
	\therefore V &= \pi \int\limits_{0}^{r} (f(y))^2 \dif y\\
	&= \pi \int\limits_{0}^{r} (r^2 - y^2) \dif y\\
	&= \left. \pi r^2 x - \pi \dfrac{y^3}{3} \right\rvert_{0}^{r}\\
	&= \pi r^3 - \pi \dfrac{r^3}{3}\\
	&= \pi r^3 - \pi \dfrac{r^3}{3}\\
	&= \dfrac{2}{3} \pi r^3
\end{align*}

\section{Calculate the volume of the rotation body, obtained by rotating the area bounded by $f(x) = x^2$, $g(x) = \sqrt{x}$ around the $x$-axis.}

The graphs of $y = f(x)$ and $f = g(x)$ intersect at $(0,0)$ and $(1,1)$.

\begin{align*}
	V &= \left\lvert \pi \int\limits_{0}^{1} (f(x))^2 \dif x - \pi \int\limits_{0}^{1} (g(x))^2 \dif x \right\rvert\\
	&= \pi \int\limits_{0}^{1} x \dif x - \pi \int\limits_{0}^{1} x^4 \dif x\\
	&= \pi \dfrac{1^2}{2} - \pi\dfrac{1^5}{5}\\
	&= \pi \dfrac{1}{2} - \pi \dfrac{1}{5}\\
	&= \dfrac{3 \pi}{10}
\end{align*}

\section{Calculate the improper integral $\displaystyle \int\limits_{-\infty}^{\infty} \dfrac{\dif x}{1 + x^2}$}

\begin{align*}
	\int\limits_{-\infty}^{\infty} \dfrac{\dif x}{1 + x^2} &= \int\limits_{-\infty}^{0} \dfrac{\dif x}{1 + x^2} + \int\limits_{0}^{\infty} \dfrac{\dif x}{1 + x^2}\\
	&= \lim\limits_{t \to -\infty} \int\limits_{t}^{0} \dfrac{\dif x}{1 + x^2} + \lim\limits_{u \to \infty} \int\limits_{0}^{u} \dfrac{\dif x}{1 + x^2}\\
	&= \lim\limits_{t \to -\infty} \left. \tan^{-1} x \right\rvert_{t}^{0} + \lim\limits_{u \to \infty} \left. \tan^{-1} x \right\rvert_{0}^{u}\\
	&= \lim\limits_{t \to -\infty} -\tan^{-1} t + \lim\limits_{u \to \infty} \tan^{-1} u\\
	&= \dfrac{\pi}{2} + \dfrac{\pi}{2}\\
	&= \pi
\end{align*}

\section{Check convergence of the following integrals:}

\subsection{$\displaystyle \int\limits_{-\infty}^{\infty} \dfrac{\sin x}{x^2} \dif x$}

\begin{align*}
	\int\limits_{-\infty}^{\infty} \dfrac{\sin x}{x^2} \dif x &= \int\limits_{-\infty}^{-1} \dfrac{\sin x}{x^2} \dif x + \int\limits_{-1}^{0} \dfrac{\sin x}{x^2} \dif x + \int\limits_{0}^{1} \dfrac{\sin x}{x^2} \dif x + \int\limits_{1}^{\infty} \dfrac{\sin x}{x^2} \dif x
\end{align*}

\begin{align*}
	\dfrac{\sin x}{x^2} \dif x &\leq \dfrac{1}{x^2}
\end{align*}
Therefore, as $\dfrac{1}{x^2}$ converges in $(1, \infty)$ and $(-\infty, -1)$, $\dfrac{\sin x}{x^2}$ converges in $(1, \infty)$ and $(-\infty, -1)$.\\
However, the limit $\displaystyle \lim\limits_{b \to 0^+} \int\limits_{b}^{1} \dfrac{\sin x}{x^2} \dif x$ does not exist. Therefore the integral diverges.

\subsection{$\displaystyle \int\limits_{0}^{\infty} \dfrac{\dif x}{\sqrt{3 x^4 + x^2 + x}} \dif x$}

\begin{align*}
	\int\limits_{0}^{\infty} \dfrac{\dif x}{\sqrt{3 x^4 + x^2 + x}} \dif x&= \int\limits_{0}^{1} \dfrac{\dif x}{\sqrt{3 x^4 + x^2 + x}} \dif x + \int\limits_{1}^{\infty} \dfrac{\dif x}{\sqrt{3 x^4 + x^2 + x}} \dif x
\end{align*}

\begin{align*}
	\dfrac{\dif x}{\sqrt{3 x^4 + x^2 + x}} &\leq \dfrac{1}{\sqrt{x}}\\
	\lim\limits_{x \to 0^+} \dfrac{\sqrt{x}}{\sqrt{3 x^4 + x^2 + x}} &= 1
\end{align*}
Therefore, by the second comparison test, $\displaystyle \int\limits_{0}^{1} \dfrac{\dif x}{\sqrt{3 x^4 + x^2 + x}} \dif x$ converges.\\

\begin{align*}
	\dfrac{\dif x}{\sqrt{3 x^4 + x^2 + x}} &\leq \dfrac{1}{x^2}
\end{align*}
Therefore, by the first comparison test, as $\dfrac{1}{x^2}$ converges, $\displaystyle \int\limits_{1}^{\infty} \dfrac{\dif x}{\sqrt{3 x^4 + x^2 + x}} \dif x$ converges.\\
Hence, $\displaystyle \int\limits_{0}^{\infty} \dfrac{\dif x}{\sqrt{3 x^4 + x^2 + x}} \dif x$ converges.

\subsection{$\displaystyle \int\limits_{1}^{\infty} \dfrac{e^{-x} \sin 2x}{\sqrt{1 + x^4}}\dif x$}

\begin{align*}
	\dfrac{e^{-x} \sin 2x}{\sqrt{1 + x^4}}\dif x &\leq \dfrac{1}{x^2}
\end{align*}
Therefore, by the first comparison test, as $\dfrac{1}{x^2}$ converges, $\displaystyle \int\limits_{1}^{\infty} \dfrac{e^{-x} \sin 2x}{\sqrt{1 + x^4}}\dif x$ converges.

\subsection{$\displaystyle \int\limits_{0}^{\infty} \dfrac{\arctan x}{\sqrt{x + x^3}} \dif x$}

\begin{align*}
	\int\limits_{0}^{\infty} \dfrac{\arctan x}{\sqrt{x + x^3}} \dif x&= \int\limits_{0}^{1} \dfrac{\arctan x}{\sqrt{x + x^3}} \dif x + \int\limits_{1}^{\infty} \dfrac{\arctan x}{\sqrt{x + x^3}} \dif x
\end{align*}

\begin{align*}
	\dfrac{\arctan x}{\sqrt{x + x^3}} &= \dfrac{\arctan x}{\sqrt{x}\sqrt{1 + x^2}}\\
	\therefore \dfrac{\dfrac{\arctan x}{\sqrt{x + x^3}}}{\dfrac{\arctan x}{\sqrt{x}}} &= \dfrac{1}{\sqrt{1 + x^2}}\\
	\lim\limits_{x \to 0^+} \dfrac{1}{\sqrt{1 + x^2}} &= 1\\
	\dfrac{\arctan x}{\sqrt{x}} &\leq \dfrac{1}{\sqrt{x}}
\end{align*}
Therefore, as $\displaystyle \int\limits_{0}^{1} \dfrac{1}{\sqrt{x}} \dif x$ converges, $\displaystyle \int\limits_{0}^{1} \dfrac{\arctan x}{\sqrt{x}} \dif x$ converges. 
Hence, $\displaystyle \int\limits_{0}^{1} \dfrac{\arctan x}{\sqrt{x + x^3}} \dif x$ converges.

\subsection{$\displaystyle \int\limits_{0}^{1} \dfrac{\arctan x}{x^2} \dif x$}

\begin{align*}
	\dfrac{\arctan x}{x^2} &\leq \dfrac{1}{x^2}\\
	\lim\limits_{x \to 0} \dfrac{\dfrac{\arctan x}{x^2}}{\dfrac{1}{x^2}} &= 1
\end{align*}
Therefore, $\displaystyle \int\limits_{0}^{1} \dfrac{\arctan x}{x^2}$ and $\displaystyle \int\limits_{0}^{1} \dfrac{1}{x^2}$ converge or diverge simultaneously.\\
Therefore, as $\displaystyle \int\limits_{0}^{1} \dfrac{1}{x^2} \dif x$ diverges, $\displaystyle \int\limits_{0}^{1} \dfrac{\arctan x}{x^2} \dif x$ also diverges.

\section{$\displaystyle \Gamma(x) = \int\limits_{0}^{\infty} t^{x - 1} e^{-t} \dif t$}

\subsection{}

\begin{align*}
	\Gamma(1) &= \int\limits_{0}^{\infty} t^0 e^{-t} \dif t\\
	&= \lim\limits_{a \to \infty} \int\limits_{0}^{a} e^{-t} \dif t\\
	&= \lim\limits_{a \to \infty} \left. -e^{-t} \right\rvert_{0}^{a}\\
	&= \lim\limits_{a \to \infty} -e^{-a} - (- e^0)\\
	&= \lim\limits_{a \to \infty} -e^{-a} + 1\\
	&= 1
\end{align*}

%\subsection{}
%
%\begin{align*}
%	\Gamma(x + 1) &= \int\limits_{0}^{\infty} t^x e^{-t} \dif t\\
%	&= \lim\limits_{a \to \infty} \int\limits_{0}^{a} t^x r^{-t} \dif t\\
%	&= \lim\limits_{a \to \infty} \left. t^x \int e^{-t} \dif t - \int t^x \ln t \left( \int e^{-t} \dif t \right) \dif t \right\rvert_{0}^{a}\\
%	&= \lim\limits_{a \to \infty} \left. \right. -t^x e^t - \int
%\end{align*}

%\end{multicols}

\end{document}