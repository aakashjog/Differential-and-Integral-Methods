\documentclass[fleqn, a4paper, 10pt]{article}
\usepackage[top=1in, left=1in, right=1in, bottom=1in]{geometry}
\usepackage{amsmath, amssymb, esdiff}
\usepackage{gensymb}
\usepackage{tikz, pgfplots}
\usepackage[nodayofweek]{datetime}
\usepackage{ulem}
\usepackage{enumerate}
\setcounter{secnumdepth}{4}
\newcommand\numberthis{\addtocounter{equation}{1}\tag{\theequation}}
\renewcommand{\thesubsection}{(\alph{subsection})}
\renewcommand{\thesection}{(\arabic{section})}


%opening
\title{Differential and Integral Methods - Exercise 1}
\author{Aakash Jog}
\date{\formatdate{5}{11}{2014}}

\begin{document}
	
\maketitle
\setlength{\mathindent}{0pt}

%\tableofcontents

\section{Find the domain of the following functions}

\subsection{$f(x) = 2 x - 3 \sin x$}

There is no restriction on the values of $x$ in $f(x)$.\\
\begin{equation*}
	\boxed{	\therefore D(f) = \mathbb{R}}
\end{equation*}

\subsection{$f(x) = \dfrac{1}{\sqrt{|x| - x}}$}

For the square root to exist, $|x| - x \geq 0$, and for the fraction to exist, $\sqrt{|x| - x} \neq 0$.
\begin{align*}
	&\therefore |x| - x > 0\\
	&\therefore x \in (-\infty, 0)
\end{align*}
\begin{equation*}
	\boxed{	\therefore D(f) = (-\infty, 0)}
\end{equation*}

\subsection{$f(x) = \sqrt{\dfrac{x + 3}{|x^2 - 4|}}$}

Conditions on $x$:
\begin{align}
	\dfrac{x + 3}{|x^2 - 4|} \geq 0 \label{condition 1}\\ 
	|x^2 - 4| \neq 0 \label{condition 2}
\end{align}

\begin{align*}
	\eqref{condition 1} \Rightarrow x + 3 &\geq 0\\
	\therefore x &\geq -3 \\
	\eqref{condition 2} \Rightarrow x^2 &\neq 4 \\
	\therefore x &\notin \{-2,2\}
\end{align*}

\begin{equation*}
	\boxed{\therefore D(f) = [-3, \infty) - \{-2, 2\}}
\end{equation*}

\subsection{$f(x) = \ln(x+2)$}

Conditions on $x$:
\begin{align}
	x + 2 > 0
\end{align}
\begin{align*}
	\therefore x > -2
\end{align*}
\begin{equation*}
	\boxed{\therefore D(f) = (-2, \infty)}
\end{equation*}

\subsection{$f(x) = \ln(|x+2|)$}

Conditions on $x$:
\begin{align}
	|x + 2| > 0
\end{align}
\begin{equation*}
	\therefore x \neq -2
\end{equation*}
\begin{equation*}
	\boxed{\therefore D(f) = \mathbb{R} - \{-2\}}
\end{equation*}

\subsection{$f(x) = \log_2 (\log_2 x)$}

Conditions on $x$:
\begin{align}
	x &> 0\\
	\log_2 x &> 0
\end{align}
\begin{align*}
	\therefore x &> 1
\end{align*}
\begin{equation*}
	\boxed{\therefore D(f) = (1, \infty)}
\end{equation*}

\subsection{$\dfrac{1}{1 - \cos x}$}

Conditions on $x$:
\begin{align}
	1 - \cos x \neq 0\\
\end{align}
\begin{align*}
	\therefore \cos x \neq 1\\
	\therefore x \neq 2n\pi ; n \in Z
\end{align*}
\begin{equation*}
	\boxed{\therefore D(f) = \mathbb{R} - \{2n\pi : n \in Z\}}
\end{equation*}

\subsection{$f(x) = \dfrac{1 + \sin x}{\sqrt{1 - sin^2 x}}$}

\begin{align*}
	f(x) &= \dfrac{1 + \sin x}{\sqrt{1 - \sin^2 x}}\\
	&= \dfrac{1 + \sin x}{\sqrt{\cos^2 x}}\\
	&= \dfrac{1 + \sin x}{\cos x}\\
\end{align*}
Therefore, conditions on $x$:
\begin{align}
	\cos x \neq 0
\end{align}
\begin{align*}
	\therefore x \neq (2n+1)\dfrac{\pi}{2}
\end{align*}
\begin{equation*}
	\boxed{\therefore x \in \mathbb{R} - \{(2n+1)\dfrac{\pi}{2} : n \in \mathbb{Z}\}}
\end{equation*}

\subsection{$f(x) = \sqrt{\dfrac{x - 2}{x + 2}} + \sqrt{\dfrac{1 + x}{1 - x}}$}

Conditions on $x$:
\begin{align}
	\dfrac{x - 2}{x + 2} \geq 0 \label{sqrt + sqrt : condition 1}\\
	x + 2 \neq 0 \label{sqrt + sqrt : condition 2}\\
	\dfrac{1 + x}{1 - x} \geq 0 \label{sqrt + sqrt : condition 3}\\
	1 - x \neq 0 \label{sqrt + sqrt : condition 4}
\end{align}
\begin{align*}
	\therefore \eqref{sqrt + sqrt : condition 1} \Rightarrow x &\in (\infty, -2] \cup [2, \infty)\\
	\therefore \eqref{sqrt + sqrt : condition 2} \Rightarrow x &\neq -2\\
	\therefore \eqref{sqrt + sqrt : condition 1} \Rightarrow x &\in [-1,1]\\
	\therefore \eqref{sqrt + sqrt : condition 2} \Rightarrow x &\neq 1\\
\end{align*}
\begin{equation*}
	\boxed{D(f) = \{\}}
\end{equation*}

\subsection{$f(x) = \tan \left( x + \dfrac{\pi}{4} \right) $}

Conditions on $x$:
\begin{align}
	x + \dfrac{\pi}{4} \neq (2n+1)\dfrac{\pi}{2} ; n \in \mathbb{Z}
\end{align}
\begin{align*}
	\therefore x \neq (2n+1)\dfrac{\pi}{2} - \dfrac{\pi}{4}; n \in \mathbb{Z}
\end{align*}
\begin{equation*}
	\boxed{\therefore D(f) = \mathbb{R} - \{(2n+1)\dfrac{\pi}{2} - \dfrac{\pi}{4}: n \in \mathbb{Z}\}}
\end{equation*}

\newpage
\section{Write which of the following functions are even, odd or neither odd nor even:}

\subsection{$f(x) = x - x^3 +x^5$}

\begin{align*}
	f(-x) &= (-x) - (-x)^3 + (-x)^5\\
	&= (-x) - (-x^3) + (-x^5)\\
	&= -x + x^3 - x^5\\
	&= -f(x)
\end{align*} 
Therefore, $f(x)$ is odd.

\subsection{$f(x) = x^2 - x^3 + x^6$}

\begin{align*}
	f(-x) &= (-x)^2 - (-x)^3 + (-x)^6\\
	&= x^2 - (-x^3) + x^6\\
	&= x^2 + x^3 + x^6
\end{align*}
Therefore, $f(x)$ is neither odd nor even.

\subsection{$f(x) = 5^x$}

\begin{align*}
	f(-x) &= 5^{-x}\\
	&= \dfrac{1}{5^x}
\end{align*}
Therefore, $f(x)$ is neither odd nor even.

\subsection{$f(x) = \sin(\sin x)$}

\begin{align*}
	f(-x) &= \sin(\sin (-x))\\
	&= \sin(-\sin x)\\
	&= - \sin(\sin x)\\
	&= -f(x)
\end{align*}
Therefore, $f(x)$ is odd.

\subsection{$f(x) = \cos(\sin x)$}

\begin{align*}
	f(-x) &= \cos(\sin (-x))\\
	&= \cos(-\sin x)\\
	&= \cos(\sin x)\\
	&= f(x)
\end{align*}
Therefore, $f(x)$ is even.

\subsection{$f(x) = \sin(\cos x)$}

\begin{align*}
	f(-x) &= \sin(\cos (-x))\\
	&= \sin(\cos x)\\
	&= f(x)
\end{align*}
Therefore, $f(x)$ is even.

\subsection{$f(x) = x \sin x$}

\begin{align*}
	f(-x) &= (-x) \sin (-x)\\
	&= (-x)(-sin x)\\
	&= x \sin x
	&= f(x)
\end{align*}
Therefore, $f(x)$ is even.

\subsection{$f(x) = \dfrac{1 - x}{1 + x^2}$}

\begin{align*}
	f(-x) &= \dfrac{1 - (-x)}{1 + (-x)^2}\\
	&= \dfrac{1 + x}{1 + x^2}
\end{align*}
Therefore, $f(x)$ is neither odd nor even.

\newpage
\section{Find the image of the following functions in the specified domains:}

\subsection{$f(x) = \sin x + 3$ in the interval $\left[\pi, \dfrac{3\pi}{2}\right]$}

The image of $\sin x$ in $\left[ \pi, \dfrac{3\pi}{2}\right] $ is $\left[-1, 0\right]$.\\
$\therefore$ the image of $f(x)$ is $\left[(-1+3), (0+3)\right]$, i.e. $\left[2,3\right].$

\subsection{$f(x) = -x^2 + 8$ in the interval $(0,7]$}

The image of $x^2$ in $(0, 7]$ is $(0, 49]$.\\
$\therefore$ the image of $-x^2$ in $(0, 7]$ is $[-49, 0)$.\\
$\therefore$ the image of $f(x)$ in $(0, 7]$ is $[-41, 8)$.

\section{Let $f(x) = x^3$. Find the function obtained from performing the following actions on $f$:
}

\subsection{Translation downwards by 3}

\begin{align*}
	g(x) &= f(x) - 3\\
	\therefore g(x) &= x^3 - 3
\end{align*}

\subsection{Right translation by 4}

\begin{align*}
	g(x) &= f(x-4)\\
	\therefore g(x) &= (x-4)^3
\end{align*}

\subsection{Reflection around the $x$-axis.}

\begin{align*}
	g(x) &= -f(x)\\
	\therefore g(x) &= -x^3
\end{align*}

\subsection{Reflection around the $y$-axis.}

\begin{align*}
	g(x) &= f(-x)\\
	\therefore g(x) &= (-x)^3 = -x^3
\end{align*}

\subsection{Left translation by 1}

\begin{align*}
g(x) &= f(x+1)\\
\therefore g(x) &= (x+1)^3\\
\therefore g(x) &= x^3 + 3x^2 + 3x +1
\end{align*}

\section{Draw the graph of the function $f(x) = x - [x]$, where $[x]$ is the biggest integer smaller or equal to $x$.}

\vspace{3in}

\section{Solve the following inequalities:}

\subsection{$x^2 - 5x + 6 \leq 0$}

\begin{align*}
	x^2 - 5x + 6 &\leq 0\\
	\therefore (x-2)(x-3) &\leq 0
\end{align*}
\begin{equation*}
	\therefore x \in [2, 3] 
\end{equation*}

\subsection{$x^2 - 4x > 21$}

\begin{align*}
	x^2 - 4x &> 21\\
	\therefore x^2 -4x - 21 &> 0\\
	\therefore (x-7)(x+3) &> 0
\end{align*}
\begin{equation*}
	\therefore x \in (-\infty, -3) \cup (7, \infty)
\end{equation*}

\subsection{$|-x +  3| < 7$}

\begin{align*}
	|-x + 3| &< 7\\
	\therefore -7 &< -x + 3 < 7\\
	\therefore -10 &< -x < 4\\
	\therefore 10 &> x > -4 
\end{align*}
\begin{equation*}
	\therefore x \in (-4, 10)
\end{equation*}

\subsection{$|2x - 3| \leq x+3$}

\begin{align*}
	|2x - 3| &\leq x + 3\\
	\therefore -x - 3 &\leq 2x - 3 \leq x + 3\\
	\therefore 0 \leq x \leq 6
\end{align*}

\subsection{$|x + 1| - |2 - x| \geq 0$}

\begin{align*}
	|x + 1| &\geq |2 - x|\\
	\therefore (x + 1)^2 &\geq (2 - x)^2 \\
	\therefore x^2 +2x + 1 &\geq 4 - 4x + x^2\\
	\therefore 6x &\geq 3\\
	\therefore x &\geq \dfrac{1}{2}
\end{align*}
\begin{equation*}
	\therefore x \in \left[\dfrac{1}{2}, \infty\right)
\end{equation*}

\subsection{$|x - 1| + |x - 2| - |x - 3| \geq x$}

\subsubsection*{Case I: $x > 3$}
\begin{align*}
	\therefore (x - 1) + (x - 2) - (x - 3) -x &\geq 0
\end{align*}
No contradiction\\
\begin{equation*}
	\therefore x \in (3, \infty)
\end{equation*}

\subsubsection*{Case II: $2 < x \leq 3$}

\begin{align*}
	(x - 1) + (x - 2) + (x - 3) - x \geq 0\\
	\therefore x \geq 3
\end{align*}
\begin{equation*}
	\therefore x = 3
\end{equation*}

\subsection*{Case III: $1 < x \leq 2$}

\begin{align*}
	(x - 1) - (x - 2) + (x - 3) - x \geq 0\\
	\therefore -2 \geq 0
\end{align*}
$\therefore$ no values of $x$ exist.

\subsubsection*{Case IV: $x \leq 1$}

\begin{align*}
	-(x - 1) - (x - 2) + (x - 3) - x \geq 0\\
	\therefore x \leq 0
\end{align*}
\begin{equation*}
	\therefore x \in (-\infty, 0]
\end{equation*}

\begin{equation*}
	\boxed{x \in (\infty, 0] \cup [3, \infty)}
\end{equation*}

\end{document}
