\documentclass[fleqn, a4paper, 12pt]{article}
\usepackage{amsmath, amssymb, amsthm, esdiff}
\usepackage[table]{xcolor}
\usepackage{commath}
\usepackage{gensymb}
\usepackage{hyperref}
\usepackage{tikz, pgfplots}
\usetikzlibrary{calc}
\usepackage{datetime}
\usepackage{setspace}
\usepackage{ulem}
\usepackage{xfrac}

\usepackage{enumerate, enumitem}

\setcounter{secnumdepth}{4}

\newcommand\numberthis{\addtocounter{equation}{1}\tag{\theequation}}

\theoremstyle{definition}
\newtheorem{example}{Example}
\newtheorem{definition}{Definition}

\theoremstyle{theorem}
\newtheorem{theorem}{Theorem}
\newtheorem{corollary}{Corollary}

\theoremstyle{remark}
\newtheorem{remark}{Remark}
\newtheorem{case}{Case}

\newenvironment{solution}
{\begin{proof}[Solution]\let\qed\relax}
	{\end{proof}}

\makeatletter
\@addtoreset{corollary}{theorem} %resets corollary numbers after a theorem
\makeatother

%opening
\title{Lecture 17}
\author{Aakash Jog}
\date{\formatdate{28}{12}{2014}}

\begin{document}
	
\maketitle
%\setlength{\mathindent}{0pt}

\tableofcontents

\newpage

\section{Partial Derivatives of Higher Order}

\subsection{Definition}

\begin{definition}[Partial Derivatives of Higher Order]
	\begin{align*}
		(f_x)_x &= f_{xx} = \dpd{}{x} \left( \dpd{f}{x} \right) = \dpd[2]{f}{x} = D_{x^2} f\\
		(f_x)_y &= f_{xy} = \dpd{}{y} \left( \dpd{f}{x} \right) = \dfrac{\partial^2 f}{\partial y \partial x} = D_{xy} f\\
		(f_y)_x &= f_{yx} = \dpd{}{x} \left( \dpd{f}{y} \right) = \dfrac{\partial^2 f}{\partial x \partial x} = D_{yx} f\\
		(f_y)_x &= f_{yy} = \dpd{}{y} \left( \dpd{f}{y} \right) = \dpd[2]{f}{y} = D_{y^2} f
	\end{align*}
\end{definition}

\begin{theorem}
	Let $x = f(x, y)$ be continuous in some open neighbourhood of $(a, b)$. Assuming $\exists f_{xy}(x, y)$ and $\exists f_{yx}(x, y)$ and are continuous in this neighbourhood, $f_{xy}(a, b) = f_{yx}(a, b)$.
\end{theorem}

\subsection{Differential}

\begin{definition}[Differentials of $f(x, y)$]
	Consider $z = f(x, y)$. Then,
	\begin{align*}
		\dif x &= \Delta x\\
		\dif y &= \Delta y\\
		\dif z &= f_x(a, b) \dif x + f_y(a, b) \dif y
	\end{align*}
	 are the differentials of $f(x, y)$ at $(a, b)$.
\end{definition}

\subsection{Differentiability}

\begin{theorem}
	$f(x, y)$ is differentiable at $(a, b)$ if $\exists \varepsilon_1(\Delta x, \Delta y)$ and $\exists \varepsilon_2(\Delta x, \Delta y)$, with
	\begin{align*}
		\lim\limits_{(\Delta x, \Delta y) \to (0, 0)} \varepsilon_1(\Delta x, \Delta y) &= 0\\
		\lim\limits_{(\Delta x, \Delta y) \to (0, 0)} \varepsilon_2(\Delta x, \Delta y) &= 0
	\end{align*}
	such that
	\begin{equation*}
		\Delta x = \dif z + \varepsilon_1(\Delta x, \Delta y) \Delta x + \varepsilon_2 (\Delta x, \Delta y) \Delta y
	\end{equation*}
\end{theorem}

\begin{theorem}
	If $f(x, y)$ is differentiable at $(a, b)$, then $f(x, y)$ is continuous at $(a, b)$. 
\end{theorem}

\begin{proof}
	If $f(x, y)$, then
	\begin{align*}
		\Delta z &= f(a + \Delta x, b + \Delta y) - f(a, b)\\
		\Delta z &= \dif z + \varepsilon_1 \Delta x + \varepsilon_2 \Delta y\\
		&= f_x(a, b) \Delta x + f_y(a, b) \Delta y + \varepsilon_1 \Delta x + \varepsilon_2 \Delta y\\
		\therefore f(a + \Delta x, b + \Delta y) &= f(a, b) \\
		&\quad+ f_x(a, b) \Delta x + f_y(a, b) \Delta y \\
		&\quad+ \varepsilon \Delta x + \varepsilon_2 \Delta y\\
	\end{align*}
	Therefore, taking limits on both sides,
	\begin{equation*}
		\lim\limits_{(\Delta x, \Delta y) \to (0, 0)} f(a + \Delta x, b + \Delta y) = f(a, b)
	\end{equation*}
\end{proof}

\begin{theorem}
	If $\exists f_x(x, y)$ and $\exists f_y(x, y)$ in some open neighbourhood of $(a, b)$ and the partial derivatives are continuous at $(a, b)$ then $f(x, y)$ is differentiable at $(a, b)$.
\end{theorem}

\begin{example}
	Given
	\begin{equation*}
		f(x, y) = 
		\begin{cases}
			\dfrac{xy}{x^2 + y^2} &;\quad (x, y) \neq (0,0)\\
			0 &;\quad (x,y) = (0,0)\\
		\end{cases}
	\end{equation*}
	\begin{enumerate}
		\item Calculate $f_x(0,0)$ and $f_y(0,0)$.
		\item Prove: $\nexists \lim\limits_{(x, y) \to (0, 0)} f(x, y)$
		\item Is the function continuous at $(0, 0)$?
		\item Is the function differentiable at $(0, 0)$?
	\end{enumerate}
\end{example}

\begin{solution}
	\begin{align*}
		f_x(0, 0) &= \lim\limits_{\Delta x \to 0} \dfrac{f(0 + \Delta x, 0) - f(0, 0)}{\Delta x}\\
		&= \lim\limits_{\Delta x \to 0} \dfrac{\dfrac{\Delta x \cdot 0}{(\Delta x)^2 + 0}}{\Delta x}\\
		&= \lim\limits_{\Delta x \to 0} \dfrac{0}{\Delta x}\\
		&= 0
	\end{align*}
	\begin{align*}
		f_y(0, 0) &= \lim\limits_{\Delta y \to 0} \dfrac{f(0, 0 + \Delta y) - f(0, 0)}{\Delta y}\\
		&= \lim\limits_{\Delta y \to 0} \dfrac{0}{\Delta y}\\
		&= 0
	\end{align*}
	\begin{align*}
		\lim\limits_{(x, y) \stackrel{y = kx}{\to} (0, 0)} f(x, y) &= \lim\limits_{x \to 0} \dfrac{x \cdot kx}{x^2 + (kx)^2}\\
		&= \lim\limits_{x \to 0} \dfrac{k x^2}{x^2 (1 + k)}\\
		&= \lim\limits_{x \to 0} \dfrac{k}{1 + k^2} \\
		&= \dfrac{k}{1 + k^2}
	\end{align*}
	As the limit depends on $k$, $\nexists \lim\limits_{(x, y) \to (0, 0)} f(x, y)$.\\
	Hence, $f(x, y)$ is not continuous at $(a, b)$.\\
	Hence, $f(x, y)$ is not differentiable at $(a, b)$.
\end{solution}

\subsection{Chain Rule}

\begin{theorem}
	Let $z = f(x, y)$ be differentiable and let there exist derivatives of $x = g(t)$ and $y = h(t)$. Then $\exists z'(t)$ and
	\begin{equation*}
		z'(t) = z_x \cdot g'(t) + z_y \cdot h'(t)
	\end{equation*}
\end{theorem}

\begin{theorem}
	Let $z = f(x, y)$ be differentiable and $x = g(s, t)$, $y = h(s, t)$. Assuming there exist $g_s$, $g_t$, $h_s$, $h_t$, 
	\begin{align*}
		z_s &= z_x \cdot x_s + z_y \cdot y_s\\
		z_t &= z_x \cdot x_t + z_y \cdot y_t
	\end{align*}
\end{theorem}

\begin{example}
	Given 
	\begin{align*}
		z &= e^x \sin y\\
		x &= s t^2\\
		y &= s^2 \cos t
	\end{align*}
	Find $z_s$.
\end{example}

\begin{solution}
	\begin{align*}
		z_s &= z_x \cdot x_s + z_y \cdot y_s\\
		&= e^x \sin y \cdot t^2 + e^x \cos y \cdot 2 s \cos t\\
		&= e^{s t^2} \sin (s^2 \cos t) \cdot t^2 + e^{s t^2} \cos(s^2 \cos t) \cdot 2 s \cos t
	\end{align*}
\end{solution}

\subsection{Derivative of an Implicit Function}

\begin{example}
	Given $F(x, y) = k$ and $y = y(x)$, find $y'(x)$.
\end{example}

\begin{solution}
	Differentiating both sides,
	\begin{align*}
		F_x x_x + F_y y_x &= 0\\
		\therefore y_x &= -\dfrac{F_x}{F_y}
	\end{align*}
\end{solution}

\begin{example}
	Given $F(x, y, z) = k$ and $z = f(x, y)$, find $z_x$ and $z_y$.
\end{example}

\begin{solution}
	Differentiating both sides,
	\begin{align*}
		F_x x_x + F_y y_x + F_z z_x &= 0\\
		\therefore z_x &= -\dfrac{F_x}{F_z}
	\end{align*}
	Similarly,
	\begin{align*}
		z_y &= -\dfrac{F_y}{F_z}
	\end{align*}
\end{solution}

\section{Gradient and the Tangent Plane to a Surface.}

\begin{definition}[Gradient of $f(x, y)$]
	If $\exists f_x(x, y)$ and $\exists f_y(x, y)$ then the vector 
	\begin{equation*}
		\nabla f(x, y) = (f_x(x, y), f_y(x, y)) \neq 0
	\end{equation*}
	is called the gradient of $f(x, y)$.\\
	$\nabla f : \mathbb{R}^2 \to \mathbb{R}^2$ is a vector function.
\end{definition}

\begin{definition}[Gradient of $f(x, y, z)$]
	Given $f(x, y, z)$,
	\begin{align*}
		\nabla f(x, y, z) &= (f_x(x, y, z), f_y(x, y, z), f_z(x, y, z)) \neq 0
	\end{align*}
	is called the gradient of $f(x, y, z)$.\\	
	$\nabla f : \mathbb{R}^3 \to \mathbb{R}^3$ is a vector function.
\end{definition}

\begin{theorem}
	If $F(x, y, z)$ is differentiable at some point $P_0 (x_0, y_0, z_0)$ on the surface, then the tangent plane $\alpha$ to the surface at the point can be calculated by the formula
	\begin{align*}
		F_x (P_0) (x - x_0) + F_y (P_0) (y - y_0) + F_z (P_0) (z - z_0) &= 0
	\end{align*}
\end{theorem}

\begin{example}
	Find the tangent plane $\alpha$ to the ellipsoid 
	\begin{equation*}
		\dfrac{x^2}{4} + y^2 + \dfrac{z^2}{9} = 3
	\end{equation*}
	at the point $P_0 (-2, 1, 3)$.
\end{example}

\section{Extrema of $f(x, y)$}

\begin{definition}[Local maximum of $f(x, y)$]
	The function $z = f(x, y)$ is said to have a local maximum at $(a, b)$, if there exists an open neighbourhood of $(a, b)$ in which $f(x, y) \leq f(a, b)$
\end{definition}

\begin{definition}[Local minimum of $f(x, y)$]
	The function $z = f(x, y)$ is said to have a local minimum at $(a, b)$, if there exists an open neighbourhood of $(a, b)$ in which $f(x, y) \geq f(a, b)$
\end{definition}

\begin{definition}[Global maximum of $f(x, y)$]
	The function $z = f(x, y)$ is said to have a global maximum at $(a, b)$, if $f(x, y) \leq f(a, b)$, $\forall (x, y) \in D$. 
\end{definition}

\begin{definition}[Global minimum of $f(x, y)$]
	The function $z = f(x, y)$ is said to have a global minimum at $(a, b)$, if $f(x, y) \geq f(a, b)$, $\forall (x, y) \in D$. 
\end{definition}

\end{document}