\documentclass[fleqn, 12pt]{article}
\setcounter{secnumdepth}{4}
\usepackage{amsmath, amssymb, amsthm}
\usepackage{commath, esdiff}
\usepackage{datetime}
\usepackage{ulem}
\usepackage{xfrac}
\usepackage{enumerate}
\usepackage{tikz}

\newcommand\numberthis{\addtocounter{equation}{1}\tag{\theequation}}

\theoremstyle{definition}
\newtheorem{example}{Example}
\newtheorem{definition}{Definition}

\theoremstyle{theorem}
\newtheorem{theorem}{Theorem}

\newenvironment{solution}
{\begin{proof}[Solution]\let\qed\relax}
	{\end{proof}}

%opening
\title{Recitation 6}
\author{}
\date{\formatdate{3}{12}{2014}}

\begin{document}

\maketitle
%\setlength{\mathindent}{0pt}

\tableofcontents

\newpage
\section{Taylor's Series}

\begin{example}
	Calculate $\sqrt[3]{29}$ with an accuracy of $10^{-3}$.
\end{example}

\begin{solution}
	\begin{align*}
		f(x) &= x^{\sfrac{1}{3}}\\
		\therefore f'(x) &= \dfrac{1}{3} \cdot x^{-\sfrac{2}{3}}\\
		\therefore f''(x) &= -\dfrac{1}{3} \cdot \dfrac{2}{3} x^{-\sfrac{5}{3}}\\
		\therefore f'''(x) &= \dfrac{1}{3} \cdot \dfrac{2}{3} \cdot \dfrac{5}{3} x^{-\sfrac{8}{3}}
		\vdots\\
		&= f^{(n)} \dfrac{1 \cdot 2 \cdot 5 \cdot 8 \cdot \dots \cdot (3(n-1) - 1)}{3^n} (-1)^{n + 1} x^{-\frac{3n - 1}{3}}
	\end{align*}
	\begin{align*}
		f(x) &= \sum_{k = 0}^{n} \dfrac{f^{(n)}(x_0)}{{k!}} (x - x_0)^k + R_n (x)\\
		\therefore \sqrt[3]{29} &= \sum_{k = 0}^{n} \dfrac{f^{(n)}}{k!} (29 - 27)^k + R_n (x)\\
		R_n (29) &= \dfrac{f^{(n+1)}(c)}{(n+1)!} (29 - 27)^{n + 1} & ;\quad 27 \leq c \leq 29
	\end{align*}
	According to the given accuracy, 
	\begin{align*}
		|R_n (x)| &< 10^{-3}\\
		\therefore \dfrac{1 \cdot 2 \cdot 5 \cdot 8 \cdot (3n - 1)}{2^{n + 1}} c^{-\frac{3n + 2}{3}} \dfrac{2^{n + 1}}{(n + 1)!} &< 10^{-3}
	\end{align*}
	At $c = 27$, $R_n (x)$ is maximum.
	\begin{align*}
		\therefore (R_n (x))_{\text{max}} &= \dfrac{1 \cdot 2 \cdot 5 \cdot 8 \cdot (3n - 1)}{2^{n + 1}} 27^{-\frac{3n + 2}{3}} \dfrac{2^{n + 1}}{(n + 1)!} 
	\end{align*}
	For $n = 2$,
	\begin{align*}
		(R_n (x))_{\text{max}} &< 10^{-3}
	\end{align*}
	Therefore,
	\begin{align*}
		\sqrt[3]{29} \approx \sqrt[3]{27} + \dfrac{1}{3} \cdot 27^{-\sfrac{2}{3}} (29 - 27) - \dfrac{1}{3} \cdot \dfrac{2}{3} \cdot \dfrac{27^{-\sfrac{5}{3} }(29 - 27)^2}{2!}
	\end{align*}
\end{solution}

\begin{example}
	Calculate $e^x$ with accuracy of $10^{-5}$ for $0 \leq x \leq 1$.
\end{example}

\begin{solution}
	\begin{align*}
		e^x &= \sum_{k = 0}^{n} \dfrac{x^k}{k!} + R_n (x)\\
		R_n (x) &= \dfrac{e^c}{(n + 1)!} x^{n + 1}
	\end{align*}
	According to the given accuracy, 
	\begin{align*}
		|R_n (x)| &\leq 10^{-5}
		\dfrac{e^x}{(n + 1)!} x^{n + 1} &\leq \dfrac{e}{(n + 1)!}\\
		&< \dfrac{3}{(n + 1)!}
		&< 10^{-5}
	\end{align*}
	Therefore, $n = 10$ satisfies the required accuracy.
	\begin{align*}
	\therefore e^x &\approx \sum_{k = 0}^{10} \dfrac{x^k}{k!}
	\end{align*}
\end{solution}



\end{document}

