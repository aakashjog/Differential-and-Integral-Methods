\documentclass[fleqn, a4paper, 12pt]{article}
\usepackage{amsmath, amssymb, amsthm, esdiff}
\usepackage[table]{xcolor}
\usepackage{commath}
\usepackage{gensymb}
\usepackage{hyperref}
\usepackage{tikz, pgfplots}
\usetikzlibrary{calc}
\usepackage{datetime}
\usepackage{setspace}
\usepackage{ulem}
\usepackage{xfrac}
\usepackage{siunitx}

\usepackage{enumerate, enumitem}

\setcounter{secnumdepth}{4}

\newcommand\numberthis{\addtocounter{equation}{1}\tag{\theequation}}

\theoremstyle{definition}
\newtheorem{example}{Example}
\newtheorem{definition}{Definition}

\theoremstyle{theorem}
\newtheorem{theorem}{Theorem}
\newtheorem{corollary}{Corollary}

\theoremstyle{remark}
\newtheorem{remark}{Remark}
\newtheorem{case}{Case}

\newenvironment{solution}
{\begin{proof}[Solution]\let\qed\relax}
	{\end{proof}}

\makeatletter
\@addtoreset{corollary}{theorem} %resets corollary numbers after a theorem
\makeatother

%opening
\title{Lecture 19}
\author{Aakash Jog}
\date{\formatdate{4}{1}{2015}}

\begin{document}
	
\maketitle
%\setlength{\mathindent}{0pt}

\tableofcontents

\newpage

\section{Lagrange Multipliers}

Assume $f(x,y,z)$ and $g(x,y,z)$ to be differentiable. In order to find maximum and minimum values of $f(x,y,z)$ subject to constraint $g(x,y,z) = k$, find the values of $x$, $y$, $z$ and $\lambda$ which satisfy
\begin{align*}
	\nabla f(x,y,z) &= \lambda \nabla g(x,y,z)\\
	g(x,y,z) &= k
\end{align*}
Calculate the values of $f$ at the corresponding points from the above equations. The largest value is the maximum value of $f$ and the smallest values is the minimum values of $f$ subject to the constraints.

\begin{example}
	A parallelepiped box without a top has to be built using a cardboard with area 12 \si{\square\metre}. Find the box with maximum volume.
\end{example}

\begin{solution}
	\begin{align*}
		V &= x y z\\
		S &= 2 x z + 2 y z + x y\\
		&= 12
	\end{align*}
	Let
	\begin{align*}
		f(x,y,z) &= x y z\\
		g(x,y,z) &= 2 x z + 2 y z + x y
	\end{align*}
	For finding maximum value,
	\begin{align*}
		\nabla f(x,y,z) &= \lambda \nabla g(x,y,z)\\
		g(x,y,z) &= k
	\end{align*}
	Therefore,
	\begin{align*}
		f_x &= \lambda g_x\\
		f_y &= \lambda g_y\\
		f_z &= \lambda g_z\\
		g &= k
	\end{align*}
	Therefore,
	\begin{align*}
		y z &= \lambda (2z + y)\\
		x z &= \lambda (2 z + x)\\
		x y &= \lambda (2x + 2y)\\
		2 x z + 2 y z + x y &= 12
	\end{align*}
	Solving,
	\begin{align*}
		x &= 2\\
		y &= 2\\
		z &= 1\\
		\therefore V &= 4 \si{\cubic\metre}
	\end{align*}
\end{solution}

\section{Double Integrals}

\subsection{Double Integrals on Rectangular Domains}

\begin{definition}[Double integral]
	Consider $z = f(x,y)$ defined on a rectangle $R = [a,b] \times [c,d]$.\\
	Let
	\begin{equation*}
		a = x_0 < \dots < x_{i-1} < x_i < \dots < x_n = b
	\end{equation*}
	\begin{equation*}
		c = y_0 < \dots < y_{j-1} < y_j < \dots < y_n = d
	\end{equation*}
	Let
	\begin{align*}
		\Delta x_i &= x_{i-1}\\
		\Delta y_i &= y_{j-1}\\
		\Delta T &= \max\{\Delta x_i, \Delta y_j\}
	\end{align*}
	Let 
	\begin{equation*}
		{P_{ij}}^* = ({x_{ij}}^*, {y_{ij}}^*) \in R_{ij} = [x_{i-1},x_i] \times [y_{j-1},y_j]
	\end{equation*}
	Let 
	\begin{align*}
		\Delta A_{ij} &= \Delta x_i \Delta y_j
	\end{align*}
	be the area of $R_{ij}$.\\
	Then, $\displaystyle \sum_{j = 1}^{m} \sum_{i = 1}^{n} f({x_{ij}}^*, {y_{ij}}^*) \Delta A_{ij}$ is called the Riemann double integral sum.\\
	The double integral of $z = f(x,y)$ over the domain of definition $R$ is the limit, if it exists, of the Riemann double integral sum, as $\Delta T \to 0$.
	\begin{equation*}
		\iint\limits_{R} f(x,y) \dif A = \lim\limits_{\Delta T \to 0} \sum_{j = 1}^{m} \sum_{i = 1}^{n} f({x_{ij}}^*, {y_{ij}}^*) \Delta A_{ij}
	\end{equation*}
	\begin{equation*}
		\iint\limits_{R} f(x,y) \dif A = \iint\limits_{R} f(x,y) \dif x \dif y = \iint\limits_{R} f(x,y) \dif x \dif y 
	\end{equation*}
\end{definition}

\subsection{Iterated Integrals}

\begin{definition}
	Assume that $z = f(x,y)$ is integrable over $R = [a,b] \times [c,d]$.
	\begin{tabular}{c c}
		Iterated Integrals & Analogy in Differential Calculus\\
		$\displaystyle A(x) = \int\limits_{c}^{d} f(x,y) \dif y$ & $f_y (x,y)$\\
		$\displaystyle I_1 = \int\limits_{a}^{b} A(x) \dif x$ & $f_{yx} (x,y)$\\
		$\displaystyle B(y) = \int\limits_{a}^{b} f(x,y) \dif x$ & $f_x (x,y)$\\
		$\displaystyle I_1 = \int\limits_{c}^{d} B(y) \dif y$ & $f_{xy} (x,y)$\\
	\end{tabular}
\end{definition}

\begin{theorem}[Fubini Theorem]
	If $f(x,y)$ is continuous on $R$, then the double integral and two iterated integrals exist, and they are equal.
	\begin{equation*}
		\iint\limits_{R} f(x,y) \dif A = \int\limits_{a}^{b} \int\limits_{c}^{d} f(x,y) \dif y \dif x = \int\limits_{c}^{d} \int\limits_{a}^{b} f(x,y) \dif x \dif y
	\end{equation*}
\end{theorem}

\begin{remark}
	Existence of two iterated integrals does not necessarily guarantee existence of the double integral.
\end{remark}

\begin{example}
	Solve 
	\[\iint\limits_{[1,2] \times [0,3]} y \cos (x + y^2) \dif A\]
\end{example}

\begin{solution}
	\begin{align*}
		\iint\limits_{[1,2] \times [0,3]} y \cos (x + y^2) \dif A &= \int\limits_{1}^{2} \int\limits_{0}^{3} y \cos (x + y^2) \dif y \dif x\\
		&= \int\limits_{1}^{2} \left( \left. \dfrac{1}{2} \sin (x + y^2) \right\rvert_{0}^{3} \right) \dif x\\
		&= \dfrac{1}{2} \int\limits_{1}^{2} \left( \sin(x + 9) - \sin x \right) \dif x\\
		&= \left. \dfrac{1}{2} \left( -\cos (x + 9) + \cos x \right) \right\rvert_{1}^{2}\\
		&= \dfrac{1}{2} (-\cos 11 + \cos 2 + \cos 10 - \cos 1)
	\end{align*}
\end{solution}

\subsection{Double Integrals on Arbitrary Domains}

\begin{definition}[Double integral on arbitrary domain]
	Let $D$ be a bounded and closed domain in $\mathbb{R}^2$ and the function $z = f(x,y)$ be defined on $D$. Consider a rectangle $R = [a,b] \times [c,d]$, s.t. $D \subset R$. Define a function $F(x,y)$, s.t.
	\begin{align*}
		F(x,y) &= 
			\begin{cases}
				f(x,y) &;\quad (x,y) \in D\\
				0 &;\quad (x,y) \notin D
			\end{cases}
	\end{align*}
	If $F(x,y)$ is integrable over $R$, it is said to be integrable on $D$
	\begin{align*}
		\iint\limits_{D} f(x,y) \dif A &= \iint\limits_{R} F(x,y) \dif A
	\end{align*}
\end{definition}

\begin{definition}[Domain of the first kind]
	A domain $D$ is said to be the domain of the first kind if there exist continuous functions $f_1(x)$ and $f_2(x)$, s.t.
	\begin{align*}
		D_{I} &= \{(x,y) | a \leq x \leq b, f_1(x) \leq y \leq f_2(x)\}
	\end{align*}
\end{definition}

\begin{theorem}
	If $f(x,y)$ is continuous in $D_{I}$, then
	\begin{align*}
		\iint\limits_{R} f(x,y) \dif A &= \int\limits_{a}^{b} \int\limits_{f_1(x)}^{f_2(x)} f(x,y) \dif y \dif x
	\end{align*}
\end{theorem}

\begin{proof}
	\begin{align*}
		\iint\limits_{D_{I}} f(x,y) \dif A &= \iint\limits_{R} F(x,y) \dif A\\
		&= \int\limits
		_{a}^{b} \int\limits_{c}^{d} F(x,y) \dif y \dif x\\
		&= \int\limits_{a}^{b} \left( \int\limits_{c}^{f_1(x)} F \dif y + \int\limits_{f_1(x)}^{f_2(x)} F \dif y + \int\limits_{f_2(x)}^{d} F \dif y \right) \dif x\\
		&= \int\limits_{a}^{b} \left( \int\limits_{c}^{f_1(x)} 0 \dif y + \int\limits_{f_1(x)}^{f_2(x)} f \dif y + \int\limits_{f_2(x)}^{d} 0 \dif y \right) \dif x\\
		&= \int\limits_{a}^{b} \int\limits_{f_1(x)}^{f_2(x)} f(x,y) \dif y \dif x
	\end{align*}
\end{proof}

\begin{definition}[Domain of the second kind]
	A domain $D$ is said to be the domain of the second kind if there exist continuous functions $g_1(y)$ and $g_2(y)$, s.t.
	\begin{align*}
	D_{II} &= \{(x,y) | c \leq y \leq d, g_1(y) \leq x \leq g_2(y)\}
	\end{align*}
\end{definition}

\begin{theorem}
	If $f(x,y)$ is continuous in $D_{II}$, then
	\begin{align*}
		\iint\limits_{R} f(x,y) \dif A &= \int\limits_{c}^{d} \int\limits_{g_1(y)}^{g_2(y)} f(x,y) \dif x \dif y
	\end{align*}
\end{theorem}

\end{document}
