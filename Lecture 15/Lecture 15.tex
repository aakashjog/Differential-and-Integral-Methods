\documentclass[fleqn, a4paper, 12pt]{article}
\usepackage{amsmath, amssymb, amsthm, esdiff}
\usepackage[table]{xcolor}
\usepackage{commath}
\usepackage{gensymb}
\usepackage{hyperref}
\usepackage{tikz, pgfplots}
\usetikzlibrary{calc}
\usepackage{datetime}
\usepackage{setspace}
\usepackage{ulem}
\usepackage{xfrac}

\usepackage{enumerate, enumitem}

\setcounter{secnumdepth}{4}

\newcommand\numberthis{\addtocounter{equation}{1}\tag{\theequation}}

\theoremstyle{definition}
\newtheorem{example}{Example}
\newtheorem{definition}{Definition}

\theoremstyle{theorem}
\newtheorem{theorem}{Theorem}
\newtheorem{corollary}{Corollary}

\theoremstyle{remark}
\newtheorem{remark}{Remark}
\newtheorem{case}{Case}

\newenvironment{solution}
{\begin{proof}[Solution]\let\qed\relax}
	{\end{proof}}

\makeatletter
\@addtoreset{corollary}{theorem} %resets corollary numbers after a theorem
\makeatother

%opening
\title{Lecture 15}
\author{Aakash Jog}
\date{\formatdate{18}{12}{2014}}

\begin{document}
	
\maketitle
%\setlength{\mathindent}{0pt}

\tableofcontents

\newpage

\section{Applications of Definite Integrals}

\subsection{Centre of Mass}

\begin{example}
	Find the centre of mass of a metal rod $[a, b]$ with density $\rho(x)$ and mass $m$.
\end{example}

\begin{solution}
	Dividing the rod into $n$ parts, from $x_0 = a$ to $x_n = b$, and assuming that the mass $\Delta m_i$ is concentrated at $c_i$, we have a system of masses $\Delta m_1, \dots, \Delta m_n$.
	\begin{align*}
		\therefore x_{\text{COM}} &\approx \dfrac{\sum\limits_{i = 1}^{n} c_i \Delta m_i}{\sum\limits_{i = 1}^{n} \Delta m_i}\\
		\therefore x_{\text{COM}} &= \lim\limits_{\Delta T \to 0} \dfrac{\sum\limits_{i = 1}^{n} c_i \rho(c_i) \Delta x_i}{\sum\limits_{i = 1}^{n} \rho(c_i) \Delta x_i}\\
		&= \dfrac{\int\limits_{a}^{b} x \rho(x) \dif x}{\int\limits_{a}^{b}\rho(x) \dif x}
	\end{align*}
\end{solution}

\section{Improper Integrals}

\begin{definition}[Improper integral of the first kind]
	Assume that there exists the integral 
	\begin{align*}
	I(t) &= \int\limits_{a}^{t}f(x) \dif x &;\quad t \geq a
	\end{align*}
	If $\exists I = \lim\limits_{t \to \infty} I(t)$, then $I$ is called an improper integral of the first kind. The improper integral is said to converge. $I$ is denoted as
	\begin{equation*}
		I = \int\limits_{a}^{+\infty} f(x) \dif x = \lim\limits_{t \to +\infty} \int\limits_{a}^{t} f(x) \dif x
	\end{equation*}
	Otherwise, the improper integral is said to diverge.\\
	Similarly for
	\begin{equation*}
		I = \int\limits_{-\infty}^{a} f(x) \dif x = \lim\limits_{t \to -\infty} \int\limits_{t}^{a} f(x) \dif x
	\end{equation*}
	If both \[\int\limits_{-\infty}^{a} f(x) \dif x\] and \[\int\limits_{a}^{+\infty} f(x) \dif x\] converge, then
	\begin{equation*}
		\int\limits_{-\infty}^{+\infty} f(x) \dif x = \int\limits_{-\infty}^{a} f(x) \dif x + \int\limits_{a}^{+\infty} f(x) \dif x
	\end{equation*}
\end{definition}

\begin{example}
	\begin{equation*}
		\int\limits_{1}^{+\infty} \dfrac{1}{x^2} \dif x
	\end{equation*}
\end{example}

\begin{solution}
	\begin{align*}
		\int\limits_{1}^{+\infty} \dfrac{1}{x^2} \dif x &= \lim\limits_{t \to +\infty} \int\limits_{1}^{t} \dfrac{1}{x^2} \dif x\\
		&= \lim\limits_{t \to +\infty} \left. \left( -\dfrac{1}{x} \right) \right\rvert_{1}^{t}\\
		&= \lim\limits_{t \to +\infty} \left( -\dfrac{1}{t} + 1 \right)\\
		&= 1
	\end{align*}
	Hence, the integral converges.
\end{solution}

\begin{example}
	\begin{equation*}
		\int\limits_{1}^{+\infty}\dfrac{1}{x} \dif x 
	\end{equation*}
\end{example}

\begin{solution}
	\begin{align*}
		\int\limits_{1}^{+\infty}\dfrac{1}{x} \dif x &= \lim\limits_{t \to +\infty} \int\limits_{1}^{t} \dfrac{1}{x} \dif x\\
		&= \lim\limits_{t \to +\infty} \left. (\ln x) \right\rvert_{1}^{t}\\
		&= \lim\limits_{t \to +\infty} (\ln t - \ln 1)\\
		&= +\infty
	\end{align*}
	Hence, the integral diverges.
\end{solution}

\begin{theorem}[Direct comparison test]\label{Direct comparison test}
	Let $f(x)$ and $g(x)$ be two functions defined on $[a, +\infty)$ and Riemann integrable over $[a, t], \forall t \geq a$. Assume that $\exists b \geq a$, s.t. $f(x) \geq g(x) \geq 0, \forall x \geq b$. Then,
	\begin{enumerate}
		\item if $\int\limits_{a}^{+\infty} f(x) \dif x$ converges, then $\int\limits_{a}^{+\infty} g(x) \dif x$ converges.
		\item if $\int\limits_{a}^{+\infty} g(x) \dif x$ diverges, then $\int\limits_{a}^{+\infty} f(x) \dif x$ diverges.
	\end{enumerate}
\end{theorem}

\begin{example}
	Show that \[\int\limits_{0}^{+\infty} e^{-x^2} \dif x\]	converges.
\end{example}

\begin{solution}
	If
	\begin{align*}
		x &\geq 1\\
		\implies x^2 &\geq x\\
		\implies -x^2 &\leq -x
	\end{align*}
	Therefore, $\forall x \geq 1$
	\begin{align*}
		0 \leq e^{-x^2} \leq e^{-x}
	\end{align*}
	\begin{align*}
		\int\limits_{0}^{+\infty} e^{-x} \dif x &= \lim\limits_{t \to +\infty} \int\limits_{0}^{t} e^{-x} \dif x\\
		&= \lim\limits_{t \to +\infty} (-e^{-t} + 1)\\
		&= 1
	\end{align*}
	Hence, the integral converges.\\
	Therefore, by the \nameref{Direct comparison test}, 
	\[\int\limits_{0}^{+\infty} e^{-x^2} \dif x\] also converges.
\end{solution}

\begin{example}
	Show that \[\int\limits_{1}^{+\infty} \dfrac{1 + e^{-x}}{x} \dif x\] diverges.
\end{example}

\begin{solution}
	$\forall x \geq 1$,
	\begin{align*}
		\dfrac{1 + e^{-x}}{x} \geq \dfrac{1}{x} \geq 0
	\end{align*}
	and \[\int\limits_{1}^{+\infty} \dfrac{1}{x} \dif x\] diverges.\\
	Therefore, by the \nameref{Direct comparison test},	\[\int\limits_{1}^{+\infty} \dfrac{1 + e^{-x}}{x} \dif x\]
	also diverges.
\end{solution}

\begin{definition}[Absolute integrability]
	$f(x)$ is said to be absolutely integrable over $[a, +\infty)$ if the improper integral
	\[\int\limits_{a}^{+\infty} |f(x)| \dif x\]
	converges.
\end{definition}

\begin{theorem}
	If there exists the definite integral \[\int\limits_{a}^{t} f(x) \dif x, \forall t \geq a\] and $f(x)$ is absolutely integrable over $[a, +\infty)$, then \[\int\limits_{a}^{+\infty} f(x) \dif x\] converges.
\end{theorem}

\begin{definition}[Improper integral of the second kind]
	Assume that $f(x)$ is defined and is not bounded on $[a, b)$ and assume that $\forall a \leq t \leq b$, there exists the definite integral
	\begin{align*}
		I(t) &= \int\limits_{a}^{t} f(x) \dif x
	\end{align*}
	If there exists the limit
	\begin{align*}
		I &= \lim\limits_{t \to b^-} I(t)
	\end{align*}
	then $I$ is called an improper integral of the second kind of $f(x)$ on $[a, b)$ and is denoted as
	\begin{equation*}
		I = \lim\limits_{t \to b^-} \int\limits_{a}^{t} f(x) \dif x = \int\limits_{a}^{b} f(x) \dif x
	\end{equation*}
	Similarly if $f(x)$ is not bounded at $a$, 
	\begin{equation*}
		\lim\limits_{t \to a^+} \int\limits_{t}^{a} f(x) \dif x = \int\limits_{a}^{b} f(x) \dif x
	\end{equation*}
	If $f(x)$ is not bounded at $c$, s.t. $a < c < b$, and \[\exists \int\limits_{a}^{c} f(x) \dif x\] and \[\exists \int\limits_{c}^{b} f(x) \dif x\] then, 
	\begin{equation*}
		\exists \int\limits_{a}^{b} f(x) \dif x = \int\limits_{a}^{c} f(x) \dif x + \int\limits_{c}^{b} f(x) \dif x
	\end{equation*}
\end{definition}

\begin{example}
	\begin{equation*}
		\int\limits_{0}^{3} \dfrac{\dif x}{x - 1}
	\end{equation*}
\end{example}

\begin{solution}
	\begin{align*}
		\int\limits_{0}^{1} \dfrac{\dif x}{x - 1} \dif x &= \lim\limits_{t \to 1^-} \int\limits_{0}^{t} \dfrac{\dif x}{x - 1}\\
		&= \lim\limits_{t \to 1^-} \left. \ln |x - 1| \right\rvert_{0}^{1}\\\\
		&= \lim\limits_{t \to 1^-} (\ln |t - 1| - \ln 1)\\
		&= -\infty
	\end{align*}
	Therefore, as \[\int\limits_{0}^{1} \dfrac{\dif x}{x - 1}\] diverges, \[\int\limits_{0}^{3} \dfrac{\dif x}{x - 1}\] also diverges.
\end{solution}

\section{Definite Integral Calculations using Taylor's Formula}

\begin{align*}
	f(x) &= f(a)\\
	&\quad + \dfrac{f'(a)}{1!} (x - a)\\
	&\quad + \dots\\
	&\quad + \dfrac{f^{(n)}(a)}{n!} (x - a)^n\\
	&\quad + \dfrac{f^{(n + 1)}(c)}{(n + 1)!} (x - a)^{n + 1}\\
	\therefore \int\limits_{a}^{b} f(x) \dif x &= f(a) (b - a)\\
	&\quad + \dfrac{f'(a)}{1!} \dfrac{(b - a)^2}{2}\\
	&\quad + \dfrac{f''(a)}{2!} \dfrac{(b - a)^3}{3}\\
	&\quad + \dots\\
	&\quad + \dfrac{f^{(n)}(a)}{n!} \dfrac{(b - a)^{n + 1}}{n + 1}\\
	&\quad + \dfrac{1}{(n + 1)!} \int\limits_{a}^{b} f^{(n + 1)}(c) (x - a)^{n + 1} \dif x
\end{align*}

\begin{equation*}
	R_n (I) = \dfrac{1}{(n + 1)!} \int\limits_{a}^{b} f^{(n + 1)}(c) (x - a)^{n + 1} \dif x
\end{equation*}
is called the integral Lagrange remainder.

\begin{example}
	Calculate \[\int\limits_{0}^{1} e^{x^2} \dif x\] with accuracy 0.01.
\end{example}

\begin{solution}
	For $0 < c < x^2 \leq 1$,
	\begin{align*}
		e^{x^2} &= 1 + \dfrac{x^2}{1!} + \dots + \dfrac{x^2n}{n!} + \dfrac{x^{2n + 2}}{(n + 1)!} e^c\\
		\therefore \int\limits_{0}^{1} e^{x^2} \dif x &= 1 + \dfrac{1}{1!} \cdot \dfrac{1}{3} + \dfrac{1}{2!} \cdot \dfrac{1}{5} + \dots + \dfrac{1}{n!} \cdot \dfrac{1}{2n + 1} \\
		&\quad + \dfrac{1}{(n + 1)!} \int\limits_{0}^{1} e^c x^{2n + 2} \dif x
	\end{align*}
	For $n = 4$,
	\begin{equation*}
		|R_n(I)| \leq 0.01
	\end{equation*}
	Therefore,
	\begin{align*}
		\int\limits_{0}^{1} e^{x^2} \dif x &\approx 1 + \dfrac{1}{1!} \cdot \dfrac{1}{3} + \dfrac{1}{2!} \cdot \dfrac{1}{5} + \dfrac{1}{3!} \cdot \dfrac{1}{7} + \dfrac{1}{4!} \cdot \dfrac{1}{9}
	\end{align*}
\end{solution}

\end{document}