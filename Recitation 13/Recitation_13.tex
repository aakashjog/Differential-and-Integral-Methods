\documentclass[fleqn, 12pt]{article}
\setcounter{secnumdepth}{4}
\usepackage{amsmath, amssymb, amsthm}
\usepackage{commath, esdiff}
\usepackage{datetime}
\usepackage{graphicx, epstopdf}
\usepackage{ulem}
\usepackage{xfrac}
\usepackage{enumerate}
\usepackage{tikz}

\newcommand\numberthis{\addtocounter{equation}{1}\tag{\theequation}}

\theoremstyle{definition}
\newtheorem{example}{Example}
\newtheorem{definition}{Definition}

\theoremstyle{theorem}
\newtheorem{theorem}{Theorem}

\newenvironment{solution}
{\begin{proof}[Solution]\let\qed\relax}
	{\end{proof}}

%opening
\title{Recitation 13}
\author{Aakash Jog}
\date{\formatdate{18}{1}{2015}}

\begin{document}

\maketitle
%\setlength{\mathindent}{0pt}

\tableofcontents

\newpage
\section{Extrema of Functions of Two Variables}

\begin{example}
	Find local extrema for $f(x,y) = (x - y) e^{xy}$.
\end{example}

\begin{solution}
	\begin{align*}
		f_x(x,y) &= e^{xy} + (x - y) e^{xy} \cdot y\\
		f_y(x,y) &= -e^{xy} + (x- y) e^{xy} \cdot x
	\end{align*}
	Solving for $f_x = 0$ and $f_y = 0$, the critical points are $\left( \dfrac{1}{\sqrt{2}}, -\dfrac{1}{\sqrt{2}} \right)$, $\left( -\dfrac{1}{\sqrt{2}}, \dfrac{1}{\sqrt{2}} \right)$.
	\begin{align*}
		\Delta \left( \dfrac{1}{\sqrt{2}}, -\dfrac{1}{\sqrt{2}} \right) < 0\\
		\Delta \left( -\dfrac{1}{\sqrt{2}}, \dfrac{1}{\sqrt{2}} \right) < 0
	\end{align*}
	Therefore, both points are not local extrema but saddle points.
\end{solution}

\end{document}
