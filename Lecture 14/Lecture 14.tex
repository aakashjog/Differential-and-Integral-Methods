\documentclass[fleqn, a4paper, 12pt]{article}
\usepackage{amsmath, amssymb, amsthm, esdiff}
\usepackage[table]{xcolor}
\usepackage{commath}
\usepackage{gensymb}
\usepackage{hyperref}
\usepackage{tikz, pgfplots}
\usetikzlibrary{calc}
\usepackage{datetime}
\usepackage{setspace}
\usepackage{ulem}
\usepackage{xfrac}

\usepackage{enumerate, enumitem}

\setcounter{secnumdepth}{4}

\newcommand\numberthis{\addtocounter{equation}{1}\tag{\theequation}}

\theoremstyle{definition}
\newtheorem{example}{Example}
\newtheorem{definition}{Definition}

\theoremstyle{theorem}
\newtheorem{theorem}{Theorem}
\newtheorem{corollary}{Corollary}

\theoremstyle{remark}
\newtheorem{remark}{Remark}
\newtheorem{case}{Case}

\newenvironment{solution}
{\begin{proof}[Solution]\let\qed\relax}
	{\end{proof}}

\makeatletter
\@addtoreset{corollary}{theorem} %resets corollary numbers after a theorem
\makeatother

%opening
\title{Lecture 14}
\author{Aakash Jog}
\date{\formatdate{16}{12}{2014}}

\begin{document}
	
\maketitle
%\setlength{\mathindent}{0pt}

\tableofcontents

\newpage

\section{Applications of Definite Integrals}

\subsection{Area under a curve}

An area $S$ of the region which is situated between two functions $f(x)$ and $g(x)$ over $[a, b]$ is
\begin{align*}
	S &= \int\limits_{a}^{b} |f(x) - g(x)| \dif x
\end{align*}

\begin{example}
	Calculate the area $S$ of the region which is between $y = \cos x$ and $y = \sin x$ over $[0, \sfrac{\pi}{3}]$.
\end{example}

\begin{solution}
	\begin{align*}
		S &= \int\limits_{0}^{\sfrac{\pi}{3}} |\cos x - \sin x| \dif x
	\end{align*}
	\begin{figure}[h]
		\begin{tikzpicture}[scale=3]
			\draw [->] (0,0) -- (0,2) node [above] {$y$};
			\draw [->] (0,0) -- (2,0) node [right] {$x$};
		
			\draw plot [domain = 0:pi/3] (\x, {cos(\x r)});
			\draw plot [domain = 0:pi/3] (\x, {sin(\x r)});
			
			\draw [dashed] (pi/4, 0) -- (pi/4, 2) node [left] {$x = \sfrac{\pi}{4}$};
			\draw [dashed] (pi/3, 0) -- (pi/3, 2) node [right] {$x = \sfrac{\pi}{3}$};
		\end{tikzpicture}
	\end{figure}
	Therefore,
	\begin{align*}
		S &= \int\limits_{0}^{\sfrac{\pi}{4}} (\cos x - \sin x) \dif x + \int\limits_{\sfrac{\pi}{4}}^{\sfrac{\pi}{3}} (\sin x - \cos x) \dif x\\
		&= 2\sqrt{2} - \dfrac{3 + \sqrt{3}}{2}
	\end{align*}
\end{solution}

\subsection{Length of a curve}

\begin{example}
	Calculate the length $L$ of a curve which is given by a differentiable function $y = f(x)$ over $[a, b]$.
\end{example}

\begin{solution}
	\begin{align*}
		L &= \lim\limits_{\Delta T \to 0} \sum_{i = 1}^{n} |P_{i-1} P_i|\\
		&= \lim\limits_{\Delta T \to 0} \sum_{i = 1}^{n} \sqrt{(\Delta x_i)^2 + (\Delta y_i)^2}\\
		\intertext{By Lagrange Theorem for $[x_{i-1}, x_i]$,}
		\Delta y_i &= f(x_i) - f(x_{i-1})\\
		&= f'(c_i) \Delta x_i\\
		\therefore L &= \lim\limits_{\Delta T \to 0} \sum_{i = 1}^{n} \sqrt{1 + (f'(c_i))^2} \Delta x_i\\
		&= \int\limits_{a}^{b} \sqrt{1 + (f'(x))^2} \dif x
	\end{align*}
\end{solution}

\begin{example}
	Find the length of the curve $y = f(x) = x^{\sfrac{3}{2}}$ on $[1, 4]$.
\end{example}

\begin{solution}
	\begin{align*}
		L &= \int\limits_{a}^{b} \sqrt{1 + (f'(x))^2} \dif x\\
		\therefore L &= \int\limits_{1}^{4} \sqrt{1 + \left( \dfrac{3}{2} x^{\sfrac{1}{2}} \right)^2} \dif x\\
		&= \int\limits_{1}^{4} \sqrt{1 + \dfrac{9}{4} x} \dif x\\
		&= \dfrac{4}{9} \cdot \dfrac{\left( 1 + \dfrac{9}{4} (4) \right)^{\sfrac{3}{2}}}{\sfrac{3}{2}} - \dfrac{4}{9} \cdot \dfrac{\left( 1 + \dfrac{9}{4} (1) \right)^{\sfrac{3}{2}}}{\sfrac{3}{2}}\\
		&= \dfrac{8}{27} \left( 10^{\sfrac{3}{2}} - (\sfrac{13}{4})^{\sfrac{3}{2}} \right)
	\end{align*}
\end{solution}

\subsection{Volume of solids}

\begin{definition}[Volume of a cylindrical solid]
	The volume of a cylindrical solid with a base $S$ and a height $h$ is equal to $V = S h$.
\end{definition}

For a solid situated between two planes $x = a$ and $x = b$ with a cross-sectional area $S(t)$ at any slice of the solid with the plane $x = t$, $t \in [a, b]$, 
\begin{align*}
	V &= \lim\limits_{\Delta T \to 0} \sum_{i = 1}^{n} S(c_i) \Delta x_i\\
	&= \int\limits_{a}^{b} S(x) \dif x
\end{align*}

\begin{example}
	Find the volume of a sphere.
\end{example}

\begin{solution}
	\begin{align*}
		V &= \int\limits_{-R}^{R} \pi (R^2 - x^2) \dif x\\
		&= 2 \pi \int\limits_{-R}^{R} (R^2 - x^2) \dif x\\
		&= 2 \pi \left( R^3 - \dfrac{R^3}{3} \right)\\
		&= \dfrac{4}{3} \pi R^3
	\end{align*}
\end{solution}

\subsubsection{Volume of solids of revolution about $x$-axis}

If the graph of $y = f(x)$ is rotated about the $x$-axis,
\begin{align*}
	V &= \int\limits_{a}^{b} S(x) \dif x\\
	&= \int\limits_{a}^{b} \pi y^2 \dif x\\
	&= \pi \int\limits_{a}^{b} (f(x))^2 \dif x
\end{align*}

\begin{example}
	Find the volumes of solids of revolution which are obtained by rotating $y = f(x) = \sqrt{x}$, $x \in [0,4]$ about the $x$-axis and the $y$-axis.
\end{example}

\begin{solution}
	\begin{align*}
		V_x &= \pi \int\limits_{a}^{b} (f(x))^2 \dif x\\
		&= \pi \int\limits_{0}^{4} (\sqrt{x})^2 \dif x\\
		&= 8 \pi
	\end{align*}
	\begin{align*}
		V_y &= \pi \int\limits_{c}^{d} (f(y))^2 \dif y\\
		&= \pi \int\limits_{0}^{2} (y^2)^2 \dif y\\
		&= \pi \int\limits_{0}^{2} y^4 \dif y\\
		&= \dfrac{32}{5} \pi
	\end{align*}
\end{solution}

\subsubsection{Volume of solids of revolution about $y$-axis}

The volume of a solid of revolution which is obtained by rotating about the $y$-axis of a region between $y = f(x)$, $y = 0$, $x = a$, $x = b$ is
\begin{align*}
	V &= 2 \pi \int\limits_{a}^{b} x f(x) \dif x
\end{align*}

\begin{example}
	Find the volumes of solids of revolution which are obtained by rotating the area under $y = f(x) = x(x - 1)^2$, $x \in [0, 1]$ about the $x$-axis and the $y$-axis.
\end{example}

\begin{solution}
	\begin{align*}
		V_x &= \pi \int\limits_{a}^{b} (f(x))^2 \dif x\\
		&= \pi \int\limits_{0}^{1} (x(x-1)^2)^2 \dif x\\
		&= \pi \int\limits_{0}^{1} x^2 (x-1)^2 \dif x
	\end{align*}
	\begin{align*}
		V &= 2 \pi \int\limits_{a}^{b} x f(x) \dif x\\
		&= 2 \pi \int\limits_{0}^{1} x \cdot x (x-1)^2 \dif x
	\end{align*}
\end{solution}

\end{document}
