\documentclass[fleqn]{article}
\usepackage{amsmath, amssymb, esdiff}
\usepackage{gensymb}
\usepackage{tikz, pgfplots}
\usepackage{datetime}
\usepackage{ulem}
\usepackage{enumerate}
\setcounter{secnumdepth}{4}
\newcommand\numberthis{\addtocounter{equation}{1}\tag{\theequation}}


%opening
\title{Lecture 3}
\author{}
\date{\formatdate{4}{11}{2014}}

\begin{document}
	
\maketitle
\setlength{\mathindent}{0pt}

\tableofcontents

\newpage
\section{Limits \& Continuity}

\subsection{Continuity}

If $x \rightarrow a$ then $f(x) \rightarrow L$, we say that $L$ is the \emph{limit} of $f(x)$ at $x = a$.
\begin{equation*}
	\lim\limits_{x \rightarrow a} f(x) = L
\end{equation*}
We say that $f(x)$ is \emph{continuous} at $x = a$, iff \begin{equation*}
	\lim\limits_{x \rightarrow a} f(x) = L = f(a)
\end{equation*}

\subsection{Continuity}

If $x \rightarrow a^+$ then $f(x) \rightarrow L_2$, and if $x \rightarrow a^-$ then $f(x) \rightarrow L_1$.\\
We say that $\exists \lim\limits_{x \rightarrow a} f(x)$ iff $L_1 = L_2$

\subsection{Cauchy's Definition}

Let $f(x)$ be defined on an open interval about $a$, except possibly at $a$ itself.\\
A number $L$ is called the \emph{limit} of $f(x)$ at $a$ if
\begin{equation}
	\forall \epsilon > 0 \exists \delta > 0 : 0 < |x - a| < \delta \Rightarrow |f(x) - L| < \epsilon
\end{equation}

\end{document}
